\documentclass[12pt]{book}

\usepackage[utf8]{inputenc}
\usepackage[T1]{fontenc}
\usepackage[spanish]{babel}
\usepackage[normalem]{ulem}
\usepackage[hidelinks]{hyperref}

\usepackage{listings, tcolorbox, xcolor, float}
\usepackage{algorithm, algpseudocode, chngcntr}
\usepackage{graphicx, enumitem, geometry}
\usepackage{amsmath, amssymb, amsthm}
\usepackage{forest}
\usepackage{fancyhdr}

% ----- Custom Page -----
\geometry{margin=2.5cm}
\pagestyle{fancy}
\fancyhf{}
%------------------------

% ----- Custom Commands -----
\newcommand{\nt}[1]{\texttt{<#1>}}
\newcommand{\N}{\mathbb{N}}
\newcommand{\Z}{\mathbb{Z}}
\newcommand{\B}{\mathbb{B}}
\renewcommand{\P}{\mathbb{P}}
\newcommand{\minilisp}{
  {M\footnotesize INI\normalsize L\footnotesize ISP \normalsize}
}
\renewcommand{\lstlistingname}{Código}
\renewcommand{\bibname}{Bibliografía}
%----------------------------

% ----- Book Document -----
% ----- Quitar mayúsculas automáticas -----
\renewcommand{\chaptermark}[1]{\markboth{#1}{}}
\renewcommand{\sectionmark}[1]{\markright{#1}}

% ----- Encabezados -----
\lhead{\minilisp}
\rhead{\leftmark{}.}
\cfoot{\thepage} % número centrado abajo
\renewcommand{\headrulewidth}{0.4pt}

% ----- Cambiar el título del índice -----
\addto\captionsspanish{%
  \renewcommand{\contentsname}{Índice} % cambia "Índice general" a "Índice"
}

% ----- Evitar mayúsculas en el encabezado del índice -----
\makeatletter
\renewcommand{\tableofcontents}{%
  \chapter*{\contentsname}% título "Índice"
  \markboth{\contentsname}{}% encabezado sin mayúsculas
  \@starttoc{toc}% genera el contenido real
}
\makeatother
%--------------------------

% ----- Custom Colors -----
\definecolor{azulin}{HTML}{130F87}
\definecolor{grisin}{HTML}{6D6C91}
\definecolor{links}{HTML}{0D1894}
\definecolor{mainkeywordcolor}{HTML}{610E61}
\definecolor{commentcolor}{HTML}{087008}
\definecolor{stringcolor}{HTML}{AD8318}
\definecolor{ops}{HTML}{3374A1}
\definecolor{tok}{HTML}{B39309}
\definecolor{tkns}{HTML}{560769}
\definecolor{asa}{HTML}{A81666}
\definecolor{corchets}{HTML}{A82E16}

\definecolor{secondkeywordcolor}{HTML}{0D1894}
\definecolor{thirdkeywordcolor}{HTML}{0D1894}

\definecolor{backgroundcolor}{rgb}{0.95, 0.95, 0.95}
%--------------------------

% ----- Haskell Code Style -----
\lstdefinestyle{haskellstyle}{
  language=Haskell,
  basicstyle=\ttfamily\footnotesize,
  keywordstyle=[1]\color{mainkeywordcolor}\bfseries,
  commentstyle=\color{commentcolor}\itshape,
  stringstyle=\color{stringcolor},
  numbers=left,
  numberstyle=\tiny,
  stepnumber=1,
  numbersep=8pt,
  backgroundcolor=\color{backgroundcolor},
  tabsize=4,
  showspaces=false,
  showstringspaces=false,
  breaklines=true,
  frame=single,
  captionpos=b,
  morekeywords=[1]{if,then,else,let,in,case,of,Show,Eq,
    data,type,class,instance,where, do,deriving,newtype,module,import}
}
%-------------------------------

%-------------------------------
\begin{document}

% ----- Importamos la portada -----
\begin{titlepage}
  \thispagestyle{empty}
  %--------------- COMANDO LINEA----------------->
  \newcommand{\linea}{\rule{\linewidth}{0.5mm}}                 
  \center
  %<<<<<<<<<<<<<<<<<<<<<<<<<<<<<<<<<<<<<<<<<<<

  %Add logos
  \includegraphics[width=0.24\textwidth]{../../unam_logo.png} \hspace{1.5cm}
  \includegraphics[width=0.25\textwidth]{../../fc_logo.png} \\[1cm]
  \textbf{\Large UNIVERSIDAD NACIONAL AUTÓNOMA}\\[3mm]
  \textbf{\Large DE MÉXICO} \\[6mm]
  \textit{\Large Facultad de Ciencias}

  \vfill

  %---------------TÍTULO----------------->
  \linea
  \vfill \vspace{1cm}
  \textbf{\Large Lenguajes de Programación}\\[1.1cm]
  \textbf{\LARGE M\large INI\LARGE L\large ISP}\\[0.7cm]
  \linea \\
  \vfill
  \vspace{0.8cm}
  %Title of the Research
  \textbf{\LARGE  Proyecto 1}\\[6mm]
  %<<<<<<<<<<<<<<<<<<<<<<<<<<<<<<<<<<<<<<<<<<<

  %Team
  \vfill
  \textit{\small Presenta:}\\
  \textbf{\large  \textbf{Lugo Díaz Ordaz Gretel Alexandra}}\\[3mm]
  \textbf{\large  \textbf{Ramírez Juárez María Fernanda}}\\[3mm]
  \textbf{\large  \textbf{Rojo Peña Manuel Ianluck}}\\[4mm]

  %Profesor y Grupo
  \textit{\small Profesor:}\\
  \textbf{Manuel Soto Romero}\\[1.5mm]
  \textit{\small Ayudantes:}\\
  \textbf{Diego Méndez Medina}\\[1mm]
  \textbf{Erick Daniel Arroyo Martínez}\\[1mm]
  \small Grupo: 7121, 2026-1\\[3mm]
  \vfill

  %Fecha or Date
  \textit{\small Fecha de entrega:}\\
         {\large 3 de noviembre, 2025}
         
\end{titlepage}

%%%%%%%%%%%%%%%%%%%%%%%%%%%%%%%%%%%

% ----- Ajustes para el pdf -----
\clearpage
\pagenumbering{arabic}
\setcounter{page}{1}
%%%%%%%%%%%%%%%%%%%%%%%%%%%%%%%%%%%

% ----- Iniciamos el indice -----
\tableofcontents
%%%%%%%%%%%%%%%%%%%%%%%%%%%%%%%%%%

% ----- Introducción -----
%%%%%%%%%%%%%%%%%%%%%%%%%%%%%%%%%%
%\chapter{Introducción}
%%%%%%%%%%%%%%%%%%%%%%%%%%%%%%%%%
\section{Motivación}

El estudio de los lenguajes de programación es uno de los pilares fundamentales de las ciencias de la computación. Comprender cómo se definen, interpretan y ejecutan permite no solo utilizarlos como herramientas, sino también analizarlos y extenderlos desde la perspectiva formal. En este contexto, la implementación de un lenguaje estilo \textbf{Lisp} ofrece un terreno fértil para explorar la relación entre teoría y práctica: su sintaxis minimalista, su semántica clara y su estructura recursiva facilitan su análisis formal.

El proyecto \textbf{MiniLisp} surge con la intención de integrar los conceptos teóricos vistos en clase dentro de una implementación concreta en \textbf{Haskell}. De este modo, este trabajo busca fortalecer la comprensión del proceso completo de formalización de un lenguaje de programación: desde la definición léxica y gramatical, hasta la evaluación de programas mediante un intérprete funcional.

%%%%%%%%%%%%%%%%%%%%%%%%%%%%%%%%%

\section{Objetivos}

\subsubsection{Objetivo general}
Desarrollar e implementar una versión extendida del lenguaje \textbf{MiniLisp} que formalice su sintaxis y semántica, y que permita ejecutar programas a través de un intérprete en \textbf{Haskell}, manteniendo coherencia entre el modelo teórico y la implementación práctica.

\subsubsection{Objetivos específicos}
\begin{enumerate}
    \item Definir formalmente la sintaxis léxica y libre de contexto del lenguaje, incluyendo operadores, estructuras de control y mecanismos de definición local.
    \item Implementar un analizador léxico y un analizador sintáctico utilizando las herramientas \textbf{Alex} y \textbf{Happy}, respectivamente.
    \item Diseñar la sintaxis abstracta usando dos niveles, un \textbf{ASA} con azúcar sintáctica y \textbf{AST} como núcleo del lenguaje.
    \item Desarrollar un módulo de eliminación de azúcar sintáctica (\textbf{Desugar}) que traduzca expresiones superficiales a su representación mínima.
    \item Implementar un intérprete funcional (\textbf{Interp}) basado en la semántica operacional, utilizando ambientes y \textbf{evaluación ansiosa} (eager evaluation).
    \item Proveer una interfaz interactiva que permita ejecutar programas escritos en la sintaxis concreta de \textbf{MINILISP}.
\end{enumerate}

%%%%%%%%%%%%%%%%%%%%%%%%%%%%%%%%%

\section{Delimitación del Proyecto}

El proyecto se reduce al diseño e implementación de un subconjunto de \textbf{Lisp} llamado \textbf{MINILISP}, con el propósito de estudiar los principios fundamentales de los lenguajes funcionales y su formalización. Por lo tanto:
\begin{itemize}
    \item No se abordará la \textbf{gestión de tipos} ni el \textbf{análisis estático}.
    \item El sistema de evaluación se restringe a la \textbf{evaluación ansiosa} (eager evaluation).
    \item La semántica implementada se limita al \textbf{nivel estructural}, sin considerar aspectos de optimización, compilación ni concurrencia.
    \item El alcance del proyecto comprende la construcción del \textbf{intérprete}, no un compilador ni un entorno gráfico.
\end{itemize}

%%%%%%%%%%%%%%%%%%%%%%%%%%%%%%%%%%%

Pregunta: ¿Por qué algunos lenguajes modernos (como Lisp, Python o JavaScript) adoptan reglas mixtas de alcance y cómo afecta eso al razonamiento sobre el código? Búsqueda: lexical scope vs dynamic scope in Lisp, Python scope chain, closures in JavaScript Programa: Implementar un mini-lenguaje con dos modos de alcance seleccionables,
ejecutando un mismo programa bajo ambos y comparando los resultados.


% ----- Bibliografía -----
%%%%%%%%%%%%%%%%%%%%%%%%%%%%%%%%%%
\addcontentsline{toc}{chapter}{Bibliografía}
\begin{thebibliography}{99}
\bibitem{ref1} \textcolor{links}
Bibliografia
\end{thebibliography}

%%%%%%%%%%%%%%%%%%%%%%%%%%%%%%%%%%

\end{document}
%-------------------------------
