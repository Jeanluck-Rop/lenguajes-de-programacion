\chapter{Resultados}
Bien, una vez hemos visto toda la teoría de nuestro\minilisp y además de que hemos mostrado el código que lo implementa en Haskell, veamos como funciona:

%%%%%%%%%%%%%%%%%%%%%%%%%%%%%%%%%

\section{Menú interactivo}

%%%%%%%%%%%%%%%%%%%%%%%%%%%%%%%%%

\section{Funciones de prueba}

%%%%%%%%%%%%%%%%%%%%%%%%%%%%%%%%%

\subsection{Suma primeros $n$ números naturales}

%%%%%%%%%%%%%%%%%%%%%%%%%%%%%%%%%

\subsection{Factorial}

%%%%%%%%%%%%%%%%%%%%%%%%%%%%%%%%%

\subsection{Fibonacci}

%%%%%%%%%%%%%%%%%%%%%%%%%%%%%%%%%

\subsection{Función \texttt{map} para listas}

%%%%%%%%%%%%%%%%%%%%%%%%%%%%%%%%%

\subsection{Función \texttt{filter} para listas}
