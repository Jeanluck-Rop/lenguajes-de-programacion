\documentclass[12pt]{book}

\usepackage[utf8]{inputenc}
\usepackage[T1]{fontenc}
\usepackage[spanish]{babel}
\usepackage[normalem]{ulem}
\usepackage[hidelinks]{hyperref}

\usepackage{listings, tcolorbox, xcolor, float}
\usepackage{algorithm, algpseudocode, chngcntr}
\usepackage{graphicx, enumitem, geometry}
\usepackage{amsmath, amssymb, amsthm}
\usepackage{forest}
\usepackage{fancyhdr}
\usepackage{minted}

% ----- Custom Page -----
\geometry{margin=2.5cm}
\pagestyle{fancy}
\fancyhf{}
%------------------------

% ----- Custom Commands -----
\newcommand{\nt}[1]{\texttt{<#1>}}
\newcommand{\N}{\mathbb{N}}
\newcommand{\Z}{\mathbb{Z}}
\newcommand{\B}{\mathbb{B}}
\renewcommand{\P}{\mathbb{P}}
\newcommand{\minilisp}{
  {M\footnotesize INI\normalsize L\footnotesize ISP \normalsize}
}
\renewcommand{\lstlistingname}{Código}
\renewcommand{\bibname}{Bibliografía}
%----------------------------

% ----- Book Document -----
% ----- Quitar mayúsculas automáticas -----
\renewcommand{\chaptermark}[1]{\markboth{#1}{}}
\renewcommand{\sectionmark}[1]{\markright{#1}}

% ----- Encabezados -----
\lhead{\minilisp}
\rhead{\leftmark{}.}
\cfoot{\thepage} % número centrado abajo
\renewcommand{\headrulewidth}{0.4pt}

% ----- Cambiar el título del índice -----
\addto\captionsspanish{%
  \renewcommand{\contentsname}{Índice} % cambia "Índice general" a "Índice"
}

% ----- Evitar mayúsculas en el encabezado del índice -----
\makeatletter
\renewcommand{\tableofcontents}{%
  \chapter*{\contentsname}% título "Índice"
  \markboth{\contentsname}{}% encabezado sin mayúsculas
  \@starttoc{toc}% genera el contenido real
}
\makeatother
%--------------------------

% ----- Custom Colors -----
\definecolor{azulin}{HTML}{130F87}
\definecolor{grisin}{HTML}{6D6C91}
\definecolor{links}{HTML}{0D1894}
\definecolor{mainkeywordcolor}{HTML}{610E61}
\definecolor{commentcolor}{HTML}{087008}
\definecolor{stringcolor}{HTML}{AD8318}
\definecolor{ops}{HTML}{3374A1}
\definecolor{tok}{HTML}{B39309}
\definecolor{tkns}{HTML}{560769}
\definecolor{asa}{HTML}{A81666}
\definecolor{corchets}{HTML}{A82E16}

\definecolor{secondkeywordcolor}{HTML}{0D1894}
\definecolor{thirdkeywordcolor}{HTML}{0D1894}

\definecolor{backgroundcolor}{rgb}{0.95, 0.95, 0.95}
%--------------------------

% ----- Haskell Code Style -----
\lstdefinestyle{haskellstyle}{
  language=Haskell,
  basicstyle=\ttfamily\footnotesize,
  keywordstyle=[1]\color{mainkeywordcolor}\bfseries,
  commentstyle=\color{commentcolor}\itshape,
  stringstyle=\color{stringcolor},
  numbers=left,
  numberstyle=\tiny,
  stepnumber=1,
  numbersep=8pt,
  backgroundcolor=\color{backgroundcolor},
  tabsize=4,
  showspaces=false,
  showstringspaces=false,
  breaklines=true,
  frame=single,
  captionpos=b,
  morekeywords=[1]{if,then,else,let,in,case,of,Show,Eq,
    data,type,class,instance,where, do,deriving,newtype,module,import},
  literate=
  {Maybe}{{{\textcolor{azulin}{Maybe}}}}5
  {Just}{{{\textcolor{azulin}{Just}}}}4
  {Nothing}{{{\textcolor{azulin}{Nothing}}}}7
  {True}{{{\textcolor{azulin}{True}}}}4
  {False}{{{\textcolor{azulin}{False}}}}5
  {IO}{{{\textcolor{azulin}{IO}}}}2
  {=}{{{\textcolor{ops}{=}}}}1
  {|}{{{\textcolor{ops}{|}}}}1
  {::}{{{\textcolor{ops}{::}}}}2
  {[}{{{\textcolor{corchets}{[}}}}1
  {]}{{{\textcolor{corchets}{]}}}}1,
  classoffset=1,
  morekeywords={Token},
  keywordstyle=\color{tok},
  classoffset=2,
  morekeywords={ASA},
  keywordstyle=\color{asa},
  classoffset=0,
}
%-------------------------------

%-------------------------------
\begin{document}

% ----- Importamos la portada -----
\begin{titlepage}
  \thispagestyle{empty}
  %--------------- COMANDO LINEA----------------->
  \newcommand{\linea}{\rule{\linewidth}{0.5mm}}                 
  \center
  %<<<<<<<<<<<<<<<<<<<<<<<<<<<<<<<<<<<<<<<<<<<

  %Add logos
  \includegraphics[width=0.24\textwidth]{../../unam_logo.png} \hspace{1.5cm}
  \includegraphics[width=0.25\textwidth]{../../fc_logo.png} \\[1cm]
  \textbf{\Large UNIVERSIDAD NACIONAL AUTÓNOMA}\\[3mm]
  \textbf{\Large DE MÉXICO} \\[6mm]
  \textit{\Large Facultad de Ciencias}

  \vfill

  %---------------TÍTULO----------------->
  \linea
  \vfill \vspace{1cm}
  \textbf{\Large Lenguajes de Programación}\\[1.1cm]
  \textbf{\LARGE M\large INI\LARGE L\large ISP}\\[0.7cm]
  \linea \\
  \vfill
  \vspace{0.8cm}
  %Title of the Research
  \textbf{\LARGE  Proyecto 1}\\[6mm]
  %<<<<<<<<<<<<<<<<<<<<<<<<<<<<<<<<<<<<<<<<<<<

  %Team
  \vfill
  \textit{\small Presenta:}\\
  \textbf{\large  \textbf{Lugo Díaz Ordaz Gretel Alexandra}}\\[3mm]
  \textbf{\large  \textbf{Ramírez Juárez María Fernanda}}\\[3mm]
  \textbf{\large  \textbf{Rojo Peña Manuel Ianluck}}\\[4mm]

  %Profesor y Grupo
  \textit{\small Profesor:}\\
  \textbf{Manuel Soto Romero}\\[1.5mm]
  \textit{\small Ayudantes:}\\
  \textbf{Diego Méndez Medina}\\[1mm]
  \textbf{Erick Daniel Arroyo Martínez}\\[1mm]
  \small Grupo: 7121, 2026-1\\[3mm]
  \vfill

  %Fecha or Date
  \textit{\small Fecha de entrega:}\\
         {\large 3 de noviembre, 2025}
         
\end{titlepage}

%%%%%%%%%%%%%%%%%%%%%%%%%%%%%%%%%%%

% ----- Ajustes para el pdf -----
\clearpage
\pagenumbering{arabic}
\setcounter{page}{1}
%%%%%%%%%%%%%%%%%%%%%%%%%%%%%%%%%%%

% ----- Iniciamos el indice -----
\tableofcontents
%%%%%%%%%%%%%%%%%%%%%%%%%%%%%%%%%%

% ----- Introducción -----
%%%%%%%%%%%%%%%%%%%%%%%%%%%%%%%%%%
\chapter{Introducción}
%%%%%%%%%%%%%%%%%%%%%%%%%%%%%%%%%
\section{Motivación}

El estudio de los lenguajes de programación es uno de los pilares fundamentales de las ciencias de la computación. Comprender cómo se definen, interpretan y ejecutan permite no solo utilizarlos como herramientas, sino también analizarlos y extenderlos desde la perspectiva formal. En este contexto, la implementación de un lenguaje estilo \textbf{Lisp} ofrece un terreno fértil para explorar la relación entre teoría y práctica: su sintaxis minimalista, su semántica clara y su estructura recursiva facilitan su análisis formal.

El proyecto \textbf{MiniLisp} surge con la intención de integrar los conceptos teóricos vistos en clase dentro de una implementación concreta en \textbf{Haskell}. De este modo, este trabajo busca fortalecer la comprensión del proceso completo de formalización de un lenguaje de programación: desde la definición léxica y gramatical, hasta la evaluación de programas mediante un intérprete funcional.

%%%%%%%%%%%%%%%%%%%%%%%%%%%%%%%%%

\section{Objetivos}

\subsubsection{Objetivo general}
Desarrollar e implementar una versión extendida del lenguaje \textbf{MiniLisp} que formalice su sintaxis y semántica, y que permita ejecutar programas a través de un intérprete en \textbf{Haskell}, manteniendo coherencia entre el modelo teórico y la implementación práctica.

\subsubsection{Objetivos específicos}
\begin{enumerate}
    \item Definir formalmente la sintaxis léxica y libre de contexto del lenguaje, incluyendo operadores, estructuras de control y mecanismos de definición local.
    \item Implementar un analizador léxico y un analizador sintáctico utilizando las herramientas \textbf{Alex} y \textbf{Happy}, respectivamente.
    \item Diseñar la sintaxis abstracta usando dos niveles, un \textbf{ASA} con azúcar sintáctica y \textbf{AST} como núcleo del lenguaje.
    \item Desarrollar un módulo de eliminación de azúcar sintáctica (\textbf{Desugar}) que traduzca expresiones superficiales a su representación mínima.
    \item Implementar un intérprete funcional (\textbf{Interp}) basado en la semántica operacional, utilizando ambientes y \textbf{evaluación ansiosa} (eager evaluation).
    \item Proveer una interfaz interactiva que permita ejecutar programas escritos en la sintaxis concreta de \textbf{MINILISP}.
\end{enumerate}

%%%%%%%%%%%%%%%%%%%%%%%%%%%%%%%%%

\section{Delimitación del Proyecto}

El proyecto se reduce al diseño e implementación de un subconjunto de \textbf{Lisp} llamado \textbf{MINILISP}, con el propósito de estudiar los principios fundamentales de los lenguajes funcionales y su formalización. Por lo tanto:
\begin{itemize}
    \item No se abordará la \textbf{gestión de tipos} ni el \textbf{análisis estático}.
    \item El sistema de evaluación se restringe a la \textbf{evaluación ansiosa} (eager evaluation).
    \item La semántica implementada se limita al \textbf{nivel estructural}, sin considerar aspectos de optimización, compilación ni concurrencia.
    \item El alcance del proyecto comprende la construcción del \textbf{intérprete}, no un compilador ni un entorno gráfico.
\end{itemize}

%%%%%%%%%%%%%%%%%%%%%%%%%%%%%%%%%%%

% ----- Sintaxis Concreta -----
%%%%%%%%%%%%%%%%%%%%%%%%%%%%%%%%%%
\chapter{Sintaxis Concreta}

Antes de entrar de fondo en programar nuestro\minilisp en Haskell, es necesario definir la \textbf{\textit{sintaxis concreta}} que utilizaremos para el lenguaje.

Citando al profesor, en su archivo compartido \textbf{\textit{Especificación Formal de los Lenguajes de Programación. Sintaxis Concreta}}~\cite{ref13}.

\begin{quote}
  \textit{En el contexto de la teoría de lenguajes de programación y lenguajes formales, la
    \textbf{sintaxis concreta} se refiere a la estructura específica de un lenguaje de programación que define
    exactamente cómo se deben escribir los programas. Matemáticamente, esto se describe mediante una gramática
    formal que especifica las reglas de formación para las secuencias válidas de símbolos en el lenguaje.}
\end{quote}

Esta especificación formal se divide en \textbf{\textit{sintaxis léxica}} y \textbf{\textit{sintaxis libre de
    contexto}}, con los cuales podemos construir programas válidos y sin ambigüedades, asegurando que nuestro
lenguaje pueda transformarse sin problemas en su correspondiente representación abstracta. En términos simples,
la sintaxis describe \textit{cómo se ve el programa}, es la forma exacta en la que el usuario debe escribir las
expresiones, instrucciones y estructuras del lenguaje.\\

Podemos decir que la sintaxis, constituye la \textbf{puerta de entrada entre el usuario y el compilador o
  intérprete}, definiendo los símbolos, operadores, delimitadores y palabras reservadas que el lenguaje reconoce.

\bigskip

Para nuestro lenguaje\minilisp\hspace{-0.2cm}, como una introducción a la implementación que mostraremos, hemos definido las expresiones:

\begin{itemize}
  \item \textbf{Variables.}
  \item \textbf{Números entero.}
  \item \textbf{Booleanos.}
  \item \textbf{Operadores aritméticos.}
  \item \textbf{Predicados y comparaciones.}
  \item \textbf{Asignaciones y funciones.}
  \item \textbf{Pares ordenados y proyecciones.}
  \item \textbf{Condicionales.}
  \item \textbf{Listas.}
\end{itemize}

Cabe destacar que, algunas de las operaciones dadas, tendrán la característica de ser
variádicas. Entraremos en este tema más adelante.

En conclusión, podemos pensar en la sintaxis concreta como las secuencias de caracteres del alfabeto $\Sigma$
que se convierten en programas válidos del lenguaje. Mientras que la \textit{sintaxis abstracta} (\textbf{ASA},
Árbol de Sintaxis Abstracta) representa la estructura lógica del programa, la sintaxis concreta establece las
\textbf{reglas formales de escritura} que garantizan que un programa pueda ser reconocido y analizado
correctamente. Su correcta definición es fundamental para el funcionamiento del analizador léxico
(\textit{Lexer}) y del analizador sintáctico (\textit{Parser}), ya que determina las entradas válidas que ambos
deben procesar. Esto se logra mediante un \textbf{Análisis léxico} y un \textbf{Análisis sintáctico}.

%%%%%%%%%%%%%%%%%%%%%%%%%%%%%%%%%

\section{Sintaxis Léxica}
La definición léxica se establece mediante un conjunto de \textbf{expresiones regulares}, las cuales constituyen
la base formal sobre la que se construyen los componentes básicos de un lenguaje de programación.
Dichas expresiones definen los patrones válidos de caracteres que pueden formar identificadores, números,
operadores, palabras reservadas y otros símbolos que componen el vocabulario fundamental del lenguaje.

En pocas palabras, citando al profesor:
\begin{quote}
  ``\textit{Formalmente, la sintaxis léxica se define usando \textbf{expresiones regulares} y \textbf{autómatas finitos}}.''
\end{quote}

La \textbf{sintaxis léxica}, dentro del estudio de los lenguajes formales, representa la primera capa estructural
de un lenguaje de programación. Su propósito es definir el \textit{alfabeto} del lenguaje y describir cómo las
secuencias de símbolos de dicho alfabeto se agrupan en unidades con significado propio. No describe la
estructura lógica o gramatical del programa (de estos e encarga la \textit{\textbf{sintaxis libre de contexto}}),
sino que se encarga de definir los elementos básicos que lo conforman.

En términos prácticos, esta especificación léxica permitirá posteriormente implementar un \textit{analizador
  léxico}, encargado de recorrer la entrada del usuario y separar cada componente del programa según las reglas
aquí definidas.

\subsection{Análisis Léxico}
Nuestra sintaxis se constituye de un \textbf{Análisis léxico}. El análisis léxico constituye la fase inicial en
el proceso de interpretación de lenguajes de programación. Cumple una función fundamental dentro del proceso de
compilación o interpretación, ya que actúa como un filtro inicial entre el texto fuente escrito por el usuario
y las estructuras sintácticas que procesará el analizador sintáctico.\\

Su objetivo es transformar una secuencia de caracteres sin estructura en una secuencia de \textit{Tokens}, que
representan las unidades mínimas con significado léxico en nuestro lenguaje (palabras reservadas, identificadores
, literales, operadores y delimitadores) que simplifican el trabajo del parser. Cada token encapsula información
sobre el tipo de elemento reconocido y, cuando es relevante, su valor específico.

\begin{quote}
  Definimos una función $\texttt{lexer:} \sum^* \rightarrow \texttt{[}Token\texttt{]}$, que toma una cadena de caracteres y produce la lista de $tokens$ según las expresiones regulares que hayamos definido en nuestro lenguaje.
\end{quote}

Anteriormente hicimos una breve mención de las expresiones que nuestro\minilisp va a manejar en la sintaxis
léxica. Ya que el propósito de este proyecto es académico, basta con implementar los tipos de datos más simples,
como lo son los números (\texttt{Num})y booleanos (\texttt{Boolean}), también implementamos cadenas
(\texttt{String}) pero no tendremos ningún programa que opere con cadenas de caracteres, únicamente las usaremos
como asignación de variables.

Tenemos entonces, el alfabeto $\Sigma$ de nuestro lenguaje:
\[
\Sigma = \{0,1,2,3,4,5,6,7,8,9,a-z,A-Z,\texttt{-},\texttt{+},\texttt{*},\texttt{/},\texttt{=},\texttt{\textgreater},\texttt{\textless},\texttt{!},\texttt{\#},\texttt{[},\texttt{]},\texttt{,},\texttt{(},\texttt{)}\}
\]

Ahora, los $Tokens$ de nuestro lenguaje serán justamente las cadenas reservados o caracteres que podemos formar
con dichos símbolos. Una vez tenemos encuenta todo lo anterior, definimos los siguientes tipos de $Tokens$ para
nuestro lenguaje\minilisp\hspace{-0.2cm}:

\begin{itemize}
  \item \textbf{Paréntesis:} \textbf{(} y \textbf{)}, con los que indicamos cuando comienzan y terminan nuestras expresion (por eso se llaman $delimitadores$).
  \item \textbf{Variables:} cualquier secuencia de caracteres de la forma $[a-z + A-Z][a-zA-Z0-9]^*$.
  \item \textbf{Números enteros:} $x \in \Z$.
  \item \textbf{Booleanos:} \texttt{\#t} (verdadero) y \texttt{\#f} (falso), junto con la negación \texttt{(not)}.
  \item \textbf{Operadores aritméticos:} \texttt{+}, \texttt{-}, \texttt{*}, \texttt{/}, \texttt{Add1}(incremento), \texttt{Sub1}(decremento), raíz cuadrada (\texttt{sqrt}) y potencia (\texttt{**}).
  \item \textbf{Predicados y comparaciones:} igualdad y desigualdad (\texttt{=}, \texttt{!=}), así como comparaciones numéricas (\texttt{<}, \texttt{>}, \texttt{<=}, \texttt{>=}).
  \item \textbf{Asignaciones y funciones:} construcciones \texttt{let}, \texttt{let*}, \texttt{letrec}, funciones anónimas con \texttt{lambda}, y aplicación de funciones.
  \item \textbf{Pares ordenados y proyecciones:} \texttt{(}$e1$\texttt{,} $e2${)}, \texttt{first} y \texttt{second}.
  \item \textbf{Condicionales:} \texttt{if}, \texttt{if0} y \texttt{cond}.
  \item \textbf{Listas:} delimitadas por corchetes \texttt{[} y \texttt{]}, con elementos separados por comas \texttt{,}, junto con operaciones básicas \texttt{head} y \texttt{tail}.
\end{itemize}

Como primera parte de nuestra implementación en Haskell del \textit{\textbf{análisis léxico}}, utilizamos la
palabra reservada \textcolor{mainkeywordcolor}{\texttt{data}} que nos permite definir nuevos tipos de datos y
los constructores asociados a ellos.

%%%%%%%%%%%%%%%%%%%%%%%%%%%%%%%%%

\subsection{Tokens}
La estructura del tipo \textit{Token} son las piezas fundamentales que permiten construir la sintaxis del
lenguaje de manera estructurada y libre de ambigüedades. Incluso, no solo nos permiten clasificar y representar
las unidades léxicas mínimas reconocibles por el lenguaje, sino que también facilitan el trabajo del
\textit{parser}.

Para nuestro proyecto\minilisp definimos el tipo de dato \texttt{Token} en Haskell dentro del archivo
\texttt{Tokens.hs}, con el cual representamos cada posible componente léxico del lenguaje.

Queda definido como sigue:

\begin{lstlisting}[style=haskellstyle, caption={Estructura de Tokens}]
  data Token
    = TokenVar String
    | TokenNum Int
    | TokenBool Bool
    | TokenAdd
    | TokenSub
    | TokenMul
    | TokenDiv
    | TokenAdd1
    | TokenSub1
    | TokenSqrt
    | TokenExpt
    | TokenNot
    | TokenEq
    | TokenLt
    | TokenGt
    | TokenNeq
    | TokenLeq
    | TokenGeq
    | TokenIf0
    | TokenIf
    | TokenCond
    | TokenElse
    | TokenFirst
    | TokenSecond
    | TokenHead
    | TokenTail
    | TokenLet
    | TokenLetRec
    | TokenLetStar
    | TokenLambda
    | TokenLI
    | TokenLD
    | TokenComma
    | TokenPA
    | TokenPC
    deriving (Show, Eq)
\end{lstlisting}

Nótese que, los $Tokens$:\: \texttt{TokenVar}, \texttt{TokenNum}, \texttt{TokenBool}, además de encapsular el
tipo de elemento reconocido, guardan su valor específico asociado a dichos $Tokens$ con los tipos de datos en el
lenguaje anfitrión (\texttt{String}, \texttt{Int} y \texttt{Bool}).

%%%%%%%%%%%%%%%%%%%%%%%%%%%%%%%%%

\subsection{Alex}
Cada vez que el analizador léxico identifica un patrón en la entrada, genera el token correspondiente y al final,
esta función \texttt{lexer}, construirá una lista de $Tokens$ la cual recibirá el analizador sintáctico.
Utilizamos la herramienta \texttt{Alex} que nos ayudará con la implementación de este \texttt{lexer} en Haskell.\\
    
Alex es el generador de analizadores léxicos estándar para Haskell, toma una descripción de tokens basada en
expresiones regulares y genera un Haskell \texttt{module} que contiene código para escanear texto de manera
eficiente\cite{ref6}. Esta elección se fundamenta en varias ventajas significativas:

\begin{itemize}
\item \textbf{Reducción de errores:} Alex automatiza la generación de código robusto, minimizando errores comunes en implementaciones manuales.
\item \textbf{Expresividad:} Utiliza expresiones regulares extendidas para definir patrones léxicos de manera clara y concisa.
\item \textbf{Integración con Haskell:} Genera código Haskell nativo que se integra perfectamente con el resto de nuestro intérprete.
\item \textbf{Eficiencia:} Produce analizadores de alto rendimiento mediante algoritmos de coincidencia optimizados.
\end{itemize}

\bigskip

Implementamos Alex en el archivo \texttt{Lexer.x}, su estructura es la siguiente:\\

Lo primero que hacemos es importar los $Tokens$ definidos y construidos en el archivo \texttt{Tokens.hs} e
importar \texttt{Data.Char} para usar la función \texttt{isSpace} con la que normalizaremos espacios Unicode.\\

Después, definimos los patrones básicos que establecen los bloques fundamentales para construir patrones más
complejos, promoviendo la reutilización y claridad. Estas líneas no son código Haskell, sino instrucciones para
Alex, con ellos le indicamos a Alex: ``\textit{Cuando veas \$digit en las reglas, reemplázalo por 0-9}''. Lo
mismo para $\$alpha$ con \texttt{[a-zA-Z]} y $\$alphnum$ \texttt{[a-zZ-Z0-9]}.\\

Además de incluir con la definición de los espacios ($whitespaces$): espacio ASCII ($\backslash$ x20), tabulador
($\backslash$ x09), LF ($\backslash$ x0A), CR ($\backslash$ x0D), FF ($\backslash$ x0C), VT ($\backslash$ x0B).
Definirlos explícitamente nos ayuda a ignorarlos al definir la regla de construcción o de lectura para generar
los $Tokens$ de la cadena recibida.\\

Por último \texttt{tokens :-} marca el comienzo de la sección de patrones de las expresiones regulares que Alex
convertirá en la lista de $Tokens$ \texttt{regex \{ Token \}}. Declarando también la regla de ignorar los
espacios y salto de (\texttt{\$white+}).\\

\bigskip

\begin{lstlisting}[style=haskellstyle, caption={Lexer con Alex.}]
  {
  module Lexer where
    
  import Token
  import Data.Char (isSpace)
  }
  
  %wrapper "basic"

  -- Definiciones de patrones
  $digit   = 0-9
  $alpha   = [a-zA-Z]
  $alnum   = [a-zA-Z0-9]
  
  -- Usamos codigos hex para los espacios en blanco Unicode mas comunes:
  --   \x20 = ' ' (space), \x09 = tab, \x0A = LF, \x0D = CR, \x0C = FF, \x0B = VT
  $white = [\x20\x09\x0A\x0D\x0C\x0B]
  
  tokens :-
  
  -- Ignoramos espacios y saltos de linea
  $white+                       ;
\end{lstlisting}

Continuamos con la definición de los \textbf{delimitadores estructurales} y los \textbf{operadores básicos} de
nuestro lenguaje, los cuales constituyen los símbolos fundamentales que permiten organizar y expresar la
estructura de los programas en\minilisp.\\

Cada una de estas reglas dentro del analizador léxico de Alex consta de dos componentes principales:

\begin{itemize}
\item \textbf{Patrón o expresión regular:} Es la secuencia de caracteres que el lexer debe reconocer. En este
  caso, se trata de los símbolos estructurales o palabras clave como \texttt{(}, \texttt{let}, \texttt{+}, etc.
  Cabe mencionar que estos caracteres pueden definirse de manera personalizada; sin embargo, para mantener la
  coherencia con la notación tradicional de los lenguajes de programación, utilizamos los símbolos comúnmente
  aceptados, como \texttt{+} para la suma y \texttt{-} para la resta.  
\item \textbf{Bloque de acción:} Es el fragmento de código en Haskell que se ejecuta cuando se reconoce el
  patrón. Su función es generar el token correspondiente, por ejemplo:\\ \texttt{\{ \_ -> TokenPA \}}. 
\item \textbf{Expresión lambda:} Dentro del bloque de acción, la expresión lambda define cómo se construye el
  token. En el ejemplo anterior, \texttt{\_ -> TokenPA}, el símbolo \texttt{\_} representa la cadena de texto
  que coincidió con el patrón (la entrada reconocida), el operador \texttt{->} separa el parámetro del resultado,
  y \texttt{TokenPA} es el constructor del token que se devuelve al análisis sintáctico.
\end{itemize}

Es importante resaltar el caso de las \textbf{palabras reservadas}, como \texttt{let*}, \texttt{letrec},
\texttt{!=}, \texttt{add1}, entre otros. En el diseño del lexer, estas reglas deben escribirse \textit{antes} que
las reglas más generales o más cortas (por ejemplo, \texttt{let}, \texttt{<}, \texttt{!}, \texttt{+}).\\

Esto se debe a que el generador de analizadores léxicos Alex aplica la estrategia conocida como \textit{longest
  match}, que selecciona la coincidencia más larga posible. En caso de empate entre dos patrones de igual
longitud, prevalece la primera regla definida en el archivo.\\

Por lo tanto, si definiéramos la regla de \texttt{let} antes que \texttt{let*}, la cadena \texttt{let*} nunca
coincidiría correctamente, ya que la primera regla (más corta) interceptaría el patrón. Este ordenamiento de las
reglas garantiza un análisis léxico preciso y evita ambigüedades en el reconocimiento de tokens.

\bigskip

\begin{lstlisting}[style=haskellstyle, caption={Lexer con Alex.}]
  \(                            { \_ -> TokenPA }
  \)                            { \_ -> TokenPC }
  \[                            { \_ -> TokenLI }
  \]                            { \_ -> TokenLD }
  \,                            { \_ -> TokenComma }
  \+                            { \_ -> TokenAdd }
  \-                            { \_ -> TokenSub }
  \*                            { \_ -> TokenMul }
  \/                            { \_ -> TokenDiv }
  \=                            { \_ -> TokenEq }
  \<                            { \_ -> TokenLt }
  \>                            { \_ -> TokenGt }
  "add1"                          { \_ -> TokenAdd1 }
  "sub1"                          { \_ -> TokenSub1 }
  "sqrt"                        { \_ -> TokenSqrt }
  "**"                          { \_ -> TokenExpt }
  "!="                          { \_ -> TokenNeq }
  "<="                          { \_ -> TokenLeq }
  ">="                          { \_ -> TokenGeq }
  "not"                         { \_ -> TokenNot }
  "if0"                         { \_ -> TokenIf0 }
  "if"                          { \_ -> TokenIf }
  "first"                       { \_ -> TokenFst }
  "second"                      { \_ -> TokenSnd }
  "letrec"                      { \_ -> TokenLetRec }
  "let*"                        { \_ -> TokenLetStar }
  "let"                         { \_ -> TokenLet }
  "lambda"                      { \_ -> TokenLambda }
  "head"                        { \_ -> TokenHead }
  "tail"                        { \_ -> TokenTail }
  "cond"                        { \_ -> TokenCond }
  "else"                        { \_ -> TokenElse }
  "#t"                          { \_ -> TokenBool True }
  "#f"                          { \_ -> TokenBool False }
  "-"?$digit+                   { \s -> TokenNum (read s) }
  $alpha ($alnum)*              { \s -> TokenVar s }
\end{lstlisting}

\newpage

Nótese que tenemos las reglas para booleanos y literales con \texttt{$\#$t} y \texttt{$\#$f} para
\texttt{TokenBool}, mientras que con \texttt{"-"?$\$$digit+} para uno o más digitos a partir de la cadena $s$
incluyendo los números negativos con \texttt{"-"?} y las variables con \texttt{$\$$alpha}
(\texttt{$\$$alnum$^*$}). Son los elementos fundamentales que representan los valores básicos y nombres en
nuestro lenguaje, son las expresiones que contienen datos específicos en el programa usando el lenguaje anfitrión
para guardar estos datos.\\

Por último definimos un \textit{catch-all} para diagnosticar caracteres inesperados. Es una depuración útil, si
el usuario introduce un carácter inválido, el \texttt{lexer} falla con un mensaje claro y el código Unicode del
carácter. Además definimos la función normalizeSpaces para que los espacios en Unicode los consuma
\texttt{$\$$white+}.\footnotemark{}

\bigskip

\begin{lstlisting}[style=haskellstyle, caption={Lexer con Alex.}]  
  -- Catch-all para diagnosticar caracteres inesperados
  .                     { \s -> error ("Lexical error: caracter no reconocido = "
                                      ++ show s
                                      ++ " | codepoints = "
                                      ++ show (map fromEnum s)) }

  {
    -- Normaliza cualquier espacios en blanco Unicode a ' ' para que $white+ lo consuma
    normalizeSpaces :: String -> String
    normalizeSpaces = map (\c -> if isSpace c then '\x20' else c)
    
    lexer :: String -> [Token]
    lexer = alexScanTokens . normalizeSpaces
  }
\end{lstlisting}

\medskip

Finalmente, definimos la firma del \texttt{lexer} como \texttt{lexer :: String -> [Token]}, cumpliendo así con la
función esencial del \textit{análisis léxico}: recibir una cadena de entrada (el código fuente escrito por el
usuario en nuestro lenguaje\minilisp\hspace{-0.2cm}) y transformarla en una secuencia de \texttt{Tokens} reconocibles.\\

En el capítulo dedicado a los \textbf{Resultados}, se muestran distintos ejemplos de
ejecución de esta módulo, donde mostramos la $tokenización$ de expresiones dadas dentro del lenguaje\minilisp\hspace{-0.2cm}.

\footnotetext{Para realizar el lexer tomamos como referencia lo visto en clase con el profesor y el material compartido en su GitHub, además de usar la documentación oficial de Alex\cite{ref4} para desarrollar nuestro lexer.}

%%%%%%%%%%%%%%%%%%%%%%%%%%%%%%%%%
\newpage
%%%%%%%%%%%%%%%%%%%%%%%%%%%%%%%%%

\section{Sintaxis Libre de Contexto}
La \textit{\textbf{sintaxis libre de contexto}} se refiere a la estructura de un lenguaje de programación en la
que las reglas de formación de sus sentencias se pueden describir mediante una gramática libre de contexto. En
ella especificamos cómo se pueden combinar las secuencias de \textit{tokens} para formar expresiones y
sentencias válidas para el lenguaje. Sin la gramática no podemos darle la estructura necesaria a para que, tanto el usuario como el interprete puedan hacer su trabajo.\\

En otras palabras, la \textit{\textbf{sintaxis libre de contexto}} constituye el \textit{esqueleto sintáctico} del lenguaje. Si el \textbf{análisis léxico} segmenta la entrada en \textit{Tokens}, el \textbf{análisis sintáctico} (guiado por una \textit{gramática libre de contexto}) se encarga de verificar que dichos $Tokens$ se ensamblen de manera coherente conforme a las reglas del lenguaje. Sin una gramática bien definida, no sería posible darle forma ni estructura a los programas escritos en\minilisp\hspace{-0.2cm}, ni mucho menos permitir que el intérprete los procese correctamente. Necesitamos de la gramática para dar orden, decidir qué aceptamos y cómo lo aceptamos, de este modo damos más formalidad y menos ambigüedad al lenguaje.

\subsection{La gramática de\minilisp}
Según Hopcroft y Ullman es su libro \textit{\textbf{Introduction to Automata Theory, Languages, and Computation}}~\cite{ref14}, una \textit{\textbf{gramática libre de contexto}} se define formalmente como una tupla:
\[
G = (V, T, P, S)
\]
donde:
\begin{itemize}
\item $V$ es un conjunto finito de símbolos \textbf{no terminales} o variables las cuales representan conjuntos
  de cadenas que están siendo definidos recursivamente, es decir, cada variable genera un lenguaje.
\item $T$ es un conjunto finito, disjunto de $V$ \textbf{de simbolos terminales}.
\item $P$ es el conjunto finito de \textbf{reglas de producción}; cada producción es de la forma $A \to \alpha$, donde $A$ es una variable [en $V$] y $\alpha$ es una cadena de símbolos en $(V \cup T)^*$
\item $S$ es una variable en $V$ llamada el símbolo inicial.
\end{itemize}

\bigskip

En nuestro proyecto, la gramática de\minilisp\hspace{-0.2cm} está definida mediante la notación \textbf{BNF} (\textit{Backus-Naur Form}), en particular la notación de \textbf{EBNF}.\\

Al rededor de los años 1950 y 1960, John Backus y Peter Naur desarrollaron esta notación (\textbf{BNF}) como una
solución a la necesidad de definir de manera clara y precisa la sintaxis de los lenguajes de programación.
Sin embargo, aunque \textbf{BNF} es efectiva, tiene ciertas limitaciones en términos de expresividad,
especialmente para describir repeticiones y agrupaciones de una manera más compacta.

Con la notación \textbf{EBNF}:
\begin{itemize}
\item Las \textbf{variables} o no terminales se denotan entre los símbolos \texttt{<}\texttt{>}.
\item Las \textbf{reglas de producción} se escriben con el operador \texttt{::=}.
\item El símbolo \texttt{|} se utiliza para indicar \textbf{alternativas}, permitiendo expresar diferentes formas de una misma construcción sintáctica.
\item La extensión de \textbf{EBNF} agrega el uso de \texttt{\{} \texttt{\}} para indicar \textbf{repetición} de cero o más veces.
\end{itemize}

\bigskip

De esta manera, cada producción de la gramática define cómo los \textit{tokens} generados por el analizador
léxico (como \texttt{TokenAdd}, \texttt{TokenIf}, \texttt{TokenLet}, etc.) se combinan para formar expresiones
válidas en el lenguaje. Recordemos que en la sección de \textbf{Sintaxis Léxica} se estableció la correspondencia
entre patrones de texto y sus respectivos tokens; ahora, en esta etapa, esos mismos tokens se convierten en los
símbolos terminales de nuestra gramática.\\

Las reglas sintácticas que definen la forma de las expresiones en esta versión de\minilisp\ son las siguientes:
\begin{itemize}
\item Toda expresión está delimitada por paréntesis.
\item Usamos la notación prefija, donde el operador precede a sus argumentos (operandos).
\item Las operaciones aritméticas \texttt{+}, \texttt{-}, \texttt{*} y \texttt{/} son $n$-$arias$ ($variádicas$), permitiendo una cantidad arbitraria de argumentos.
\item Los predicados sobre enteros (igualdad y comparaciones) \texttt{=}, \texttt{<}, \texttt{>}, \texttt{>=}, \texttt{<=} y \texttt{!=} también admiten múltiples argumentos.
\item Las asignaciones \texttt{let} y \texttt{let*} son igualmente son variádicas, es decir, permiten realizar asiganaciones locales con múltiples variables.
\item Las listas se denotan mediante el uso de \texttt{[} \texttt{]} con la característica de que cada elemento ($expresión$) nuevo en la lista es separado del anterior con una coma \texttt{,} .
\item Por último la expresión condicional \texttt{cond}, permite escribir múltiples condiciones de forma ordenada.
\end{itemize}

\newpage

Con esto explicado, definimos la Gramática para\minilisp\hspace{-0.2cm} en notación \textbf{EBNF} como sigue:

\begin{tcolorbox}[colback=azulin!5!white, colframe=azulin!80, title=Gramática\minilisp]
\renewcommand{\arraystretch}{1.05}
\[
\begin{array}{rcl}
\nt{Expr} &::=& \nt{Var} \\
          &\mid& \nt{Num} \\
          &\mid& \nt{Bool} \\
          &\mid& \texttt{(+ \nt{Expr} \nt{Expr} \{\nt{Expr}\})} \\
          &\mid& \texttt{(- \nt{Expr} \nt{Expr} \{\nt{Expr}\})} \\
          &\mid& \texttt{(* \nt{Expr} \nt{Expr} \{\nt{Expr}\})} \\
          &\mid& \texttt{(/ \nt{Expr} \nt{Expr} \{\nt{Expr}\})} \\
          &\mid& \texttt{(add1 \nt{Expr})} \\
          &\mid& \texttt{(sub1 \nt{Expr})} \\
          &\mid& \texttt{(sqrt \nt{Expr})} \\
          &\mid& \texttt{({*}{*} \nt{Expr})} \\
          &\mid& \texttt{(not \nt{Expr})} \\
          &\mid& \texttt{(= \nt{Expr} \nt{Expr} \{\nt{Expr}\})} \\
          &\mid& \texttt{(<\: \nt{Expr} \nt{Expr} \{\nt{Expr}\})} \\
          &\mid& \texttt{(>\: \nt{Expr} \nt{Expr} \{\nt{Expr}\})} \\
          &\mid& \texttt{(<= \nt{Expr} \nt{Expr} \{\nt{Expr}\})} \\
          &\mid& \texttt{(>= \nt{Expr} \nt{Expr} \{\nt{Expr}\})} \\
          &\mid& \texttt{(!= \nt{Expr} \nt{Expr} \{\nt{Expr}\})} \\
          &\mid& \texttt{(\nt{Expr}, \nt{Expr})} \\
          &\mid& \texttt{(fst \nt{Expr})} \\
          &\mid& \texttt{(snd \nt{Expr})} \\
          &\mid& \texttt{(let ((\nt{Var} \nt{Expr}) \{\nt{Var} \nt{Expr}\}) \nt{Expr})} \\
          &\mid& \texttt{(letrec (\nt{Var} \nt{Expr}) \nt{Expr})} \\
          &\mid& \texttt{(let* ((\nt{Var} \nt{Expr}) \{\nt{Var} \nt{Expr}\}) \nt{Expr})} \\
          &\mid& \texttt{(if0 \nt{Expr} \nt{Expr} \nt{Expr})} \\
          &\mid& \texttt{(if \nt{Expr} \nt{Expr} \nt{Expr})} \\ 
          &\mid& \texttt{(lambda (\nt{Var} \{\nt{Var}\}) \nt{Expr})} \\
          &\mid& \texttt{(\nt{Expr} \nt{Expr} \{\nt{Expr}\})} \\
          &\mid& \texttt{``['' [ \nt{Expr} \{"," \nt{Expr}\} ] ``]''} \\
          &\mid& \texttt{(head \nt{Expr})} \\
          &\mid& \texttt{(tail \nt{Expr})} \\
          &\mid& \text{\texttt{(cond "["\nt{Expr} \nt{Expr}"]"} \texttt{\{"["\nt{E} \nt{E}"]" \} "['' else \nt{Expr}"]")}} \\
\\
\nt{Var} &::=& \textit{Identificador de variable} \\
\nt{Num} &::=& \textit{Constante entera} \\
\nt{Bool} &::=& \texttt{\#t} \mid \texttt{\#f}
\end{array}
\]
\end{tcolorbox}

\bigskip

Como se puede apreciar, definimos en las reglas para la gramática, que los operadores aritméticos(suma, resta,
multiplicación y división) que son variádicos, efectivamente lo sean. Decidimos forzar que cada uno de ellos
reciba al menos dos expresiones, ya que el uso de las llaves \texttt{{}} en la notación \textbf{EBNF} indica que
puede haber cero o más repeticiones. Por ello, cualquier invocación de un operador aritmético con menos de dos
operandos no será aceptada por el lenguaje.\\

En contraste, los operadores de incremento y decremento se definieron para aceptar únicamente una expresión. Ya
que así modelamos su comportamiento natural: ambos operan sobre un solo valor, aumentando o disminuyendo su
contenido en una unidad. De forma similar, en el caso de la raíz cuadrada, solo se requiere una expresión, dado
que su propósito es calcular la raíz cuadrada de un único número.
Para el operador del exponente, decidimos mantener el mismo comportamiento, por lo que en nuestro lenguaje, este
operador eleva al cuadrado el valor de la expresión proporcionada, así solo necesita de un argumento. El operador
not, su regla también refleja ese uso unario, pues su función es negar el valor booleano de un único argumento.\\

De manera análoga a los operadores aritméticos, las operaciones de comparación (=, <, >, <=, >=, !=), al también
ser definidos como variádicos, exigimos que al menos se especifiquen dos expresiones y damos la posibilidad de
que haya más, ya que una sola no permitiría realizar una comparación válida.\\

En cuanto a las expresiones de asignación y alcance como \texttt{let}, \texttt{letrec} y \texttt{let*},
establecimos que debe haber al menos un par \texttt{(\nt{Var} \nt{Expr})}, permitiendo además la inclusión de
múltiples pares adicionales. Así reflejamos la posibilidad de definir una o más asociaciones dentro de un mismo
bloque, manteniendo la flexibilidad solicitada para el proyecto.

Smilarmente con la aplicación de funciones y las funciones $\lambda$, donde especificamos que debe haber almenos
, una variable para la función lambda y  dos expresiones expresiones para la aplicación de función: donde la
primera corresponde a la función a aplicar y la segunda a su primer argumento; seguidas opcionalmente de más
variables para la función o más argumentos para la aplicación. Con esto aseguramos que la aplicación de
funciones y las funciones lambda siempre sean válidas y tengan sentido semántico.\\

Por último, cabe resaltar que en nuestra gramática el uso de los corchetes \texttt{[} y \texttt{]} tiene dos
propósitos: En \textbf{EBNF}, los corchetes se utilizan para denotar opcionalidad, sin embargo, en \minilisp
decidimos emplear comillas dobles \texttt{""} alrededor de los corchetes literales (\texttt{"["} y \texttt{"]"})
para distinguirlos de los usados por la notación formal. De esta manera, los corchetes con comillas representan
la sintaxis concreta del lenguaje (las listas y condicionales \texttt{cond}), mientras que los corchetes sin
comillas siguen indicando opcionalidad en la notación formal. Así, la regla:

\[
\text{\texttt{"[" [ \nt{Expr} \{ "," \nt{Expr} \} ] "]"}}
\]
\noindent
permite definir listas que pueden estar vacías o contener una o más expresiones separadas por comas,
representando correctamente la flexibilidad del manejo de listas dentro del lenguaje.

\subsection{Análisis sintáctico}
Una vez definida la \textbf{gramática libre de contexto} para \minilisp, podemos pasar a la etapa de \textbf{análisis sintáctico}, también conocida como \textit{parsing}.

Como bien mencionamos, mientras que el \textbf{análisis léxico} se encarga de transformar la cadena de entrada
en una secuencia de \textit{Tokens}, el \textbf{análisis sintáctico} tiene la tarea de verificar que dicha
secuencia respete las reglas estructurales del lenguaje, tal como fueron establecidas por la gramática.\\

En otras palabras, el analizador sintáctico organiza los tokens generados por el \texttt{lexer} conforme a las
producciones de la gramática, construyendo una representación jerárquica del programa. Esta representación se
denomina \textbf{árbol de sintaxis abstracta} (\textit{ASA} o \textit{Abstract Syntax Tree}-\textbf{AST}), el
cual captura la estructura lógica del programa, eliminando detalles superficiales como los paréntesis o
separadores que solo sirven para dar forma a la sintaxis concreta.\\

Formalmente, definimos una función sintáctica \texttt{parser}:
\[
\text{\texttt{parser: [Token]}} \rightarrow \text{\texttt{ASA}}
\]

Toma una secuencia de tokens y produce un \textbf{árbol de sintaxis abstracta} ($ASA$) según la gramática. Si el
programa no respeta las reglas de sintaxis, este árbol no puede ser construido, lo que implica un \textbf{error
  sintáctico}.\\

El análisis sintáctico representa entonces, una etapa intermedia y esencial dentro del proceso de interpretación:
traduce la estructura lineal de los tokens en una forma jerárquica que puede ser fácilmente interpretada y
evaluada por etapas posteriores de\minilisp\hspace{-0.2cm}.

\bigskip

\noindent
Con todo lo visto en este capítulo, podemos concebir la \textit{\textbf{Sintaxis Concreta}} de nuestro lenguaje
como la composición funcional entre el \textbf{analizador léxico} y el \textbf{analizador sintáctico}, donde
ambos trabajan en conjunto para transformar una cadena de caracteres en una estructura interna coherente:
\[
\text{\texttt{(parser}} \circ \text{\texttt{lexer):}}\;\; \Sigma^{*} \rightarrow \text{\texttt{ASA}}
\]

donde $\Sigma^{*}$ representa todas las cadenas posibles de símbolos del alfabeto del lenguaje, y \textbf{ASA}
(\textit{Árbol de Sintaxis Abstracta}) es la estructura resultante. Y con esto cubrimos las fases que onforman
el puente entre la entrada textual del usuario y las representaciones internas que permiten la evaluación del
lenguaje.\\

\noindent
A partir de este punto, continuaremos con las definiciones formales que dan estructura interna a nuestro lenguaje
\minilisp\hspace{-0.2cm}, entramos en el tema de la construcción del \textbf{Árbol de Sintaxis Abstracta} y la
implementación del \textbf{analizador sintáctico} (\textit{parser}) usando Happy.

%%%%%%%%%%%%%%%%%%%%%%%%%%%%%%%%%%

% ----- Sintaxis Abstracta -----
%%%%%%%%%%%%%%%%%%%%%%%%%%%%%%%%%%
\chapter{Sintaxis Abstracta}
La \textit{sintaxis abstracta} es la representación interna de la estructura del lenguaje, se enfoca en los
componentes esenciales y en cómo se relacionan entre sí, ignorando los detalles concretos del código fuente
escritos por el usuario (detalles necesarios para nosotros como programadores, pero generalmente irrelevantes
para el intérprete).

Está enfocada en capturar la \textbf{lógica} y \textbf{jerarquía} del programa dejando de lado elementos
puramente sintácticos como los paréntesis o el formato. Formalizar en ella nos permitirá desarrollar un
intérprete más eficiente y robusto, ya que nos facilita la detección de errores, la optimización del código y de
incorporar en un futuro nuevas funcionalidades sin mayor problema. Justo como en todo nuestro campo de trabajo,
buscamos hacer que nuestro código sea expandible y para ello tenemos que definir una estructura sólida desde el
comienzo.\\

En comparación con la sintaxis concreta, la sintaxis abstracta es más clara y simple, pues elimina los detalles
sintácticos (como paréntesis), enfocándose en la estructura lógica de las operaciones. Esta simplificación
reduce la complejidad del análisis y mejora la eficiencia de las herramientas que operan sobre el código, como
analizadores, optimizadores e intérpretes.

Mientras la sintaxis concreta nos da una representación más cercana al lenguaje humano (legible y expresiva), la sintaxis abstracta nos brinda una representación más adecuada para el procesamiento automático. Ambas son
complementarias: la primera facilita la escritura del código, y la segunda permite su interpretación y
evaluación.\\

Cabe mencionar que existe un concepto intermedio denominado \textbf{azúcar sintáctica}, el cual se refiere a
aquellas construcciones del lenguaje que hacen más legible el código sin agregar nueva funcionalidad. En
términos prácticos, la relación entre la sintaxis concreta, la abstracta y la azúcar sintáctica puede
entenderse como un proceso progresivo de simplificación: primero eliminamos los elementos puramente sintácticos
(paréntesis, separadores, etc.), y posteriormente reducimos aún más la estructura, obteniendo así una versión
mínima que el intérprete pueda evaluar directamente y nos facilite la implementación del mismo. Profundizaremos
más adelante en este aspecto al tratar la reducción de expresiones y la eliminación del azúcar sintáctica.

%%%%%%%%%%%%%%%%%%%%%%%%%%%%%%%%%

\section{Árboles de Sintaxis Abstracta}
A menudo, la sintaxis abstracta suele representarse como un \textit{\textbf{Árbol de Sintaxis Abstracta}}. Esta
es una representación jerárquica modela la estructura lógica del programa: cada nodo del árbol corresponde a una
construcción del lenguaje, y las hojas representan valores o identificadores.

A diferencia de la sintaxis concreta, en un \textbf{ASA} los paréntesis, comas y demás símbolos delimitadores no
se representan explícitamente, pues su propósito es estructural, no semántico. Lo que sí se conserva es la
relación jerárquica entre las partes del programa: qué elementos dependen de otros y cómo se combinan.\\

Formalmente, un \textit{Árbol de Sintaxis Abstracta} puede definirse como una tupla ordenada $A =$ ($N, E, R$)
donde:

\begin{itemize}
  \item $N$ es un conjunto finito de \textbf{nodos}, que representan las construcciones del lenguaje mediante etiquetas y las hojas representan a sus respectivos valores.
  \item $E \subseteq N \times N$ es el conjunto de \textbf{aristas dirigidas} que conectan los nodos, representando las relaciones jerárquicas entre ellos.
  \item $R \in N$ es la \textbf{raíz} del árbol, correspondiente a la expresión o programa principal.
\end{itemize}

Cada subárbol dentro del \textbf{ASA} puede interpretarse como una subexpresión del programa, lo que permite
recorrerlo de forma recursiva para su evaluación, análisis o transformación. De esta manera, el \textit{Árbol
  de Sintaxis Abstracta} constituye el puente entre la entrada textual del usuario y la representación interna
que manipula el intérprete de nuestro lenguaje\minilisp\hspace{-0.3cm}.

%%%%%%%%%%%%%%%%%%%%%%%%%%%%%%%%%

\section{ASA para\minilisp}
Con lo anterior establedico, necesitamos ahora definir formalmente las reglas que nos permitan especificar los
\textbf{ASA}'\textbf{s} de\minilisp\hspace{-0.2cm}.\\

Para formalizar esta descripción, definimos la relación:
\[
X \text{ ASA}
\]
que se lee como “\textit{X es un Árbol de Sintaxis Abstracta}”. A partir de esta relación, especificamos las
reglas que determinan qué estructuras son consideradas válidas como \textbf{ASA} dentro del lenguaje. De forma
intuitiva, cada expresión definida en la gramática tiene su respectiva etiqueta, en el \textbf{ASA} es el
\textit{nodo padre}, y cada sub-expresión asociada a esta nueva etiqueta serán sus \textit{nodos hijo}.\\

Hacemos la pequeña pero importante aclaración de que, las expresiones para pares, listas, aplicación de funciones
, etc., que no hayamos definido una palabra reservada o algún caracter que el usuario deba escribir para
identificarlos como tales, tendrán su etiqueta. Pues aunque no tengas estas palabras clave, se disntinguen por
su sintaxis definida en la gramática.

Con un extraordinario uso de la imaginación, tenemos las siguientes etiquetas para nuestras expresiones
de\minilisp\hspace{-0.2cm} y definimos la descripción de sus reglas:

%%%%%%%%%%%%%%%%%%%%%%%%%%%%%%%%%
\subsection{Expresiones atómicas}
\begin{itemize}
\item \textbf{Variables}
  
  \textit{Var}($s$) es un \textbf{ASA} si s es una cadena válida en el lenguaje (en particular si $s$ es de tipo
  \texttt{String} en el lenguaje anfitrión Haskell).
  \[
  \frac{s \in \text{\texttt{String}}}{\text{\textit{Var}(\textit{s}) \textbf{ASA}}}
  \]
  
  Su \textbf{ASA}, dado que es expresión atómica es la siguiente:
  \[
  \begin{forest}
    for tree={ math content, edge={-}, s sep=10pt, l sep=12pt, align=center }
    [Var [$s$] ]
  \end{forest}
  \]

\item \textbf{Números}
  
  \textit{Num}($n$) es un \textbf{ASA} si $n \in \Z$ ($n$ es de tipo \texttt{Int} en el lenguaje anfitrión Haskell)
  \[
  \frac{n \in \Z}{\text{\textit{Num}(\textit{n}) \textbf{ASA}}}
  \]
  
  Su \textbf{ASA}, dado que es expresión atómica es la siguiente:
  \[
  \begin{forest}
    for tree={ math content, edge={-}, s sep=10pt, l sep=12pt, align=center }
    [Num [$n$] ]
  \end{forest}
  \]

\item \textbf{Booleanos}
  
  \textit{Boolean}($b$) es un \textbf{ASA} si b es \texttt{True} o \texttt{False} ($b$ es de tipo \texttt{Bool} en
  Haskell).
  \[
  \frac{b \in \{\text{\texttt{True}, \texttt{False}\}}}{\text{\textit{Boolean}(\textit{b}) \textbf{ASA}}}
  \]
  
  Su \textbf{ASA}, dado que es expresión atómica es la siguiente:
  \[
  \begin{forest}
    for tree={ math content, edge={-}, s sep=10pt, l sep=12pt, align=center }
    [Boolean [$b$] ]
  \end{forest}
  \]
\end{itemize}

Las reglas anteriores podemos considerarlas como los \textit{nodos hoja} de\minilisp\hspace{-0.2cm} ya que no
tenemos que evaluar nada más.

%%%%%%%%%%%%%%%%%%%%%%%%%%%%%%%%%

\subsection{Operadores aritméticos}
Como bien explicamos, dentro de nuestro lenguaje algunos de los operadores aritméticos se caracterizan por ser
\textit{variádicos}, por lo que su representación dentro del \textbf{ASA} se confroma de un árbol $n$-$ario$,
mientras que el resto de los operadores aplican únicamente a un único argumento, es decir, son $unarios$. De este
modo tenemos los dos casos:
\begin{itemize}
\item Variádicos
  
  Los \textbf{ASA} de estos operadores son prácticamente los mismos, donde el \textit{nodo raíz}\footnote{Aunque
  no es precisamente un nodo raíz, lo tomamos como tal para indicar que es el nodo principal de donde parten sus
  expresiones.}
  corresponde al operador, y cada uno de sus hijos representa un ``sub-árbol'' asociado a las
  expresiones que son parte de la operación. Cada una de estas expresiones debe ser a su vez un \textbf{ASA}
  válido, y en última instancia debe corresponder a un \textbf{ASA} de tipo \textit{Num}.

  \hspace{-1.2cm}
  \begin{minipage}[t]{0.45\textwidth}
    \textit{Add}($e_1, e_2, \cdots, e_n$) es un \textbf{ASA} si cada $e_i$ es un \textbf{ASA}. Es un árbol $n$-$ario$.
    \[
    \frac{e_1, e_2, \dots, e_n \text{\textbf{ ASA}}}{\textit{Add}(e_1, e_2, \dots, e_n)\ \textbf{ASA}}
    \]
    \vspace{0.5em}
    \[
    \begin{forest}
      for tree={
        math content,
        edge={-}, % solo lineas
        s sep=10pt, % separacion horizontal
        l sep=12pt, % separacion vertical
        align=center,
      }
      [Add
        [ASA
          [$\vdots$
            [Num
              [$k_1$]
            ]
          ]
        ]
        [ASA
          [$\vdots$
            [Num
              [$k_2$]
            ]
          ]
        ]
        [ $\vdots$ ]
        [ASA
          [$\vdots$
            [Num
              [$k_{n-1}$]
            ]
          ]
        ]
        [ASA
          [$\vdots$
            [Num
              [$k_n$]
            ]
          ]
        ]
      ]
    \end{forest}
    \]
  \end{minipage}
  \hfill
  \begin{minipage}[t]{0.45\textwidth}
    \textit{Sub}($e_1, e_2, \cdots, e_n$) es un \textbf{ASA} si cada $e_i$ es un \textbf{ASA}.
    \[
    \frac{e_1, e_2, \dots, e_n \text{\textbf{ ASA}}}{\textit{Sub}(e_1, e_2, \dots, e_n)\ \textbf{ASA}}
    \]
    \vspace{0.5em}
    \[
    \begin{forest}
      for tree={ math content, edge={-}, s sep=10pt, l sep=12pt, align=center }
      [Sub
        [ASA [$\vdots$ [Num [$k_1$] ] ] ]
        [ASA [$\vdots$ [Num [$k_2$] ] ] ]
        [ $\vdots$ ]
        [ASA [$\vdots$ [Num [$k_{n-1}$] ] ] ]
        [ASA [$\vdots$ [Num [$k_n$]    ] ] ]
      ]
    \end{forest}
    \]
  \end{minipage}

  \vspace{2cm}
  
  \hspace{-1.2cm}
  \begin{minipage}[t]{0.45\textwidth}
    \textit{Mul}($e_1, e_2, \cdots, e_n$) es un \textbf{ASA} si cada $e_i$ es un \textbf{ASA}.
    \[
    \frac{e_1, e_2, \dots, e_n \text{\textbf{ ASA}}}{\textit{Mul}(e_1, e_2, \dots, e_n)\ \textbf{ASA}}
    \]
    \vspace{0.5em}
    \[
    \begin{forest}
      for tree={ math content, edge={-}, s sep=10pt, l sep=12pt, align=center }
      [Mul
        [ASA [$\vdots$ [Num [$k_1$] ] ] ]
        [ASA [$\vdots$ [Num [$k_2$] ] ] ]
        [ $\vdots$ ]
        [ASA [$\vdots$ [Num [$k_{n-1}$] ] ] ]
        [ASA [$\vdots$ [Num [$k_n$]    ] ] ]
      ]
    \end{forest}
    \]
  \end{minipage}
  \hfill
  \begin{minipage}[t]{0.45\textwidth}
    \textit{Div}($e_1, e_2, \cdots, e_n$) es un \textbf{ASA} si cada $e_i$ es un \textbf{ASA}.
    \[
    \frac{e_1, e_2, \dots, e_n \text{\textbf{ ASA}}}{\textit{Div}(e_1, e_2, \dots, e_n)\ \textbf{ASA}}
    \]
    \vspace{0.5em}
    \[
    \begin{forest}
      for tree={ math content, edge={-}, s sep=10pt, l sep=12pt, align=center }
      [Sub
        [ASA [$\vdots$ [Num [$k_1$] ] ] ]
        [ASA [$\vdots$ [Num [$k_2$] ] ] ]
        [ $\vdots$ ]
        [ASA [$\vdots$ [Num [$k_{n-1}$] ] ] ]
        [ASA [$\vdots$ [Num [$k_n$]    ] ] ]
      ]
    \end{forest}
    \]
  \end{minipage}

  \vspace{0.8cm}
  
\item No variádicos
  
  En su representación abstracta, estos operadores se modelan como árboles \textit{unarios}, en los cuales el
  \textit{nodo raíz} contiene la etiqueta del operador y posee exactamente un hijo, el cual representa la
  expresión sobre la que opera. Del mismo modo, la última instancia debe corresponder a un \textbf{ASA} de tipo
  \textit{Num}.

  \vspace{1cm}
  
  \hspace{-1.2cm}
  \begin{minipage}[t]{0.45\textwidth}
    \textit{Add1}($e$) es un \textbf{ASA} si $e$ es un \textbf{ASA}.
    \[
    \frac{e \text{\textbf{ ASA}}}{\textit{Add1}(e)\ \textbf{ASA}}
    \]
    \vspace{0.5em}
    \[
    \begin{forest}
      for tree={ math content, edge={-}, s sep=10pt, l sep=12pt, align=center }
      [Add1
        [ASA [$\vdots$ [Num [$n$] ] ] ]
      ]
    \end{forest}
    \]
  \end{minipage}
  \hfill
  \begin{minipage}[t]{0.45\textwidth}
    \textit{Sub1}($e$) es un \textbf{ASA} si $e$ es un \textbf{ASA}.
    \[
    \frac{e \text{\textbf{ ASA}}}{\textit{Add1}(e)\ \textbf{ASA}}
    \]
    \vspace{0.5em}
    \[
    \begin{forest}
      for tree={ math content, edge={-}, s sep=10pt, l sep=12pt, align=center }
      [Sub1
        [ASA [$\vdots$ [Num [$n$] ] ] ]
      ]
    \end{forest}
    \]
  \end{minipage}

  \vspace{2cm}
  
  \hspace{-1.2cm}
  \begin{minipage}[t]{0.45\textwidth}
    \textit{Sqrt}($e$) es un \textbf{ASA} si $e$ es un \textbf{ASA}.
    \[
    \frac{e \text{\textbf{ ASA}}}{\textit{Sqrt}(e)\ \textbf{ASA}}
    \]
    \vspace{0.5em}
    \[
    \begin{forest}
      for tree={ math content, edge={-}, s sep=10pt, l sep=12pt, align=center }
      [Sqrt
        [ASA [$\vdots$ [Num [$n$] ] ] ]
      ]
    \end{forest}
    \]
  \end{minipage}
  \hfill
  \begin{minipage}[t]{0.45\textwidth}
    \textit{Expt}($e$) es un \textbf{ASA} si $e$ es un \textbf{ASA}.
    \[
    \frac{e \text{\textbf{ ASA}}}{\textit{Expt}(e)\ \textbf{ASA}}
    \]
    \vspace{0.5em}
    \[
    \begin{forest}
      for tree={ math content, edge={-}, s sep=10pt, l sep=12pt, align=center }
      [Expt
        [ASA [$\vdots$ [Num [$n$] ] ] ]
      ]
    \end{forest}
    \]
  \end{minipage}
  
\end{itemize}

%%%%%%%%%%%%%%%%%%%%%%%%%%%%%%%%%

\subsection{Predicados y comparaciones}
Aquí también tenemos el caso donde la negación es unaria.

\begin{minipage}[t]{0.9\textwidth}
  \textit{Not}($p$) es un \textbf{ASA} si $p$ es un \textbf{ASA}. Debe concluir en un \textbf{ASA} de tipo
  \textit{Boolean}.
  \[
  \frac{p \text{\textbf{ ASA}}}{\textit{Not}(p)\ \textbf{ASA}}
  \]
  \vspace{0.5em}
  \[
  \begin{forest}
    for tree={ math content, edge={-}, s sep=10pt, l sep=12pt, align=center }
    [Not
      [ASA [$\vdots$ [Boolean [$n$] ] ] ]
    ]
  \end{forest}
  \]
\end{minipage}

Para el resto de expresiones, son \textbf{ASA} $n$-$arios$ y cada uno debe terminar con \textbf{ASA} de tipo
\textit{Num}:

\vspace{1cm}
\hspace{-1.2cm}
\begin{minipage}[t]{0.45\textwidth}
  \textit{Equal}($e_1, e_2, \cdots, e_n$) es un \textbf{ASA} si cada $e_i$ es un \textbf{ASA}.
  \[
  \frac{e_1, e_2, \dots, e_n \text{\textbf{ ASA}}}{\textit{Equal}(e_1, e_2, \dots, e_n)\ \textbf{ASA}}
  \]
  \vspace{0.5em}
  \[
  \begin{forest}
    for tree={ math content, edge={-}, s sep=10pt, l sep=12pt, align=center }
    [Equal
      [ASA [$\vdots$ [Num [$k_1$] ] ] ]
      [ASA [$\vdots$ [Num [$k_2$] ] ] ]
      [ $\vdots$ ]
      [ASA [$\vdots$ [Num [$k_{n-1}$] ] ] ]
      [ASA [$\vdots$ [Num [$k_n$]    ] ] ]
    ]
  \end{forest}
  \]
\end{minipage}
\hfill
\begin{minipage}[t]{0.45\textwidth}
  \textit{Diff}($e_1, e_2, \cdots, e_n$) es un \textbf{ASA} si cada $e_i$ es un \textbf{ASA}.
  \[
  \frac{e_1, e_2, \dots, e_n \text{\textbf{ ASA}}}{\textit{Diff}(e_1, e_2, \dots, e_n)\ \textbf{ASA}}
  \]
  \vspace{0.5em}
  \[
  \begin{forest}
    for tree={ math content, edge={-}, s sep=10pt, l sep=12pt, align=center }
    [Diff
      [ASA [$\vdots$ [Num [$k_1$] ] ] ]
      [ASA [$\vdots$ [Num [$k_2$] ] ] ]
      [ $\vdots$ ]
      [ASA [$\vdots$ [Num [$k_{n-1}$] ] ] ]
      [ASA [$\vdots$ [Num [$k_n$]    ] ] ]
    ]
  \end{forest}
  \]
\end{minipage}

\newpage

\hspace{-1.2cm}
\begin{minipage}[t]{0.45\textwidth}
  \textit{Less}($e_1, e_2, \cdots, e_n$) es un \textbf{ASA} si cada $e_i$ es un \textbf{ASA}.
  \[
  \frac{e_1, e_2, \dots, e_n \text{\textbf{ ASA}}}{\textit{Less}(e_1, e_2, \dots, e_n)\ \textbf{ASA}}
  \]
  \vspace{0.5em}
  \[
  \begin{forest}
    for tree={ math content, edge={-}, s sep=10pt, l sep=12pt, align=center }
    [Less
      [ASA [$\vdots$ [Num [$k_1$] ] ] ]
      [ASA [$\vdots$ [Num [$k_2$] ] ] ]
      [ $\vdots$ ]
      [ASA [$\vdots$ [Num [$k_{n-1}$] ] ] ]
      [ASA [$\vdots$ [Num [$k_n$]    ] ] ]
    ]
  \end{forest}
  \]
\end{minipage}
\hfill
\begin{minipage}[t]{0.45\textwidth}
  \textit{Greater}($e_1, e_2, \cdots, e_n$) es un \textbf{ASA} si cada $e_i$ es un \textbf{ASA}.
  \[
  \frac{e_1, e_2, \dots, e_n \text{\textbf{ ASA}}}{\textit{Greater}(e_1, e_2, \dots, e_n)\ \textbf{ASA}}
  \]
  \vspace{0.5em}
  \[
  \begin{forest}
    for tree={ math content, edge={-}, s sep=10pt, l sep=12pt, align=center }
    [Greater
      [ASA [$\vdots$ [Num [$k_1$] ] ] ]
      [ASA [$\vdots$ [Num [$k_2$] ] ] ]
      [ $\vdots$ ]
      [ASA [$\vdots$ [Num [$k_{n-1}$] ] ] ]
      [ASA [$\vdots$ [Num [$k_n$]    ] ] ]
    ]
  \end{forest}
  \]
\end{minipage}

\vspace{1cm}

\hspace{-1.2cm}
\begin{minipage}[t]{0.45\textwidth}
  \textit{Leq}($e_1, e_2, \cdots, e_n$) es un \textbf{ASA} si cada $e_i$ es un \textbf{ASA}.
  \[
  \frac{e_1, e_2, \dots, e_n \text{\textbf{ ASA}}}{\textit{Leq}(e_1, e_2, \dots, e_n)\ \textbf{ASA}}
  \]
  \vspace{0.5em}
  \[
  \begin{forest}
    for tree={ math content, edge={-}, s sep=10pt, l sep=12pt, align=center }
    [Leq
      [ASA [$\vdots$ [Num [$k_1$] ] ] ]
      [ASA [$\vdots$ [Num [$k_2$] ] ] ]
      [ $\vdots$ ]
      [ASA [$\vdots$ [Num [$k_{n-1}$] ] ] ]
      [ASA [$\vdots$ [Num [$k_n$]    ] ] ]
    ]
  \end{forest}
  \]
\end{minipage}
\hfill
\begin{minipage}[t]{0.45\textwidth}
  \textit{Geq}($e_1, e_2, \cdots, e_n$) es un \textbf{ASA} si cada $e_i$ es un \textbf{ASA}.
  \[
  \frac{e_1, e_2, \dots, e_n \text{\textbf{ ASA}}}{\textit{Geq}(e_1, e_2, \dots, e_n)\ \textbf{ASA}}
  \]
  \vspace{0.5em}
  \[
  \begin{forest}
    for tree={ math content, edge={-}, s sep=10pt, l sep=12pt, align=center }
    [Geq
      [ASA [$\vdots$ [Num [$k_1$] ] ] ]
      [ASA [$\vdots$ [Num [$k_2$] ] ] ]
      [ $\vdots$ ]
      [ASA [$\vdots$ [Num [$k_{n-1}$] ] ] ]
      [ASA [$\vdots$ [Num [$k_n$]    ] ] ]
    ]
  \end{forest}
  \]
\end{minipage}

%%%%%%%%%%%%%%%%%%%%%%%%%%%%%%%%%

\subsection{Expresiones Let y Términos $\lambda$}
Para estas reglas tenemos un caso especial, ya que no todas las expresiones en cada regla involucradas son
variádicas. A diferencia de los operadores aritméticos, aquí las estructuras del \textbf{ASA} reflejan la
manera en que se manejan los entornos y la aplicación de funciones dentro de nuestro lenguaje\minilisp\hspace{-0.2cm}.\\

En primer lugar, las expresiones de tipo \texttt{let}, \texttt{let*} y \texttt{letrec} se utilizan para
introducir nuevas asociaciones de variables dentro de un entorno local. Demanera informal, una expresión
\texttt{let} consta de tres elementos básicos: \textbf{identificadores}, \textbf{valores} y un \textbf{cuerpo}.
\\

Como se pudo ver en la \textbf{Sintaxis Concreta}, las tres expresiones \texttt{let} tienen el par de
(\textbf{identificador} \textbf{valor}) y una tercera expresión que vendría siendo el \textbf{body}, con la
característica de que \texttt{let} y \texttt{let*} tienen el par de asignación variádico.\\

Cada una de estas construcciones se representa en el \textbf{ASA} mediante una lista de pares (\texttt{Var},
\texttt{ASA}), donde cada par asocia un identificador con su correspondiente expresión. De este modo, el
analizador sintáctico puede reconstruir de manera estructurada las relaciones entre las variables y sus valores
dentro del entorno.\\

En particular:
\begin{itemize}
\item \texttt{let} define un nuevo entorno donde las variables se evalúan en paralelo (\textbf{alcance estático}).
\item \texttt{let*} permite una evaluación secuencial, donde las definiciones anteriores pueden ser utilizadas
  en las siguientes (\textbf{alcance dinámico}).
\item \texttt{letrec} introduce definiciones recursivas, es decir, variables que pueden hacer referencia a sí
  mismas dentro de sus expresiones. En este caso, el \textbf{ASA} no es variádico, ya que su estructura se
  limita a dos componentes bien definidos: la lista de asociaciones y el cuerpo de la expresión.
\end{itemize}

Estos identificadores deben ser \textbf{ASA} de tipo \texttt{String}.

Por lo que una vez explicado lo anterior, tenemos las siguientes reglas:

\vspace{1cm}

\begin{minipage}[t]{0.9\textwidth}
  \textit{LetRec}($i, v, b$) es un \textbf{ASA} si cada identifcador $i$ es de tipo \texttt{String}, el valor
  $v$ y el cuerpo $b$ son todos \textbf{ASA}.
  \[
  \frac{i \text{\texttt{: String}}\;\; v, b \text{\textbf{ ASA}}}{\textit{LetRec}(i, v, b)\ \textbf{ASA}}
  \]
  \vspace{0.5em}
  \[
  \begin{forest}
    for tree={ math content, edge={-}, s sep=10pt, l sep=12pt, align=center }
    [LetRec
      [String [$s$] ]
      [ASA [$\vdots$] ]
      [ASA [$\vdots$] ]
    ]
  \end{forest}
  \]
\end{minipage}

\vspace{1cm}

\hspace{-1.2cm}
\begin{minipage}[t]{0.45\textwidth}
  \textit{Let}(($i_1, v_1$), ($i_2, v_2$), $\cdots$, ($i_n, v_n$), $b$) es un \textbf{ASA} si cada identificador
  $i_j$ es de tipo \texttt{String} y cada valor $v_j$ y cuerpo $b$ son todos \textbf{ASA}.
  \[
  \frac{i_1, i_2 \cdots, i_n \text{\texttt{: String}}\;\; v_1, v_2, \cdots, v_n, b \text{\textbf{ ASA}}}{\textit{Let}((i_1, v_1), (i_2, v_2), \cdots, (i_n, v_n), b)\ \textbf{ASA}}
  \]
  \vspace{0.5em}
  \[
  \begin{forest}
    for tree={ math content, edge={-}, s sep=10pt, l sep=12pt, align=center }
    [Let
      [String [$s_1$] ]
      [$ASA_1$ [$\vdots$] ]
      [ $\vdots$ ]
      [String [$s_n$] ]
      [$ASA_n$ [$\vdots$] ]
      [ASA [$\vdots$] ]
    ]
  \end{forest}
  \]
\end{minipage}
\hfill
\begin{minipage}[t]{0.45\textwidth}
  \textit{LetStar}(($i_1, v_1$), ($i_2, v_2$), $\cdots$, ($i_n, v_n$), $b$) es un \textbf{ASA} si cada
  identificador $i_j$ es de tipo \texttt{String} y cada valor $v_j$ y cuerpo $b$ son todos \textbf{ASA}.
  \[
  \frac{i_1, i_2 \cdots, i_n \text{\texttt{: String}}\;\; v_1, v_2, \cdots, v_n, b \text{\textbf{ ASA}}}{\textit{LetStar}((i_1, v_1), (i_2, v_2), \cdots, (i_n, v_n), b)\ \textbf{ASA}}
  \]
  \vspace{0.5em}
  \[
  \begin{forest}
    for tree={ math content, edge={-}, s sep=10pt, l sep=12pt, align=center }
    [LetStar
      [String [$s_1$] ]
      [$ASA_1$ [$\vdots$] ]
      [ $\vdots$ ]
      [String [$s_n$] ]
      [$ASA_n$ [$\vdots$] ]
      [ASA [$\vdots$] ]
    ]
  \end{forest}
  \]
\end{minipage}

\vspace{1cm}

Por otra parte, tenemos los términos $\lambda$, para dicha implementación recordemos que los términos lambda se
dividen en: \textbf{variables}, \textbf{abstracciones $\lambda$} y la aplicación de funciones.\\

Nuestras funciones \texttt{Lambda} se representan en el \textbf{ASA} como una lista de encabezados y una
expresión que constituye el cuerpo de la función, que vendrían sinedo las \textbf{variables} y
\textbf{abstracciones} $\lambda$.

Cada parámetro debe ser un identificador válido, por lo que en el \textbf{ASA} estos terminan \textbf{ASA} de
tipo \textit{String}.
Con este diseño nos permitimos modelar funciones con múltiples argumentos de manera flexible, ya que el número
de parámetros puede variar según la definición.\\

Finalmente, para la aplicación de funciones (\texttt{App}).

Dada la expresión de una aplicación de función $e_0$ con $n$ expresiones ($e_1$, $\ldots$, $e_2$), se dice que
$e_0$ es la posición de la función lambda y que cada $e_i$ están en la posición de argumentos de la función. De
esta forma, una aplicación representa el proceso de evaluar una función con sus respectivos $n$ argumentos. 
\\

De este modo, la estructura del \textbf{ASA} distingue entre dos partes:
\begin{enumerate}
\item Como explicamos, la primer expresión representa la función que será aplicada (intuitivamente una expresión \texttt{Lambda} en nuestro lenguaje), esta no es variádica.
\item La segunda expresión corresponde a un lista de $n$ argumentos sobre los cuales se aplicará la función anterior, y sí es variádica, ya que puede contener un número arbitrario ($n$) de expresiones.
\end{enumerate}

De esta forma, en el \textbf{árbol de sintaxis abstracta} conservamos una representación fiel de la sintaxis
para después implementar la semántica funcional de nuestro lenguaje.

Las cuales definimos como sigue:\\

\hspace{-1.2cm}
\begin{minipage}[t]{0.45\textwidth}
  \textit{Lambda}($i_1, i_2, \cdots, i_n$, $c$) es un \textbf{ASA} si cada identificador $i_j$ es de tipo \texttt{String} que representa el nombre de su parámetro y $b$ es \textbf{ASA} que representa el cuerpo.
  \[
  \frac{i_1, i_2 \cdots, i_n \text{\texttt{: String}}\;\; b \text{\textbf{ ASA}}}{\textit{Lambda}(i_1, i_2, \cdots, i_n, b)\ \textbf{ASA}}
  \]
  \vspace{0.5em}
  \[
  \begin{forest}
    for tree={ math content, edge={-}, s sep=10pt, l sep=12pt, align=center }
    [Lambda
      [String [$s_1$] ]
      [String [$s_2$] ]
      [ $\vdots$ ]
      [String [$s_n$] ]
      [ASA [$\vdots$] ]
    ]
  \end{forest}
  \]
\end{minipage}
\hfill
\begin{minipage}[t]{0.45\textwidth}
  \textit{App}($f, e_1, e_2, \cdots, e_n$) es un \textbf{ASA} si $f$ que representa la función que se aplicará
  es un \textbf{ASA} y cada expresión $e_i$ (los argumentos) son todos \textbf{ASA}.
  \[
  \frac{f \textbf{ ASA}\; e_1, e_2, \cdots, e_n \text{\textbf{ ASA}}}{\textit{App}(e_0, e_1, e_2, \cdots, e_n)\ \textbf{ASA}}
  \]
  \vspace{0.5em}
  \[
  \begin{forest}
    for tree={ math content, edge={-}, s sep=10pt, l sep=12pt, align=center }
    [App
      [ASA [$\vdots$] ]
      [$ASA_1$ [$\vdots$] ]
      [$ASA_2$ [$\vdots$] ]
      [ $\vdots$ ]
      [$ASA_n$ [$\vdots$] ]
    ]
  \end{forest}
  \]
\end{minipage}

%%%%%%%%%%%%%%%%%%%%%%%%%%%%%%%%%

\subsection{Pares ordenados y Proyecciones}
Para los pares ordenados y sus proyecciones basta con tener en cuenta las siguientes reglas:

\hspace{-1.2cm}
\begin{minipage}[t]{0.45\textwidth}
  \textit{Pair}($f, s$) es un \textbf{ASA} si las expresiones $f$ y $s$ son \textbf{ASA}.
  \[
  \frac{f \text{\textbf{ ASA}}\; s \text{\textbf{ ASA}}}{\textit{Pair}(f, s)\ \textbf{ASA}}
  \]
  \vspace{0.5em}
  \[
  \begin{forest}
    for tree={ math content, edge={-}, s sep=10pt, l sep=12pt, align=center }
    [Pair
      [ASA [$\vdots$] ]
      [ASA [$\vdots$] ]
    ]
  \end{forest}
  \]
\end{minipage}
\hfill
\begin{minipage}[t]{0.45\textwidth}
  \textit{Fst}($p$) y \textit{Snd}($p$) son \textbf{ASA} si la expresión $p$ es \textbf{ASA}.
  \[
  \frac{p \text{\textbf{ ASA}}}{\textit{Fst}(p)\ \textbf{ASA}}
  \hfill
  \frac{p \text{\textbf{ ASA}}}{\textit{Snd}(p)\ \textbf{ASA}}
  \]
  \vspace{0.5em}
  \[
  \begin{forest}
    for tree={ math content, edge={-}, s sep=10pt, l sep=12pt, align=center }
    [Fst / Snd
      [ASA [$\vdots$] ]
    ]
  \end{forest}
  \]
\end{minipage}

%%%%%%%%%%%%%%%%%%%%%%%%%%%%%%%%%

\subsection{Condicionales}
Tenemos dos condicionales \texttt{if0} e \texttt{if}, ambas son similares en cuanto sintaxis abstracta y por
ende, en cuanto a su \textbf{ASA}:

\hspace{-1.2cm}
\begin{minipage}[t]{0.45\textwidth}
  \textit{If0}($c, t, e$) es un \textbf{ASA} si $c$, $t$ y $e$ son \textbf{ASA}.
  \[
  \frac{c \textbf{ ASA} t \textbf{ ASA} e \textbf{ ASA}}{\textit{If0}(c, t, e)\ \textbf{ASA}}
  \]
  \[
  \begin{forest}
    for tree={ math content, edge={-}, s sep=10pt, l sep=12pt, align=center }
    [If 0
      [ASA]
      [ASA]
      [ASA]
    ]
  \end{forest}
  \]
\end{minipage}
\hfill
\begin{minipage}[t]{0.45\textwidth}
  \textit{If}($c, t, e$) es un \textbf{ASA} si $c$, $t$ y $e$ son \textbf{ASA}.
  \[
  \frac{c \textbf{ ASA} t \textbf{ ASA} e \textbf{ ASA}}{\textit{If0}(c, t, e)\ \textbf{ASA}}
  \]
  \[
  \begin{forest}
    for tree={ math content, edge={-}, s sep=10pt, l sep=12pt, align=center }
    [If
      [ASA]
      [ASA]
      [ASA]
    ]
  \end{forest}
  \]
\end{minipage}

\newpage

No obstante, la condicional \texttt{cond} al ser variádica, genera un \textbf{ASA} al menos $tri$-$ario$, pero
puede variar en más ramas:

\begin{minipage}[t]{0.9\textwidth}
  \textit{Cond}(($c_1, t_1$), ($c_2, t_2$), $\ldots$, ($c_n, t_n$), $e$) es un \textbf{ASA} si cada par de $c_i$,
  $t_i$ son todos \textbf{ASA} y la expresión $e$ también es \textbf{ASA}.
  \[
  \frac{c_1, t_1, c_2, t_2, \ldots, c_n, t_n,\: e \textbf{ ASA}}{\textit{Cond}((c_1, t_1), (c_2, t_2), \ldots, (c_n, t_n), e)\ \textbf{ASA}}
  \]
  \vspace{0.5em}
  \[
  \begin{forest}
    for tree={ math content, edge={-}, s sep=10pt, l sep=12pt, align=center }
    [Cond
      [$ASA_1$]
      [$ASA_2$]
      [$\vdots$]
      [$ASA_n$]
      [ASA]
    ]
  \end{forest}
  \]
\end{minipage}

%%%%%%%%%%%%%%%%%%%%%%%%%%%%%%%%%

\subsection{Listas}
Por último pero no menos importante tenemos los siguientes \textbf{ASA} para las listas.

\begin{minipage}[t]{0.9\textwidth}
  \textit{List}($e_1, e_2, \cdots, e_n$) es un \textbf{ASA} si cada expresión $e_i$ es un \textbf{ASA}. Es un
  árbol $n$-$ario$.
    \[
    \frac{e_1, e_2, \dots, e_n \text{\textbf{ ASA}}}{\textit{List}(e_1, e_2, \dots, e_n)\ \textbf{ASA}}
    \]
    \vspace{0.5em}
    \[
    \begin{forest}
      for tree={ math content, edge={-}, s sep=10pt, l sep=12pt, align=center }
      [List
        [ASA [$\vdots$] ]
        [ASA [$\vdots$] ]
        [ $\vdots$ ]
        [ASA [$\vdots$] ]
      ]
    \end{forest}
    \]
  \end{minipage}

En cambio \texttt{head} y \texttt{tail}, solo requieren una expresión en nuestro lenguaje.

\hspace{-1.2cm}
\begin{minipage}[t]{0.45\textwidth}
  \textit{Head}($l$) es un \textbf{ASA} si $l$ es un \textbf{ASA}, en particular una lista.
  \[
  \frac{l \textbf{ ASA}}{\textit{Head}(l)\ \textbf{ASA}}
  \]
  \[
  \begin{forest}
    for tree={ math content, edge={-}, s sep=10pt, l sep=12pt, align=center }
    [Head
      [ASA [$vdots$] ]
    ]
  \end{forest}
  \]
\end{minipage}
\hfill
\begin{minipage}[t]{0.45\textwidth}
  \textit{Tail}($l$) es un \textbf{ASA} si $l$ es un \textbf{ASA}, en particular una lista.
  \[
  \frac{l \textbf{ ASA}}{\textit{Tail}(l)\ \textbf{ASA}}
  \]
  \[
  \begin{forest}
    for tree={ math content, edge={-}, s sep=10pt, l sep=12pt, align=center }
    [Tail
      [ASA [$vdots$] ]
    ]
  \end{forest}
  \]
\end{minipage}

%%%%%%%%%%%%%%%%%%%%%%%%%%%%%%%%%

\section{ASA en Haskell}

Las etiquetas que hemos definido para cada expresión, funcionarán como los constructores de nuestro tipo de
dato en Haskell, ya que modelamos el tipo de dato \textbf{ASA} (\textit{Árbol de Sintaxis Abstracta}) en Haskell
mediante el tipo algebraico de datos, así es más sencillo expresar la variedad de formas que pueden adoptar las
expresiones en\minilisp\hspace{-0.2cm}.\\

Definimos el tipo de dato \textbf{ASA} en Haskell para\minilisp\hspace{-0.2cm} en el archivos \texttt{ASA.hs}
como sigue:

\bigskip

\begin{lstlisting}[style=haskellstyle, caption={Tipo de dato ASA con azúcar}]
data ASA
  = Var String
  | Num Int
  | Boolean Bool
  | Add [ASA]
  | Sub [ASA]
  | Mul [ASA]
  | Div [ASA]
  | Add1 ASA
  | Sub1 ASA
  | Sqrt ASA
  | Expt ASA
  | Not ASA
  | Equal [ASA]
  | Less [ASA]
  | Greater [ASA]
  | Diff [ASA]
  | Leq [ASA]
  | Geq [ASA]
  | Pair ASA ASA
  | Fst ASA
  | Snd ASA
  | Let [(String, ASA)] ASA
  | LetRec String ASA ASA
  | LetStar [(String, ASA)] ASA
  | If0 ASA ASA ASA
  | If ASA ASA ASA
  | Lambda [String] ASA
  | App ASA [ASA]
  | List [ASA]
  | Head ASA
  | Tail ASA
  | Cond [(ASA, ASA)] ASA
  deriving (Show, Eq)
\end{lstlisting}

\bigskip

Con este diseño reflejamos directamente la estructura lógica del lenguaje que hemos definido en las reglas
gramaticales y sus respectivas reglas para \textbf{ASA}.

\begin{itemize}
\item \textbf{Expresiones atómicas.}
  Los constructores \texttt{Var}, \texttt{Num} y \texttt{Boolean} representan las expresiones más simples del
  lenguaje. Cada una encapsula directamente un valor del tipo correspondiente en Haskell: \texttt{String},
  \texttt{Int}, \texttt{Bool}. Y como explicamos, constituyen las \textit{hojas} del \textbf{ASA}, pues no se
  descomponen en subexpresiones.

\item \textbf{Operadores aritméticos y Not.}  
  Para los operadores \textit{variádicos} (\texttt{Add}, \texttt{Sub}, \texttt{Mul}, \texttt{Div}, etc.), se
  implementó el uso de listas de expresiones \texttt{[ASA]}, de modo que cada operador pueda aplicarse a un
  número arbitrario de argumentos.
  
  De este modo, podemos representar eficientemente construcciones como:
  \[
  (+\ 1\ 2\ 3\ 4) \Rightarrow \texttt{Add [Num 1, Num 2, Num 3, Num 4]}
  \]
  En cambio, los operadores \textbf{unarios} (\texttt{Add1}, \texttt{Sub1}, \texttt{Sqrt}, \texttt{Expt}, \texttt{Not}) reciben únicamente un argumento \texttt{ASA}, justo como los hemos definido para el lenguaje.
  
\item \textbf{Expresiones \texttt{Let}}
  Se representan como una lista de pares \texttt{[(String, ASA)]} para las expresiones \texttt{let} y
  \texttt{let*}, donde cada par vincula el nombre de la variable con la expresión que se le asigna.
  El segundo argumento \texttt{ASA} representa el cuerpo en el que dichas variables estarán disponibles.
  
  Por ejemplo:
  \[
  \hspace{-1.5cm}
  (\texttt{let ((x 2) (y 3)) (+ x y)}) \Rightarrow \texttt{Let [("x", Num 2), (``y'', Num 3)] (Add [Var "x", Var ``y''])}
  \]

\item \textbf{Expresiones condicionales.}
  Los constructores \texttt{If0} e \texttt{If} representan las estructuras condicionales del lenguaje. Cada una
  contiene tres subexpresiones: la condición, la rama verdadera y la rama falsa.
  Por su parte, el constructor \texttt{Cond} modela una estructura condicional más general, en la que se evalúan
  múltiples condiciones.  
  Este se representa como una lista de pares \texttt{[(ASA, ASA)]}, donde cada par asocia una condición con su
  expresión correspondiente a ejecutar en caso de que se cumpla, permitiendo así una evaluación secuencial de
  casos.
  
  Tenemos por ejemplo:
  \[
  (\texttt{if 0 1 -1}) \Rightarrow \texttt{If0} (\texttt{Num 0}) (\texttt{Num 1}) (\texttt{Num -1})
  \]
  \[
  (\texttt{cond [}(\texttt{> x 0})\texttt{ 1] [}(\texttt{< x 0})\texttt{ 2] [else 3]}) \Rightarrow \texttt{Cond [}(\texttt{Greater [Var "x", Num 0], Num 1}) (\texttt{Less [Var "x", Num 0], Num 2})] (\texttt{Num 3})
  \]
  
\item \textbf{Funciones y aplicación.}
  La abstracción lambda (\texttt{Lambda}) se define con una lista de encabezados \texttt{[String]} y un cuerpo
  \texttt{ASA}. La aplicación de funciones (\texttt{App}) se modela mediante dos expresiones formados por la
  expresión que representa la función y una lista de argumentos \texttt{[ASA]}, que serán evaluados y aplicados
  en orden.
  
  Por ejemplo:
  \[
  (\texttt{lambda} (\texttt{x y}) (\texttt{+ x y})) \Rightarrow \texttt{Lambda ["x",``y'']} (\texttt{Add [Var "x", Var ``y''])}
  \]
  \[
  \hspace{-1.8cm}
  ((\texttt{lambda} (\texttt{x y}) (\texttt{+ x y})) \texttt{2 3}) \Rightarrow \texttt{App} (\texttt{Lambda [``x'', ``y'']} (\texttt{Add [Var ``x'', Var ``y'']}) \texttt{[Num 2, Num 3]})
  \]
  
\item \textbf{Listas y operaciones sobre listas.}  
  Las listas se modelan naturalmente como \texttt{List [ASA]}, donde cada elemento de la lista es a su vez un
  \textbf{ASA}. Los constructores \texttt{Head} y \texttt{Tail} representan las operaciones de acceso al primer
  elemento y al resto de la lista, respectivamente, y poseen un único hijo que corresponde a la lista sobre la
  que se aplican.
  
  Un ejemplo sería:
  \[
  (\texttt{[1, 2, 3, 4]}) \Rightarrow \texttt{List [Num 1, Num 2, Num 3, Num 4]} 
  \]
  
\item \textbf{Pares y proyecciones.}  
  Los constructores \texttt{Pair}, \texttt{Fst} y \texttt{Snd} permiten representar estructuras de pares
  ordenados, así como las operaciones para obtener su primer y segundo elemento. Estos casos son binarios y
  unarios, respectivamente, de acuerdo con su aridad.

  Por ejemplo:
  \[
  ((\texttt{+ 3 2})\texttt{,} (\texttt{- 2 1})) \Rightarrow \texttt{Pair} (\texttt{Add}(\texttt{Num 3, Num 2})
  \texttt{Sub}(\texttt{Num 2, Num 1}))
  \]
  
\end{itemize}

Produndizaremos más en esto y veremos las pruebas de que nuestra implementación es correcta en la siguiente
sección.

%%%%%%%%%%%%%%%%%%%%%%%%%%%%%%%%%

\section{Parser con Happy}
Happy es es un sistema generador de \textbf{analizadores sintácticos} (\textit{parsers}) para Haskell. Su
función es transformar una especificación de una \textbf{gramática libre de contexto}, escrita en una notación
similar a \textbf{BNF}, en un módulo de Haskell que implementa automáticamente el parser correspondiente.\\

En nuestro caso, Happy toma como entrada el conjunto de \textit{Tokens} producidos por nuestro \texttt{Lexer}
en Alex, y construye a partir de ellos una estructura de ASA (definida en \texttt{ASA.hs}), que representa el
programa en\minilisp\hspace{-0.2cm}.\\

La idea central al usar Happy es definir formalmente la gramática de\minilisp\hspace{-0.2cm} y dejar que Happy
genere el código que realice el proceso de parseo. Esto nos permite integrar el módulo generado
(\texttt{Grammar.hs}) al resto del proyecto, de modo que podamos transformar cualquier secuencia de \textit{Tokens} válida en una estructura \textbf{ASA} con azúcar. De este modo podemos desazucarala sin problemas y una vez desazucarada, tener la estructura semántica lista para ser evaluada por nuestro intérprete.\\

En Happy, cada producción de la gramática se asocia con una acción semántica en Haskell, que construye el nodo correspondiente en el \textbf{ASA}. De este modo, Happy no solo valida la estructura del programa, sino que también construye automáticamente la representación estructural.\\

Mostramos a continuación nuestra implementación del \textbf{analizador léxico} con Happy.

\bigskip

\begin{lstlisting}[style=haskellstyle, caption={Parser de Gramática con Happy.}]
{
module Grammar where

import Lexer
import Token
import ASA
}

%name parse
%tokentype { Token }
%error { parseError }

...

{
parseError :: [Token] -> a
parseError _ = error "Parser Error"
}
\end{lstlisting}

\bigskip
  
No hay mucho que enfatizar en el comienzo y definición de errores del parser, solo es sintaxis de Happy:

\begin{itemize}
\item En el encabezado Haskell, definimos el módulo que Happy generará y los imports necesarios.
\begin{itemize}
\item \texttt{Lexer} provee el flujo de \textit{Tokens}, la entrada al parser.
\item \texttt{Token} define los tipos de \textit{Tokens} reconocidos por el analizador léxico.
\item \texttt{ASA} contiene las definiciones de los constructores de los \textbf{ASA}.
\end{itemize}

\item \texttt{$\%$name parse}: indica el nombre de la función principal que Happy generará. Esta es de la forma:
\[
\texttt{parse :: [Token]} \rightarrow \texttt{ASA}
\]
y será el punto de entrada del parser.

\item \texttt{$\%$tokentype \{ Token \}}: especifica el tipo de dato que Happy debe esperar como entrada
  (nuestros \textit{Tokens}).

\item Y con $\%$error \{ parseError \} definimos la función que se ejecutará en caso de error sintáctico.
  
  En este caso, parseError simplemente lanza una excepción con el mensaje "\textit{Parser Error}", indicando que
  la cadena no cumple la gramática definida. esta función la definimos al final del archivo \texttt{Grammar.y}.
\end{itemize}

Una vez inicializamos el parser en Happy, continuamos con la declaración de \textit{Tokens}:

\bigskip

\begin{lstlisting}[style=haskellstyle, caption={Declaración de Tokens en Happy.}]
%token
  var             { TokenVar $$ }
  num             { TokenNum $$ }
  boolean         { TokenBool $$ }
  '('             { TokenPA }
  ')'             { TokenPC }
  '['             { TokenLI }
  ']'             { TokenLD }
  ','             { TokenComma }
  '+'             { TokenAdd }
  '-'             { TokenSub }
  '*'             { TokenMul }
  '/'             { TokenDiv }
  '='             { TokenEq }
  '<'             { TokenLt }
  '>'             { TokenGt }
  "!="            { TokenNeq }
  "<="            { TokenLeq }
  ">="            { TokenGeq }
  "++"            { TokenAdd1 }
  "--"            { TokenSub1 }
  "**"            { TokenExpt }
  "sqrt"          { TokenSqrt }
  "not"           { TokenNot }
  "if0"           { TokenIf0 }
  "if"            { TokenIf }
  "first"         { TokenFst }
  "second"        { TokenSnd }
  "let"           { TokenLet }
  "letrec"        { TokenLetRec }
  "let*"          { TokenLetStar }
  "head"          { TokenHead }
  "tail"          { TokenTail }
  "lambda"        { TokenLambda }
  "cond"          { TokenCond }
  "else"          { TokenElse }
\end{lstlisting}

\bigskip

En esta sección declaramos los \textit{Tokens} terminales que Happy reconocerá como símbolos del lenguaje. Son los elementos básicos que el parser reconocerá.
Cada entrada de \texttt{$\%$token} indica cómo un token léxico (producido por el \textbf{analizador léxico} \texttt{Lexer}) se relaciona con un nombre dentro de la gramática.

Para nuestros valores, por ejemplo:\texttt{var \{ TokenVar $\$\$$ \}}, indicamos que cuando el lexer produzca un \texttt{TokenVar "x"}, el parser reconocerá \texttt{var} y podrá acceder al valor $x$ a través de $\$\$$.

Los símbolos reservados o caracteres especiales del lenguaje los denotamos comillas simples '' y las palabras reservadas con comillas dobles "", asociando cada una con su respectivo \texttt{Token}.

Esta correspondencia permite que Happy comprenda la estructura léxica y la relacione con la estructura sintáctica del lenguaje\minilisp\hspace{-0.2cm}.\\

Continuando, tenemos las reglas de producción del lenguaje, sección delimitada por $\%\%$ donde definimos cómo se construye el \textbf{ASA} a partir de los \textit{Tokens}.

Intuitivamente, esta sección es prácticamente igual a nuestra definición de la gramática para\minilisp\hspace{-0.2cm}, por lo que en Happy cada regla tiene la forma:

\[ \text{NoTerminal : simbolos\_de\_produccion \{ accion\_semantica \}}\]

Donde:
\begin{itemize}
\item \textbf{NoTerminal} es una categoría gramatical, como \texttt{ASA}.
\item \textbf{simbolos\_de\_produccion} son tokens o no terminales.
\item \{ \textbf{accion\_semantica} \} es código Haskell que construye el nodo del \textbf{ASA} correspondiente.
\end{itemize}
  
Ya con esto podemos definir las reglas de la gramática y producción del \textbf{ASA}.

\begin{lstlisting}[style=haskellstyle, caption={Reglas principales de la gramática con Happy.}]
%%
  
ASA
  : var                                       { Var $1 }
  | num                                       { Num $1 }
  | boolean                                   { Boolean $1 }
  | '(' '+' opArgs ')'                        { Add (reverse $3) }
  | '(' '-' opArgs ')'                        { Sub (reverse $3) }
  | '(' '*' opArgs ')'                        { Mul (reverse $3) }
  | '(' '/' opArgs ')'                        { Div (reverse $3) }
  | '(' '=' opArgs ')'                        { Equal (reverse $3) }
  | '(' '<' opArgs ')'                        { Less (reverse $3) }
  | '(' '>' opArgs ')'                        { Greater (reverse $3) }
  | '(' "!=" opArgs ')'                       { Diff (reverse $3) }
  | '(' "<=" opArgs ')'                       { Leq (reverse $3) }
  | '(' ">=" opArgs ')'                       { Geq (reverse $3) }
  | '(' "++" ASA ')'                          { Add1 $3 }
  | '(' "--" ASA ')'                          { Sub1 $3 }
  | '(' "sqrt" ASA ')'                        { Sqrt $3 }
  | '(' "**" ASA ')'                          { Expt $3 }
  | '(' "not" ASA ')'                         { Not $3 }
  | '(' ASA ',' ASA ')'                       { Pair $2 $4 }
  | '(' "first" ASA ')'                       { Fst $3 }
  | '(' "second" ASA ')'                      { Snd $3 }
  | '(' "let" '(' ids ')' ASA ')'             { Let (reverse $4) $6 }
  | '(' "letrec" '(' var ASA ')' ASA ')'      { LetRec $4 $5 $7 }
  | '(' "let*" '(' ids ')' ASA ')'            { LetStar (reverse $4) $6 }
  | '(' "if0" ASA ASA ASA ')'                 { If0 $3 $4 $5 }
  | '(' "if" ASA ASA ASA ')'                  { If $3 $4 $5 }
  | '(' "lambda" '(' vars ')' ASA ')'         { Lambda (reverse $4) $6 }
  | '(' ASA appArgs ')'                       { App $2 (reverse $3) }
  | '(' '[' listArgs ']' ')'                  { List (reverse $3) }
  | '(' "head" ASA ')'                        { Head $3 }
  | '(' "tail" ASA ')'                        { Tail $3 }
  | '(' "cond" condis '[' "else" ASA ']' ')'  { Cond (reverse $3) $6 }
\end{lstlisting}

\bigskip

El uso de \texttt{reverse} es importante pues, durante el análisis, Happy construye las listas en orden inverso
por eficiencia (debido a la \textit{recursión por izquierda}). Aplicar reverse al final restaura el orden
original de los argumentos según fueron escritos por el usuario.\\

Nótese además que, para algunas producciones definimos nuevas reglas, estas reglas las definimos para llevar un mejor control de su \textbf{análisis sintáctico}. Las podemos ver como sigue:

\bigskip

\begin{lstlisting}[style=haskellstyle, caption={Reglas auxiliares para la gramática.}]
opArgs
  : ASA ASA                               { [$2, $1] }
  | opArgs ASA                            { $2 : $1 }

ids
  : id                                    { [$1] }
  | ids id                                { $2 : $1 }

id
  : '(' var ASA ')'                       { ($2, $3) }
  
vars
  : var                                   { [$1] }
  | vars var                              { $2 : $1 }

appArgs
  : ASA                                   { [$1] }
  | appArgs ASA                           { $2 : $1 }

listArgs
  : {- empty -}                           { [] }
  | ASA                                   { [$1] }
  | listArgs ',' ASA                      { $3 : $1 }

condis
  : condy                                 { [$1] }
  | condis condy                          { $2 : $1 }

condy
  : '[' ASA ASA ']'                       { ($2, $3) }
\end{lstlisting}

\bigskip

Estas reglas complementarias definen la estructura interna de las construcciones del lenguaje:
\begin{itemize}
\item \texttt{opArgs}: permite operadores aritméticos con un número variable de argumentos.
  La forma \texttt{[$\$$2, $\$$1]} y \texttt{($\$$2 : $\$$1)} implementa recursión por izquierda.
  Gracias a ello, el parser consume la entrada de forma eficiente, usando espacio constante en la pila.
  Luego, \texttt{reverse} corrige el orden de evaluación.
  Con esta regla nos aseguramos que los operadores aritméticos dados por el usuario tengan al menos dos
  expresiones \texttt{ASA} \texttt{ASA} para ser válidos.
  
\item \texttt{ids} e \texttt{id}: definimos los pares \texttt{(var ASA)} de las expresiones \texttt{let}, \texttt{id} verifica que sea un par correcto y \texttt{ids} acumula los pares variádicos.
  
\item \texttt{vars}: define las listas de variables, lo usamos para la lista de encabezados para la expresión
  \texttt{lambda}.
  
\item \texttt{appArgs}: generamos una lista de expresiones, los argumentos de la aplicación de función.
  
\item \texttt{listArgs} manejamos los elementos de la lista, permitiendo la lista vacia y por otro lado el
  manejo de la lista con sus elementos separados por comas.
  
\item \texttt{condis} y \texttt{condy}: podemos definir la estructura variádica para \texttt{cond}, además de
  establecer que las expresiones están delimitadas por corchetes.
\end{itemize}

De este modo hemos definimos correctamente las reglas para el parser y obtenemos correctamente los \textbf{ASA}
de cada expresión.\\

La razón por la cual usamos \textit{recursión izquierda} en Happy es porque así podemos definir las reglas de
producción de forma más eficiente, ya que construimos el resultado del parser en una sola pasada utilizando
espacio constante en la pila. Mientras que la \textit{recursión por derecha} podría incrementar la complejidad
de tiempo y memoria de manera significativa.

Por ello utilizamos \textit{recursión por izquierda} de la forma \texttt{\{ [$\$$2, $\$$1]\}} y \texttt{\{$\$$2 : $\$$1\}}, así construimos la lista de expresiones variádicas hacia la izquierda y, apoyándonos de \texttt{reverse}, hacemos que esta lista de expresiones vuelva al orden de como el usuario escribió las reglas.\\

\bigskip

\noindent
Y así, como hemos visto en el caítulo, el estudio y la construcción de la \textbf{sintaxis abstracta}
representan uno de los pasos más importantes en el diseño de un lenguaje de programación (como es el caso con
nuestra implementación de\minilisp\hspace{-0.2cm}). A través del \textit{Árbol de Sintaxis Abstracta (ASA)},
logramos capturar la estructura esencial de los programas sin los elementos superficiales de la \textbf{sintaxis
  concreta}, lo que nos permite trabajar directamente con la lógica y la jerarquía de las expresiones. Esta
representación no solo facilita el análisis y la transformación del código, sino que también hace posible
implementar de manera más limpia y coherente cada parte del lenguaje.\\

\noindent
Sin embargo, nuestro\minilisp\hspace{-0.2cm} aún no está listo para pasar directamente a la etapa del intérpretE. Aunque el \textbf{ASA} ya elimina mucho del "ruido" sintáctico, todavía contiene construcciones que, si las evaluáramos directamente, requerirían una gran cantidad de reglas semánticas adicionales. Por eso, antes de interpretar, necesitamos un paso intermedio: el proceso de \textbf{desazucarización}, distinguir la \textbf{azúcar sintáctica} y eliminarla.\\

\noindent
Lo que buscamos es simplificar aún más nuestra \textbf{sintaxis abstracta}, transformando las construcciones más complejas o redundantes en formas más básicas que el intérprete pueda manejar de manera uniforme.

%%%%%%%%%%%%%%%%%%%%%%%%%%%%%%%%%%

% ----- Azúcar Sintáctica -----
%%%%%%%%%%%%%%%%%%%%%%%%%%%%%%%%%%
\chapter{Azúcar Sintáctica}
El término \textit{azúcar sintáctica} \textit{(syntactic sugar)} fue introducido por Peter J.Landin en 1964. La azúcar sintáctica consiste en construcciones del lenguaje que pueden ser sistemáticamente traducidas a formas más básicas sin alterar la semántica del programa. Es decir, no añaden nuevas capacidades expresivas, sino que facilitan la escritura o lectura del código.

Landin fue un pionero de la teoría de lenguajes de programación, utilizó el término para describir aquellas construcciones que, aunque convenientes para el programador, pueden eliminarse mediante una transformación mecánica sin modificar el significado del programa.

\begin{quote}
  Landin uso este termino por primera vez en su trabajo sobre la correspondencia entre programación y cálculo lambda, específicamente en el artículo \textit{"The Next 700 Programming Languages"}~\cite{landin}, aunque la idea aparece en escritos de 1964.
\end{quote}

Continunado con nuestro proyecto. Una vez que hemos elimindo todos los elementos superficilaes y auxiliares de la sintaxis concreta y léxica para quedarnos únicamente con la representación mínima y estructural del programa, es decir, el \textbf{Arbol de Sintaxis Abstracta} (ASA) logramos capturar la estructura esencial del programa. 
Pasamos a la siguiente fase, reducir aún más estos \textbf{ASA} porque a pesar de ya haber eliminado los elementos superficiales de la Sintaxis Concreta, hay expresiones redundantes o que pueden representarse como otras, hacer esto nos reduce en mayor medida las reglas que debemos implementar al momento de la evaluación, en nuestra implementación pasamos de ASA (con azúcar sintáctica) a AST (sin azúcar sintáctica).

Para explicar lo anterior con mas detalle tomaremos un ejemplo:
\[
\texttt{(if0 (+ 8 -8) 0 -1)} \Rightarrow \texttt{(if (= (+ 8 -8) 0) 0 -1)}
\]
\texttt{if0} es análogo a if con la comprobación de que el resultado sea igual a cero, pero con \texttt{if0} el usuario se ahorra hacer cada vez esa comprobación de igualdad, pero esto sigue siendo azúcar para nuestro intérprete.

Estas expresiones \textbf{ASA} sin azúcar sintáctica pertenecen al conjunto que denominamos como \textit{núcleo}, o \textit{core}.

\begin{tcolorbox} [title=\textbf{Corrección 11: Notación $\Rightarrow$}, colframe=red!75!black, colback=red!5!white]

La notación $\Rightarrow$ utilizada en este capítulo no debe confundirse con la empleada en el 
Capítulo 3. En la sección 3, la flecha $\Rightarrow$ la cual ya fue explicada en su respectiva corrección (9).

En cambio, en el presente capítulo la notación $\Rightarrow$ sí representa una \textbf{transformación 
sintáctica real}: la desugarización del ASA (con azúcar sintáctica) hacia un AST más básico 
y uniforme. Aquí la flecha indica una operación semánticamente significativa y definida de 
manera recursiva en nuestra implementación.

Para evitar confusiones, aclaramos que ambos usos corresponden a procesos distintos: 
(1) la construcción del ASA desde la sintaxis concreta, y (2) la eliminación de azúcar 
sintáctica mediante reglas de desugarización.
\end{tcolorbox}

%%%%% 4.1 Sintaxis Abstracta sin azúcar %%%% 
\section{Sintaxis Abstracta sin azúcar en\minilisp}

Necesitamos definir una función que realice el procedimiento e desazucarar los \textbf{ASA} en núcleos. Por lo que nombramos a esta función como \texttt{desugar} y queda definida como sigue:
\[
\texttt {desugar :: =} \Rightarrow \texttt{Sugared\_ASA} \rightarrow \texttt{Desugared\_ASA}
\]
  
\begin{itemize}
\item \textbf{Variables:}
  Las variables no necesitan desazucarizarse pues ya son expresiones atómicas, ya pertenecen al núcleo, simplemente renombramos a ASA sin azúcar preservando el valor.
  \begin{center}
  \texttt{desugar}(\texttt{SugarVar i}) $\Rightarrow$ \texttt{Var i}\\
  \texttt{desugar}(\texttt{SugarNum n}) $\Rightarrow$ \texttt{Num n}\\
  \texttt{desugar}(\texttt{SugarBool b}) $\Rightarrow$ \texttt{Bool b}
  \end{center}

\begin{tcolorbox} [title=\textbf{Corrección 12: Corrección sobre SugarVar }, colframe=red!75!black, colback=red!5!white]
Fue un error de notación en el texto. Intenté diferenciar visualmente el ASA del AST usando el prefijo Sugar.

    \begin{center}
    \texttt{desugar}(\texttt{Var } $i$) $\Rightarrow$ \texttt{VarC } $i$\\
    \texttt{desugar}(\texttt{Num } $n$) $\Rightarrow$ \texttt{NumC } $n$\\
    \texttt{desugar}(\texttt{Boolean } $b$) $\Rightarrow$ \texttt{BoolC } $b$
    \end{center}
\end{tcolorbox}
  
\item \textbf{Operadores:}
  Nuestros operadores en su gran mayoría son variádicos, preservan la lista de ASA (los operandos), esto es azúcar sintáctica para el intérprete ya que nuestros operadores \texttt{+}, \texttt{-}, \texttt{*} y \texttt{/} son binarios. Por lo que podemos representar a los operadores variádicos como un encademiento del mismo. Por ejemplo:
  \begin{center}
  \texttt{desugar}(\texttt{SugarAdd[} $n_1,\:n_2,\ldots,\:n_k$\texttt{]}) $\Rightarrow$ \texttt{Add} $n_1$ (\texttt{Add} $n_2 \ldots$ (\texttt{Add} $n_{k-1}\; n_k$))
  \end{center}
  
\begin{tcolorbox} [title=\textbf{Corrección 13: Transformación composicional no entendida }, colframe=red!75!black, colback=red!5!white]
\item \textbf{Operadores:}
  Nuestros operadores en el ASA son variádicos (\texttt{Add}), preservan la lista de operandos. Esto es azúcar sintáctica para el intérprete, ya que en el núcleo (\texttt{AST}) nuestros operadores \texttt{AddC}, \texttt{SubC}, \texttt{MulC} y \texttt{DivC} son binarios.
  Por lo que representamos la desazucarización como un plegado de la lista. Por ejemplo:

  \[
  \texttt{desugar}(\texttt{Add }[e_1, e_2, \ldots, e_n]) \Rightarrow \texttt{AddC}(\texttt{desugar}(e_1), \texttt{desugar}(\texttt{Add}[e_2, \ldots, e_n]))
  \]
  
\end{tcolorbox}
  

  Para el caso de los operadores unarios, estos ya de por sí son azúcar sintáctica a excepción de Sqrt el cuál es un operador único por lo que ya es \textit{núcleo}. En el caso de \texttt{Add1} $n$ y \texttt{Sub1} $n$ podemos reexpreasarlos como  $n + 1$ y $n - 1$ respectivamente, y de manera similar con \texttt{Expt}, dado que en nuestro lenguaje representa elevar al cuadrado el número $n$, entonces \texttt{Expt} es azúcar sitáctica y lo reexpresamos como una multiplicación se $n \times n$:
  \begin{center}
  \texttt{desugar}(\texttt{Add1 } $n$) $\Rightarrow$ \texttt{Add} $n$ $1$\\
  \texttt{desugar}(\texttt{SugarSqrt } $n$) $\Rightarrow$ \texttt{Sqrt} $n$\\
  \texttt{desugar}(\texttt{Expt} $n$) $\Rightarrow$ \texttt{Mul} $n$ $n$
  \end{center}


\begin{tcolorbox} [title=\textbf{Corrección 14: Formalización de constructores con el prefijo Sugar}, colframe=red!75!black, colback=red!5!white]

Para el caso de los operadores unarios, algunos son azúcar pura, a excepción de \texttt{Sqrt}, el cual se transforma directamente al constructor del núcleo \texttt{SqrtC}.
  En el caso de \texttt{Add1} $n$ y \texttt{Sub1} $n$, podemos reexpresarlos como $n + 1$ y $n - 1$ utilizando los operadores binarios del núcleo y constantes numéricas. De manera similar con \texttt{Expt}, que representa elevar al cuadrado ($n \times n$):

  \begin{center}
  \texttt{desugar}(\texttt{Add1 } $n$) $\Rightarrow$ \texttt{AddC} (\texttt{desugar } $n$) (\texttt{NumC } 1)\\
  \texttt{desugar}(\texttt{Sub1 } $n$) $\Rightarrow$ \texttt{SubC} (\texttt{desugar } $n$) (\texttt{NumC } 1)\\
  \texttt{desugar}(\texttt{Sqrt } $n$) $\Rightarrow$ \texttt{SqrtC} (\texttt{desugar } $n$)\\
  \texttt{desugar}(\texttt{Expt } $n$) $\Rightarrow$ \texttt{MulC} (\texttt{desugar } $n$) (\texttt{desugar } $n$)
  \end{center}

\end{tcolorbox}

\item \textbf{Comparadores:}
  En el caso de los comparadores, es muy similar su representación en \textbf{ASA} sin azúcar como lo fue para los operadores variádicos, ya que estos también lo son. El núcleo de cada comparadaor es el mismo pues forzosamente tenemos que definir uno  para cada uno de ellos, ya que necesitamos preservar el tipo de comparación. Sin embargo, a diferencia de los operadores, no es conveniente representarlos como un encademiento de comparadores, pues al evaluar la comparación entre dos números, el resultado es de tipo \texttt{Bool}, sino como un encademiento de condicionales if.
  Donde la condición inicial es la comparación de los dos primeros argumentos y el consecuente son las comparaciones de la argumentos restantes, si alguna de las condiciones no se cumple entonces caemos en el else False y de otro modo las comparaciones son válidas y el resultado es \texttt{True}:

  \begin{center}
    \texttt{desugar}(\texttt{SugarEqual[} $n_1,\:n_2,\ldots,\:n_k$\texttt{]}) $\Rightarrow$ \texttt{If} (\texttt{Equal} $n_1$ $n_2$) (\texttt{Equal} $n_2$ $n_3$) $\ldots$ (\texttt{Equal} $n_{k-1}$ $n_k$) (\texttt{Bool False})
  \end{center}

\begin{tcolorbox} [title=\textbf{Corrección 15: Transformación correcta del SugarEqual}, colframe=red!75!black, colback=red!5!white]

La transformación correcta es un \textbf{encadenamiento de condicionales anidados} (\texttt{IfC}). La lógica es: comparamos los dos primeros elementos; si son iguales, procedemos a comparar el segundo con el tercero (en la rama \textit{then}), y así sucesivamente. Si alguna comparación falla, caemos inmediatamente en la rama \textit{else} con \texttt{BoolC False}.

  \begin{center}
    \texttt{desugar}(\texttt{Equal [} $n_1, n_2, \ldots, n_k$ \texttt{]}) $\Rightarrow$ \\
    \texttt{IfC} (\texttt{EqualC} $n_1$ $n_2$)
      (\texttt{IfC} (\texttt{EqualC} $n_2$ $n_3$) (\ldots \texttt{BoolC True}) (\texttt{BoolC False}))
      (\texttt{BoolC False})
  \end{center}

  De este modo, respetamos que el constructo \texttt{IfC} del núcleo es ternario y no variádico.

\end{tcolorbox}

\item \textbf{Not y Pares:}
\texttt{Not} y las \textbf{ASA} sobre pares \texttt{Pair}, \texttt{Fst} y \texttt{Snd} ya son núcleos, no hace falta definir una desazucarización específica.

\begin{tcolorbox} [title=\textbf{Corrección 16: Pares y Not}, colframe=red!75!black, colback=red!5!white]
\item \textbf{Not y Pares:}
  Aunque los constructos \texttt{Not}, \texttt{Pair}, \texttt{Fst} y \texttt{Snd} tienen una correspondencia directa en el núcleo, en la implementación \textbf{sí} es necesario definir una regla de transformación explícita.
  Esto se debe a que primero, debemos traducir del tipo de dato de la superficie (\texttt{ASA}) al tipo del núcleo (\texttt{AST}) cambiando los constructores (por ejemplo, de \texttt{Pair} a \texttt{PairC}) y ademas es indispensable propagar la desazucarización a sus sub-expresiones recursivamente.

  \begin{center}
    \texttt{desugar}(\texttt{Not } $e$) $\Rightarrow$ \texttt{NotC} (\texttt{desugar } $e$)\\
    \texttt{desugar}(\texttt{Pair } $e_1$ $e_2$) $\Rightarrow$ \texttt{PairC} (\texttt{desugar } $e_1$) (\texttt{desugar } $e_2$)\\
    \texttt{desugar}(\texttt{Fst } $e$) $\Rightarrow$ \texttt{FstC} (\texttt{desugar } $e$)\\
    \texttt{desugar}(\texttt{Snd } $e$) $\Rightarrow$ \texttt{SndC} (\texttt{desugar } $e$)
  \end{center}
  
\end{tcolorbox}

\item \textbf{Condicionales:}
  \texttt{If0} y \texttt{Cond} solo son azúcar sintáctica de \texttt{If}. \texttt{If0} como mencionamos es comprobar que el resultado al terminar de evaluarse sea igual a cero. Mientras que \texttt{Cond} es igualmente un encadenamiento de \texttt{If}. Por ello estas tres expresiones las representamos como un único núcleo If:

  \begin{center}
    \texttt{desugar}(\texttt{If0 c t e}) $\Rightarrow$ \texttt{If} (\texttt{Equal c 0}) t e\\
    \texttt{desugar}(\texttt{Cond [$x_1$ $e_1$] [$x_2$ $e_2$] $\ldots$ [$x_n$ $e_n$] [else $e_k$]}) $\Rightarrow$ \texttt{If} ($x_1$) $e_1$ (If ($x_2$) $n_2$) $\ldots$ (If ($x_n$) $e_n$ ($e_k$))
  \end{center}

\begin{tcolorbox} [title=\textbf{Corrección 17: Desugar IfO falta de desarrollo}, colframe=red!75!black, colback=red!5!white]

  Para \texttt{If0}, la transformación consiste en crear un \texttt{IfC} donde la condición es una comparación explícita con cero. Es crucial notar que debemos aplicar \texttt{desugar} recursivamente a las sub-expresiones ($c, t, e$) para asegurar que todo el árbol resultante pertenezca al núcleo.

  \begin{center}
    \texttt{desugar}(\texttt{If0 } $c\ t\ e$) $\Rightarrow$ \\
    \texttt{IfC} (\texttt{EqualC} (\texttt{desugar } $c$) (\texttt{NumC } 0)) (\texttt{desugar } $t$) (\texttt{desugar } $e$)
  \end{center}

  Por otro lado, \texttt{Cond} se desazucara como un encadenamiento de \texttt{IfC} anidados. Cada condición se transforma en un nivel del anidamiento, y el caso \texttt{else} final se convierte en la última rama alternativa.

  \begin{center}
    \texttt{desugar}(\texttt{Cond } [($c_1, e_1$), ($c_2, e_2$), $\dots$] \texttt{else } $e_{final}$) $\Rightarrow$ \\
    \texttt{IfC} (\texttt{desugar } $c_1$) (\texttt{desugar } $e_1$) \\
      (\texttt{IfC} (\texttt{desugar } $c_2$) (\texttt{desugar } $e_2$) (\dots (\texttt{desugar } $e_{final}$)\dots))
  \end{center}
\end{tcolorbox}
  
\item \textbf{Lets:}
  Los \texttt{Let} son solo la azúcar de la aplicación de funciones donde la sustitucion del valor con el identificador, el par (\textit{id, value}) es el argumento, y el cuerpo del \texttt{let} es la función que se va a aplicar. Por otro lado, \texttt{LetStar} es azúcar sintácica de \texttt{Let}, de modo que \texttt{LetStar} se reexpresa como \texttt{lets} anidados.
  
  Por lo que el proceso de desazucarización sería similar a:

  \begin{center}
    \texttt{desugar}(\texttt{Let}($id$, $value$) $body$)  $\Rightarrow$ \texttt{App} ($body$) ($id$, $value$)\\
    \texttt{desugar}(\texttt{LetStar}($id_1$, $value_1$) ($id_2$, $value_2$) $\ldots$ ($id_n$, $value_n$) $body$) $\Rightarrow$ \texttt{Let} (($id_1$, $value_1$) (\texttt{Let} ($id_2$, $value_2$) $\ldots$ (\texttt{Let}($id_n$, $value_n$) $body$)))
  \end{center}

\begin{tcolorbox} [title=\textbf{Corrección 18: Lets y LetStar desglosado}, colframe=red!75!black, colback=red!5!white]
\item \textbf{Lets:}
  Los \texttt{Let} son azúcar sintáctica para la aplicación de funciones. Semánticamente, \texttt{let x = v in body} es equivalente a definir una función anónima $(\lambda x. body)$ y aplicarla al valor $v$.
  Por lo tanto, la desazucarización transforma el constructo en una aplicación (\texttt{AppC}) de una función (\texttt{FunC}), asegurando aplicar \texttt{desugar} recursivamente tanto al cuerpo como al valor ligado.

  \begin{center}
    \texttt{desugar}(\texttt{Let } [($i, v$)] $b$) $\Rightarrow$ \texttt{AppC} (\texttt{FunC } $i$ (\texttt{desugar } $b$)) (\texttt{desugar } $v$)
  \end{center}

  Por otro lado, \texttt{LetStar} es azúcar sintáctica de \texttt{Let} anidados (evaluación secuencial). Su transformación no genera directamente un AST final, sino que se reescribe como un \texttt{Let} que contiene otro \texttt{LetStar} (o el cuerpo final), y se vuelve a invocar \texttt{desugar} sobre el resultado de esta expansión.

  \begin{center}
    \texttt{desugar}(\texttt{LetStar } [($i_1, v_1$), ($i_2, v_2$), \dots] $b$) $\Rightarrow$ \\
    \texttt{desugar}(\texttt{Let } [($i_1, v_1$)] (\texttt{LetStar } [($i_2, v_2$), \dots] $b$))
  \end{center}

\end{tcolorbox}
  
Pasando al caso específico de \texttt{letrec}, este nos permite definir funciones recursivas locales. Formalmente, la recursión puede desazucararse utilizando un operador de punto fijo. Para lenguajes de evaluación ansiosa se utiliza el operador Z, que implementa recursión sin necesidad de auto-referencia explícita\cite{Baren-Lambda}.

\begin{center}
\texttt{desugar}(\texttt{LetRec}($id$, $value$)$body$) $\Rightarrow$ \texttt{Let} (($id$, App (Lambda ($id$) $value$)(Lambda ($id$) $value$)))
$body$
\end{center}

\begin{tcolorbox} [title=\textbf{Corrección 19: Corrigiendo al LetRec}, colframe=red!75!black, colback=red!5!white]

Pasando al caso específico de \texttt{LetRec}, este nos permite definir funciones recursivas locales. Dado que nuestro núcleo no soporta recursión directa (referencias circulares en el entorno), utilizamos el combinador de punto fijo $Z$ para desazucararlo.

En nuestra implementación, asumimos la existencia de un combinador $Z$ definido en el entorno global. La regla de transformación reescribe el \texttt{LetRec} como un \texttt{Let} estándar donde aplicamos $Z$ a la función lambda generada. Es fundamental notar que el resultado de esta reescritura es una expresión en sintaxis concreta (ASA) que debe ser procesada nuevamente por \texttt{desugar}.

\[
\texttt{desugar}(\texttt{LetRec } i\ v\ b) \Rightarrow \texttt{desugar}(\texttt{Let } [(i, \texttt{App } (\texttt{Var } "Z")\ [\texttt{Lambda } [i]\ v])]\ b)
\]

De esta manera, transformamos la recursión implícita en una aplicación explícita del operador de punto fijo, delegando la resolución de la recursión a la semántica del combinador $Z$ durante la evaluación.
\end{tcolorbox}

\begin{quote}
En el trabajo \textit{The Formal of Programming Languages}, el concepto de punto fijo es la herramienta matemática fundamental para dar un significado preciso a las construcciones recurisivas de un lenguaje, como los bucles y las funciones que se llaman a sí mismas~\cite{winskel} 
\end{quote}

\begin{center}
El \textbf{combinador de punto fijo Z} se define en cálculo lambda como:
\[
Z = \lambda f. (\lambda x. f \, (\lambda y. x \, x \, y)) \, (\lambda x. f \, (\lambda y. x \, x \, y))
\]
que satisface la propiedad:
\[
Z \, F = F \, (Z \, F) \quad \text{para cualquier término } F
\] \cite{Baren-Lambda}
\end{center}

\textit{¿Por qué es necesario el combinador Z de punto fijo?}
En la implementación de nuestro MiniLisp, hemos definido las funciones anónimas ($\lambda$) como un núcleo fundamental del lenguaje. Esta decisión se tomo con el fin de poder desazucarar  muchas construcciones del lenguaje.Por ejemplo, let y letstar pueden ser traducidos de forma directa a simples aplicaciones de lambda.

Sin embargo, letrec (la construcción que permite definir funciones recursivas) presenta un desafío.

La recursión, se basa en nombres: una función se llama a sí misma por su propio nombre hasta llegar a un caso base. Pero esto crea una paradoja en nuestro sistema: si el núcleo solo entiende de funciones anónimas, ¿cómo puede una función sin nombre llamarse a sí misma?

Es aquí donde la teoría del punto fijo se vuelve clave.

En lugar de intentar que la función se llame a sí misma (lo cual es imposible sin un nombre), se crea una función "generadora" (un funcional), llamémosla $G$. Esta función $G$ no es recursiva, sino que acepta un argumento. Dicho argumento, llamémoslo $g$, representa la suposición de la función recursiva.

Para solucionar esto es necesario el uso de una "maquina"
que haga esto automáticamente. Esa máquina es un combinador de punto fijo (como el Combinador Y visto en clase).

\begin{tcolorbox} [title=\textbf{Corrección 20: Corrigiendo al LetRec}, colframe=red!75!black, colback=red!5!white]
En lugar de definir una función que se llame a sí misma por nombre, definimos un funcional (una función de orden superior), al que llamaremos $G$. Este funcional $G$ no es recursivo por sí mismo; en su lugar, recibe un argumento extra, digamos $f$, que servirá como el mecanismo para realizar las llamadas futuras. Es decir, $f$ actúa como un placeholder o sustituto de la propia función recursiva.

Sin embargo, $G$ por sí sola no puede ejecutarse indefinidamente. Necesitamos un mecanismo que, en tiempo de ejecución, se encargue de suministrar a $G$ una referencia a sí misma cada vez que sea necesario. Este mecanismo es el Combinador de Punto Fijo (en nuestro caso, el combinador $Z$, adecuado para evaluación estricta). Matemáticamente, el combinador calcula el punto fijo de $G$, construyendo efectivamente la función recursiva al "pasar" la función como argumento a sí misma de manera automática \cite{Friedman1996}
\end{tcolorbox}

\item \textbf{Expresiones lambda:}
  Nuestras expresiones lambda son variádicas, por lo que para representarlas en núcleo necesitamos currificarlas., es decir, necesitamos convertiralas en funciones de un solo argumento:
  
  \begin{center}
    \texttt{desugar}(\texttt{Lambda [$x_1$, $x_2$, $\ldots$ $x_n$]} $b$) $\Rightarrow$ (\texttt{Fun $x_1$ (Fun $x_2$ $\ldots$ (Fun $x_n$} $b$)))
  \end{center}
  
  De igual forma para la aplicación de funciones, ya que \texttt{App} en \textbf{ASA} maneja una lista de argumentos, necesitamos currificar estos argumentos ya que las funciones ya están en forma de un argumento a la vez:
  
  \begin{center}
    \texttt{desugar}(\texttt{App $e$ [$x_1$, $x_2$, $\ldots$ $x_n$]}) $\Rightarrow$ (\texttt{App} (\texttt{App} (\texttt{App} $\ldots$ (\texttt{App} $e$ $x_1$) $x_2$) $\ldots$) $x_n$))
  \end{center}
  
\item \textbf{Listas:}
  Para las listas definimos su desazucaricación con el uso de \texttt{Nil} y \texttt{Cons}, el cuál funciona de manera similar al encadenamiento de pares:
  
  \begin{center}
    \texttt{desugar}(\texttt{List [$x_1$, $x_2$, $\ldots$ $x_n$]}) $\Rightarrow$ (\texttt{Cons $x_1$ (Cons $x_2$ $\ldots$ (Con $x_{n-1}$ $x_n$}))
  \end{center}

\begin{tcolorbox}[title=\textbf{Corrección 21: Lambda y Listas }, colframe=red!75!black, colback=red!5!white]

\item \textbf{Expresiones Lambda:}
  Nuestras expresiones lambda en ASA son variádicas. Para representarlas en el núcleo (AST), necesitamos currificarlas, transformándolas en una serie de funciones unarias anidadas (\texttt{FunC}).
  
  \[
  \texttt{desugar}(\texttt{Lambda } [x_1, x_2, \ldots, x_n]\ b) \Rightarrow \texttt{FunC } x_1 (\texttt{FunC } x_2 \ldots (\texttt{FunC } x_n (\texttt{desugar } b))\ldots)
  \]
  
  De igual forma para la \textbf{aplicación de funciones}, ya que \texttt{App} en \textbf{ASA} maneja una lista de argumentos, necesitamos currificar la aplicación utilizando \texttt{AppC} (que es binaria), respetando la asociatividad por la izquierda y desazucarando tanto la función como cada argumento:
  
  \[
  \texttt{desugar}(\texttt{App } e\ [x_1, \ldots, x_n]) \Rightarrow \texttt{AppC } (\ldots (\texttt{AppC } (\texttt{desugar } e) (\texttt{desugar } x_1)) \ldots) (\texttt{desugar } x_n)
  \]
  
\item \textbf{Listas:}
  Para las listas, definimos su desazucarización utilizando el constructor \texttt{ConS}. Es importante notar que, según nuestra implementación en Haskell (donde el caso base para un solo elemento devuelve el elemento procesado), la estructura resultante es un encadenamiento donde el último elemento es el final de la estructura, sin utilizar \texttt{NiL} como terminador explícito para listas no vacías.
  
  \[
  \texttt{desugar}(\texttt{List } [x_1, \ldots, x_n]) \Rightarrow \texttt{ConS } (\texttt{desugar } x_1) (\texttt{ConS } \ldots (\texttt{ConS } (\texttt{desugar } x_{n-1}) (\texttt{desugar } x_n))\ldots)
  \]

\end{tcolorbox}

De esta manera, la estructura superficial de las listas en la sintaxis concreta, que aparenta ser n-aria se reexpresa en términos de una estructura binaria en el núcleo del lenguaje. En otras palabras, una lista como \texttt{[a,b,c]} no es realmente un nodo con tres hijos, sino una secuencia anidada de constructores
\texttt{Cons}, cada uno tomando dos argumentos: el elemento y la referencia al resto de la lista. 

Esto implica que algunas construcciones que originalmente se representan como árboles n-arios en el \textbf{ASA}, después de ser desazucarizadas corresponden a árboles binarios en el \textbf{AST}. \texttt{Head}, \texttt{Tail} ya no operan sobre listas \textbf{ASA}, sino directamente sobre la estructura \texttt{Nil} y \texttt{Cons}, que constituye el núcleo semántico para listas.

\begin{tcolorbox}[title=\textbf{Corrección 22: Corrección del texto}, colframe=red!75!black, colback=red!5!white]
Esto implica que algunas construcciones que originalmente se representan como árboles n-arios en el ASA, después de ser desazucarizadas corresponden a árboles binarios en el AST. 
\texttt{Head} y \texttt{Tail} ya no operan sobre listas ASA, sino directamente sobre la estructura \texttt{ConS}, que constituye el núcleo semántico principal para las listas en nuestra implementación (donde el final de la lista es el propio elemento y no un terminador explícito).
\end{tcolorbox}

\end{itemize}
  
%%%%%%%%%%%%%%%%%%%%%%%%%%%%%%%%%

\subsection{Desugar en Haskell}

Para nuestro proyecto en \minilisp definimos el siguiente tipo de dato:

\begin{lstlisting}[style=haskellstyle, caption={Tipo de dato ASA sin azúcar, AST}]
module AST where
  
-- ASA sin azucar (AST)
data AST
  = VarC String
  | NumC Int
  | BoolC Bool
  | AddC AST AST
  | SubC AST AST
  | MulC AST AST
  | DivC AST AST
  | SqrtC AST
  | NotC AST
  | EqualC AST AST
  | LessC AST AST
  | GreaterC AST AST
  | DiffC AST AST
  | LeqC AST AST
  | GeqC AST AST
  | PairC AST AST
  | FstC AST
  | SndC AST
  | IfC AST AST AST
  | FunC String AST
  | AppC AST AST
  | ConS AST AST
  | HeadC AST
  | TailC AST
  | NiL
  deriving (Show, Eq)

\end{lstlisting}

\begin{tcolorbox}[title=\textbf{Corrección 23: Nota sobre notación}, colframe=red!75!black, colback=red!5!white]
En las cajas anteriores se corrige los nombres de los constructores.
\end{tcolorbox}

\textbf{AST} (\textit{Abstract Sintaxis Tree}) es nuestro tipo de dato \textbf{ASA} sin azúcar, no hay un razón especial por la que la hayamos nombrado \textbf{AST}, nos pareció práctico y nada más. En esta seccón nos referiremos como \textbf{AST} a nuestra sintaxis abstracta sin azúcar en\minilisp\hspace{-0.2cm}. Notemos que, los tipos de dato que trabajaban sobre listas\\

\begin{tcolorbox}[title=\textbf{Corrección 24: Redacción formal}, colframe=red!75!black, colback=red!5!white]

El \textbf{AST} \textit{Abstract Syntax Tree} constituye nuestro tipo de dato para la representación de \textbf{ASA} (\textit {Árbol de Sintaxis Abstracta}) sin azúcar sintáctico. La elección de la denominación \textbf{AST} responde principalmente a consideraciones prácticas de implementación, sin implicaciones semánticas adicionales. 

\end{tcolorbox}

Anteriormente hicimos una breve mención, casi de manera superficial, de cómo se re-expresan nuestro \textbf{ASA} a \textbf{AST} a través de un función especial conocida como \texttt{desugar}, de tal modo que nos quedamos con las estructuras núcleo y no ahorramos futuras reglas para el intérprete.\\

En nuestro proyecto de\minilisp\hspace{-0.2cm}, definimos la función \texttt{desugar} en el arhcivo \texttt{Desugar.hs} como sigue:

\begin{lstlisting}[style=haskellstyle, caption={Firma y casos base de la función desugar}]
  module Desugar where
  
  import ASA
  import AST
  import ASV
  
  {-- Desazucaramos los ASA  --}
  desugar :: ASA -> AST
  -- Casos base
  desugar (Var x) = VarC x
  desugar (Num n) = NumC n
  desugar (Boolean b) = BoolC b
\end{lstlisting}

La firma de la función refleja nuestro objetivo, dado una estructura \texttt{ASA}, \texttt{desugar} lo procesa hasta obtener un \texttt{AST}. Nótese además que las expresiones atómicas no cambian su estructura, únicamente las renombramos de tipo \texttt{ASA} a \texttt{AST}.

\begin{lstlisting}[style=haskellstyle, caption={Sección de la función \texttt{desugar} para operadores aritméticos}]
  desugar :: ASA -> AST
  -- Operaciones aritmeticas
  desugar (Add xs) = desugarOps AddC xs
  desugar (Sub xs) = desugarOps SubC xs
  desugar (Mul xs) = desugarOps MulC xs
  desugar (Div xs) = desugarOps DivC xs
  desugar (Add1 n) = AddC (desugar n) (NumC 1)
  desugar (Sub1 n) = SubC (desugar n) (NumC 1)
  desugar (Expt n) = MulC (desugar n) (desugar n)
  desugar (Sqrt n) = SqrtC (desugar n)
\end{lstlisting}

Previamente, al hacer mención de la función \texttt{desugar} omitimos explicar que los argumentos de las expresiones deben pasar también por el proceso de desazucarización, sin embargo es necesario definir la desazucarización recursivamente ya que como sabemos, un \textbf{AST} es \textbf{AST} si todos sus hijos lo son.

\begin{tcolorbox}[title=\textbf{Corrección 25: Redacción formal}, colframe=red!75!black, colback=red!5!white]
Previamente, presentamos la función \texttt{desugar} enfocándonos en la transformación del nodo raíz. Sin embargo, para garantizar la correctitud del programa, la definición debe ser recursiva. Dado que el \textbf{AST} es una estructura de árbol, la propiedad de "estar desazucarado" debe cumplirse en cada uno de sus nodos. Por lo tanto, la función se aplica no solo a la expresión principal, sino que se distribuye a través de todos los argumentos de las expresiones compuestas.
\end{tcolorbox}

\noindent
Para ello definimos una función auxiliar \texttt{desugarOps} que generaliza el trabajo de desazucarar las operaciones aritméticas pues estas pasan por el mismo procedimiento solo que cambian la etiqueta de su estructura. Mientras que \texttt{Add1} y \texttt{Sub1} como mencionamos, son azúcar para $n$ \texttt{+} 1 / $n$ \texttt{-} 1 respectivamente, además de que \texttt{Expt} es azúcar de $n \times n$. Por otro lado, \texttt{Sqrt} solo pasa a ser \texttt{SqrtC} además que aplica \texttt{desugar} a su único hijo.

\begin{lstlisting}[style=haskellstyle, caption={Función \texttt{desugarOps} como auxiliar para desazucarar operadores}]
  desugar :: ASA -> AST
  -- Operaciones aritmeticas
  desugar (Add xs) = desugarOps AddC xs
  desugar (Sub xs) = desugarOps SubC xs
  desugar (Mul xs) = desugarOps MulC xs
  desugar (Div xs) = desugarOps DivC xs

  --Funcion auxiliar para desazucarar los operadores
  desugarOps :: (AST -> AST -> AST) -> [ASA] -> AST
  desugarOps _ [] = error "[desugarOps Error]: Lista vacia (no deberia suceder)"
  desugarOps _ [x] = desugar x
  desugarOps op (x:xs) = op (desugar x) (desugarOps op xs)
\end{lstlisting}

La función \texttt{desugarOps} recibe una tupla de \texttt{AST} (\texttt{AST -> AST -> AST}) y una lista de \texttt{ASA} y devuelve un \texttt{AST} donde en la tupla, el primero es la etiqueta asociada al operador que vamos a desazucarar y los otros dos son los hijos del operador, que recordemos, en \texttt{AST} ya son árboles binarios. Y la lista de \texttt{ASA} es la lista de los operandos que vamos a separar.

\begin{tcolorbox}[title=\textbf{Corrección 26: Auxiliar desugarOps}, colframe=red!75!black, colback=red!5!white]
La función \texttt{desugarOps} recibe como primer argumento una función constructora binaria del núcleo \texttt{AST -> AST -> AST} por ejemplo \texttt{AddC o MulC} y como segundo argumento las lista de términos \texttt{ASA} que se deben procesar.
\end{tcolorbox}

\noindent
De esta forma no perdemos la referencia de qué tipo de operador \texttt{AST} estamos desazucarando mientras mantemos una única función \texttt{desugarOps} y así no tenemos que definir una función para cada operador.

\noindent
Tenemos dos casos base para la función, donde \texttt{[ASA]} es vacía, cosa que no debería suceder pues en la gramática definida en Happy justo lo implementamos de modo que los operadores rechacen un número de argumentos inválidos; además de que tampoco se puede llegar a la lista vacía por el siguiente caso base donde si la lista tiene un elemento es donde termina la recursión y devolvemos ese elemento desazucarado con \texttt{desugar}. Por otro lado el paso recursivo es donde tomamos la cabeza de la lista el cual desazucaramos para ser el primer arguemento del operador, mientras que la cola recursivamente se aplica \texttt{desugarOps} y que será el segundo argumento.\\

Continuando con los comparadores, intuitivamente pensamos en implementarlo de igual forma que con los operadores (un encadenamiento de comparadores). Sin embargo, al momento de pensar en su interpretación, nos topamos con el problema de que, al hacer la comparación entre un \texttt{Num n} y \texttt{Num m}, el resultado es de tipo \texttt{Bool}, y esto nos da una inconsistencia de tipos al momento de continuar con las evaluaciones posteriores ya que no es imposible comparar un \texttt{Num} con un \texttt{Bool}.

\noindent
Por ello cambiamos su implementación a encadenamiento de condicionales \texttt{If}, pues es la única forma en nuestro lenguaje de preservar las comparaciones correctas y detectar en donde no se cumple la comparación.

\begin{lstlisting}[style=haskellstyle, caption={Función \texttt{desugarComp} como auxiliar para desazucarar comparadores}]
  desugar :: ASA -> AST
  -- Not
  desugar (Not x) = NotC (desugar x)
  -- Comparaciones
  desugar (Equal xs) = desugarComp EqualC xs
  desugar (Less xs) = desugarComp LessC xs
  desugar (Greater xs) = desugarComp GreaterC xs
  desugar (Diff xs) = desugarComp DiffC xs
  desugar (Leq xs) = desugarComp LeqC xs
  desugar (Geq xs) = desugarComp GeqC xs

  --Funcion auxiliar para desazucarar los comparadores
  desugarComp :: (AST -> AST -> AST) -> [ASA] -> AST
  desugarComp _ [] = BoolC True
  desugarComp _ [_] = BoolC True
  desugarComp op [i, d] = op (desugar i) (desugar d)
  desugarComp op (i:d:is) = IfC (op (desugar i) (desugar d))
                            (desugarComp op (d:is))
                            (BoolC False)
\end{lstlisting}

\begin{tcolorbox}[title=\textbf{Corrección 27: Nota sobre el caso base de la recursión desugarComp}, colframe=red!75!black, colback=red!5!white]
Se podría pensar que el caso base \texttt{desugarComp \_ [\_] = BoolC True} ignora el último elemento de la lista al no aplicar \texttt{desugar} explícitamente sobre él.
  Sin embargo, dada la naturaleza recursiva de nuestra función, este último elemento \textbf{ya ha sido procesado y desazucarado} en el paso anterior de la recursión, donde actuó como el operando derecho de la comparación (el elemento \texttt{d} en el patrón \texttt{i:d:is}).
  Por lo tanto, volver a procesarlo en el caso base sería redundante. Para el caso borde de una comparación con un único argumento (ej. \texttt{(= 5)}), asumimos la verdad vacua sin efectos secundarios, consistente con la simplificación del núcleo.

\textbf{Nota sobre la separación de casos:}
  Aclarando la distinción entre el patrón \texttt{[i, d]} y el caso recursivo general \texttt{(i:d:is)}.
  Aunque teóricamente el patrón general podría capturar listas de dos elementos (donde la cola \texttt{is} sería vacía), esto resultaría en la generación de una estructura condicional redundante de la forma \texttt{IfC (op i d) True False}.
  
  Por esta razón, hemos optado por \textbf{aislar el caso base de dos elementos}. Esta decisión constituye una optimización sintáctica que permite generar directamente el nodo de comparación (\texttt{op i d}) sin envoltorios innecesarios, produciendo un AST más limpio y reduciendo la sobrecarga computacional durante la evaluación.
\end{tcolorbox}

De manera similar como fue con los operadores, definimos una función \texttt{desugarComp} que recibe una tupla de \texttt{AST} para preservar la etiqueta a desazucarar y la lista de elementos a separar que se van a comparar. Los primeros dos casos, son los casos base, donde igualmente, no podemos tener una lista de uno o ningún elemento, pues. La siguiente instrucción sería nuestro caso base real, donde establece que al tener solo dos elementos en la lista, simplemente se devuela la comparación de ambos elementos, mientras que si todavía quedan elementos en la lista, iniciemos la cadena de \texttt{IfC}, donde los primeros dos elementos se comparan en la condición y en caso de cumplirse continuamos en el entonces con la llamada recursiva de \texttt{desugarComp} del segundo elemento con el resto de la lista, y en caso de no cumplirse, el else es \texttt{BoolC False}.

\noindent
Esta separación y comparación de elementos es válida para cualquier lista de $n$ elementos sin importar si $n$ es $2k$ o $2k$ \texttt{-} $1$, es decir, si la lista tiene un número impar o par de elementos; ya que siempre hacemos la comparación de elemento por elemento hasta llegar al caso donde quedan 2 elementos en la lista que es cuando simplemente se devuelve la comparación de ambos.\\

Los pares como se mencionó no es necesario desazucararlos más que recursivamente desazucarar a sus hijos:

\begin{lstlisting}[style=haskellstyle, caption={Desazucarización de los Pares}]
  desugar :: ASA -> AST
  -- Pares
  desugar (Pair f s) = PairC (desugar f) (desugar s)
  desugar (Fst p) = FstC (desugar p)
  desugar (Snd p) = SndC (desugar p)
\end{lstlisting}

Como bien explicamos, las condicionales \texttt{If0} y \texttt{Cond} son azúcar sintáctica de \texttt{If}. \texttt{If0} pasa a \texttt{IfC} con la comprobación de que el valor en la condición sea igual a cero y nada más. 

\begin{lstlisting}[style=haskellstyle, caption={Desazucarización de los condicionales}]
  desugar :: ASA -> AST
  --Condicionales
  desugar (If0 c t e) = IfC (EqualC (desugar c) (NumC 0)) (desugar t) (desugar e)
  desugar (If c t e) = IfC (desugar c) (desugar t) (desugar e)
  desugar (Cond cs e) = desugarCond cs e
\end{lstlisting}

Por otro lado para \texttt{Cond}, tenemos que definir una función auxiliar que nos realice el paso a encadenamiento de condicionales \texttt{IfC}

\begin{lstlisting}[style=haskellstyle, caption={Desazucarización de Cond}]
  desugar :: ASA -> AST
  --Condicionales
  desugar (Cond cs e) = desugarCond cs e

  --Funcion auxiliar para desazucarar el operador cond
  desugarCond :: [(ASA, ASA)] -> ASA -> AST
  desugarCond [] e = desugar e
  desugarCond ((c, t):cs) e = IfC (desugar c) (desugar t) (desugarCond cs e)
\end{lstlisting}

Como se puede ver, la función \texttt{desugarCond} recibe una lista de pares (\textit{condición, expresión}) junto con una expresión final (el caso \texttt{else} implícito). Si la lista de pares es vacía, basta con devolver la expresión por defecto desazucarada. En caso contrario, se construye una estructura \texttt{IfC} donde la primera condición se evalúa en el \textit{\textbf{if}}, la primera expresión en el \textit{\textbf{then}}, y el resto de los pares en el \textit{\textbf{else}}, aplicando recursivamente \texttt{desugarCond}.

\noindent
De esta manera, la estructura \texttt{Cond} se traduce en una sucesión de evaluaciones \texttt{IfC} anidadas, logrando así preservar el mismo comportamiento semántico que tendría en su forma azucarada. Y así garantizamos que sólo se ejecute el cuerpo correspondiente a la primera condición verdadera, respetando la naturaleza secuencial del condicional múltiple.\\

Como se mencionó, los \texttt{Let} en nuestro lenguaje no son construcciones primitivas, sino azúcar sintáctica que se puede expresar completamente a partir de funciones y aplicaciones. Intuitivamente, un \texttt{Let} introduce una variable local asociada a un valor dentro de un cuerpo de expresión. Sin embargo, esta noción de “sustitución” puede modelarse directamente mediante la aplicación de una función anónima a un argumento\cite{Dybving}.

\noindent
Esta equivalencia se refleja directamente en nuestra función auxiliar \texttt{desugarLet}, encargada de traducir cualquier \texttt{Let} del \textbf{ASA} a su correspondiente expresión \textbf{AST} basada en aplicaciones de funciones:

\begin{lstlisting}[style=haskellstyle, caption={Desazucarización de Let y LetStar}]
  desugar :: ASA -> AST
  --Lets
  desugar (Let iv b) = desugarLet iv b
  desugar (LetStar [] body) = desugar body
  desugar (LetStar (iv:ivs) b) = desugar (Let [iv] (LetStar ivs b))

  --Funciones auxiliares para desazucarar let
  desugarLet :: [(String, ASA)] -> ASA -> AST
  desugarLet [] b = desugar b
  desugarLet ((p, v):ps) b = AppC (FunC p (desugarLet ps b)) (desugar v)
\end{lstlisting}

\begin{tcolorbox}[title=\textbf{Corrección 28: Nota sobre la semántica de Let y Let*}, colframe=red!75!black, colback=red!5!white]
  \textbf{Nota sobre la semántica de Let y Let*:}
  Es importante señalar que, debido a la naturaleza recursiva de nuestra función auxiliar \texttt{desugarLet}, las vinculaciones en el constructo \texttt{Let} se generan creando ámbitos anidados para cada variable.
  
  Esto implica que, en nuestra implementación actual, \texttt{Let} adopta una \textbf{semántica secuencial}, comportándose de manera idéntica a \texttt{Let*}. Aunque en Lisp estándar el \texttt{Let} realiza asignaciones paralelas (donde una variable no puede referenciar a otra definida en el mismo bloque), hemos optado por unificar ambos comportamientos. Esta decisión de diseño simplifica considerablemente la lógica de desazucarado, permitiendo utilizar un único mecanismo de abstracción recursiva para ambas estructuras sin perder capacidad expresiva.
\end{tcolorbox}

La función \texttt{desugarLet} recibe una lista de pares \texttt{(id, valor)} y el cuerpo del \texttt{Let}. En el caso base, cuando no hay más pares, simplemente se desazucara el cuerpo, ya que no hay variables locales restantes por introducir. En el caso general, se construye una función anónima con el primer identificador \texttt{p} y cuerpo el resultado de seguir desazucarando los pares restantes junto con el cuerpo \texttt{b}.
A continuación, esta función se aplica (\texttt{AppC}) al valor \texttt{v} correspondiente, el cual también se desazucara antes de la aplicación.\\

\begin{tcolorbox}[title=\textbf{Corrección 29: Sobre los Lets sin bindings}, colframe=red!75!black, colback=red!5!white]
La función \texttt{desugarLet} recibe una lista de pares $(id, valor)$ y el cuerpo del \texttt{Let}.
El caso base de la recursión ocurre cuando la lista de pares es vacía (lo cual corresponde sintácticamente a un \texttt{Let} sin vinculaciones, ej. \texttt{(let () body)}).
En esta situación, dado que no hay nuevas variables que introducir en el entorno, la expresión es semánticamente equivalente al cuerpo mismo, por lo que simplemente se devuelve el resultado de desazucarar dicho cuerpo (\texttt{desugar b}).
\end{tcolorbox}

Además, \texttt{LetStar} -como también se  mencionó en la sección anterior- es simplemente azúcar sintáctica de \texttt{Let}. El \texttt{LetStar} permite escribir múltiples asignaciones secuenciales en un mismo bloque, pero semánticamente equivale a una serie de \texttt{Let} anidados. Su desazucarización se implementa recursivamente, construyendo un \texttt{Let} por cada par \texttt{(id, valor)} y utilizando como cuerpo el siguiente \texttt{LetStar}, hasta llegar al cuerpo final.\\

Por otro lado tenemos el caso del \texttt{LetRec}:

\begin{lstlisting}[style=haskellstyle, caption={Desazucarización de LetRec}]
  desugar :: ASA -> AST
  ---LetRec
  desugar (LetRec i v b) = desugar (Let [(i, App (Var "Z") [Lambda [i] v])] b)
\end{lstlisting}

La construcción \texttt{letrec} permite la definición de funciones recursivas locales.
Dado que el núcleo del lenguaje no incluye recursión como mecanismo primitivo, esta
debe expresarse mediante azúcar sintáctica utilizando construcciones ya presentes,
como funciones anónimas y \texttt{let}.

La desazucarización transforma la expresión:
\[
\texttt{desugar}(\texttt{LetRec } i\ v\ b)
\]
en la siguiente forma equivalente:
\[
\texttt{desugar}(\texttt{Let } [(i, \texttt{App } (\texttt{Var } "Z")\ [\texttt{Lambda } [i]\ v])]\ b)
\]
Este patrón corresponde a la expansión operacional del combinador de punto fijo para
lenguajes con evaluación estricta (combinador \(Z\) mencionado anteriormente). La función \(\lambda i.\, v\) se
aplica a sí misma, permitiendo que \(i\) haga referencia a su propia definición dentro
de \(v\), sin necesidad de introducir recursión directamente en el núcleo semántico del
lenguaje.

\bigskip
De esta manera, \texttt{letrec} se reduce a una combinación de aplicación lambda y
\texttt{let*}, asegurando que el análisis y la ejecución posterior puedan realizarse de
acuerdo con las reglas ya definidas para el lenguaje base. Una vez realizada la
transformación, se invoca nuevamente a \texttt{desugar} para continuar el proceso hasta
obtener una expresión completamente desazucarizada en el árbol AST.

\bigskip

Una vez establecido lo anterior, continuamos con el caso de los \textbf{ASA} \texttt{Lambda} en nuestro lenguaje, operan con una lista de parámetros siendo esto azúcar sintáctica, por lo que definimos la función auxiliar \texttt{desugarLmb} done ''\textit{currificamos}`` la función en \texttt{FunC} que trabaja sobre un parámetro:

\begin{lstlisting}[style=haskellstyle, caption={Desazucarización de las funciones lambda}]
  desugar :: ASA -> AST
  --Expresiones lambda
  desugar (Lambda ps b) = desugarLmb ps b

  --Funcion auxiliar para desazucarar las funcioneslambda
  desugarLmb :: [String] -> ASA -> AST
  desugarLmb [] b = desugar b
  desugarLmb (p:ps) b = FunC p (desugarLmb ps b)
\end{lstlisting}

Si la lista de parámetros está vacía, simplemente se devuelve el cuerpo desazucarado; en caso contrario, se crea un \texttt{FunC} con el primer parámetro y como cuerpo el resultado de desazucarar los parámetros restantes junto con el cuerpo.

De igual manera, las aplicaciones de función en \texttt{ASA} trabajan con una lista de argumentos, por lo que debemos desazucarlo en aplicaciones sucesivas de \texttt{AppC}.

\begin{lstlisting}[style=haskellstyle, caption={Desazucarización de la aplicación de función}]
  desugar :: ASA -> AST
  --Expresiones lambda
  desugar (App f as) = desugarApp (desugar f) as

  --Funcion auxiliar para desazucarar las aplicaciones de funcion
  desugarApp :: AST -> [ASA] -> AST
  desugarApp f [] = f
  desugarApp f (a:as) = desugarApp (AppC f (desugar a)) as
\end{lstlisting}

Donde el caso base es que al quedarnos sin argumentos que desazcuarar, devolvemos la función a aplicar, ya que esta fue desazucarada en \texttt{AST} desde la primer llamada a \texttt{desugarApp}. Por otra parte, si quedan argumentos en la lista continuamos con la llamda recursiva para que formen las aplicaciones sucesivas de \texttt{AppC}.\\

Finalmente tenemos \texttt{desugar} para listas. En un principio, la idea fue implementar las listas como encademaiento de pares, no obstante, esto nos trajo problemas al momento de implementar la función que devuelve el resultado al usuario\footnote{Abordaremos más sobre esta situación en próximos capítulos pero en términos simples usar \texttt{ConS} y \texttt{NiL} mejoró la parte de diferenciar totalementre entre una lista con pares a un par con listas, entre otras combinaciones para poder reflejar correctamente la entrada del usuario sin modificaciones inesperadas.}, pero aunque realizar esta desazucarización requiere de definir cuatro \texttt{AST}, nos facilita el trabajo al momento de evaluar acertadamente lo que el usuario da como programa.

\begin{lstlisting}[style=haskellstyle, caption={Desazucarización de Listas}]
  desugar :: ASA -> AST
  --Listas
  desugar (List l) = desugarList l
  desugar (Head l) = HeadC (desugar l)
  desugar (Tail l) = TailC (desugar l)

  --Funcion auxiliar para construir listas como cons y nil
  desugarList :: [ASA] -> AST
  desugarList [] = NiL
  desugarList [x] = desugar x
  desugarList (x:xs) = ConS (desugar x) (desugarList xs)
\end{lstlisting}

\begin{tcolorbox}[title=\textbf{Corrección 30: Lista de un elemento}, colframe=red!75!black, colback=red!5!white]
Nuestro caso base \texttt{desugarList [x] = desugar x} implica que una lista que contiene un único elemento en el \textbf{ASA} se transforma directamente en dicho elemento atómico en el \textbf{AST}, perdiendo su estructura de lista.
  
Esto es consistente con nuestra implementación de listas impropias. En este esquema, la lista no termina con un marcador vacío explícito (\texttt{NiL}), sino que el último nodo \texttt{ConS} contiene el dato final directamente en su segunda posición. Por lo tanto, una lista de un solo elemento se reduce semánticamente al elemento mismo dentro de esta estructura simplificada, eliminando la necesidad de un nodo terminador adicional.
\end{tcolorbox}

De manera muy similar a como fuimos creando el encadenamiento de operadores o de condicionales \texttt{IfC} para los comparadores, en este caso, vamos elemento por elemento de la lista creando un encadenamiento de únicamente \texttt{ConS}, mas no utilizamos \texttt{NiL}. Este \texttt{AST} lo utilizamos de manera reservada para representar las listas vacías, mientras que \texttt{HeadC} y \texttt{TailC} son similares a \texttt{FstC} y \texttt{SndC} pero sobre listas.\\

De este modo terminamos con el algoritmo de la función \texttt{desugar} para desazucarar nuestros \textbf{ASA} y convertilos a \texttt{AST}:

\begin{lstlisting}[style=haskellstyle, caption={Algoritmo para función \texttt{desugar} en Haskell completo}]
  {-- Desazucaramos los ASA  --}
  desugar :: ASA -> AST
  -- Casos base
  desugar (Var x) = VarC x
  desugar (Num n) = NumC n
  desugar (Boolean b) = BoolC b
  -- Operaciones aritmeticas
  desugar (Add xs) = desugarOps AddC xs
  desugar (Sub xs) = desugarOps SubC xs
  desugar (Mul xs) = desugarOps MulC xs
  desugar (Div xs) = desugarOps DivC xs
  desugar (Add1 n) = AddC (desugar n) (NumC 1)
  desugar (Sub1 n) = SubC (desugar n) (NumC 1)
  desugar (Expt n) = MulC (desugar n) (desugar n)
  desugar (Sqrt n) = SqrtC (desugar n)
  -- Not
  desugar (Not x) = NotC (desugar x)
  -- Comparaciones
  desugar (Equal xs) = desugarComp EqualC xs
  desugar (Less xs) = desugarComp LessC xs
  desugar (Greater xs) = desugarComp GreaterC xs
  desugar (Diff xs) = desugarComp DiffC xs
  desugar (Leq xs) = desugarComp LeqC xs
  desugar (Geq xs) = desugarComp GeqC xs
  -- Pares
  desugar (Pair f s) = PairC (desugar f) (desugar s)
  desugar (Fst p) = FstC (desugar p)
  desugar (Snd p) = SndC (desugar p)
  --Condicionales
  desugar (If0 c t e) = IfC (EqualC (desugar c) (NumC 0)) (desugar t) (desugar e)
  desugar (If c t e) = IfC (desugar c) (desugar t) (desugar e)
  desugar (Cond cs e) = desugarCond cs e
  --Lets
  desugar (Let iv b) = desugarLet iv b
  desugar (LetStar [] body) = desugar body
  desugar (LetStar (iv:ivs) b) = desugar (Let [iv] (LetStar ivs b))
  --Let recursivo
  desugar (LetRec i v b) = desugar (Let [(i, App (Var "Z") [Lambda [i] v])] b)
  --Expresiones lambda
  desugar (Lambda ps b) = desugarLmb ps b
  desugar (App f as) = desugarApp (desugar f) as
  --Listas
  desugar (List l) = desugarList l
  desugar (Head l) = HeadC (desugar l)
  desugar (Tail l) = TailC (desugar l)

  {-- Funciones auxiliares para desugar --}
  --Funcion auxiliar para desazucarar los operadores
  desugarOps :: (AST -> AST -> AST) -> [ASA] -> AST
  desugarOps _ [] = error "[desugarOps Error]: Lista vacia (no deberia suceder)"
  desugarOps _ [x] = desugar x
  desugarOps op (x:xs) = op (desugar x) (desugarOps op xs)

  --Funcion auxiliar para desazucarar los comparadores
  desugarComp :: (AST -> AST -> AST) -> [ASA] -> AST
  desugarComp _ [] = BoolC True
  desugarComp _ [_] = BoolC True
  desugarComp op [i, d] = op (desugar i) (desugar d)
  desugarComp op (i:d:is) = IfC (op (desugar i) (desugar d))
  (desugarComp op (d:is))
  (BoolC False)
  
  --Funcion auxiliar para desazucarar el operador cond
  desugarCond :: [(ASA, ASA)] -> ASA -> AST
  desugarCond [] e = desugar e
  desugarCond ((c, t):cs) e = IfC (desugar c) (desugar t) (desugarCond cs e)

  --Funciones auxiliares para desazucarar let
  desugarLet :: [(String, ASA)] -> ASA -> AST
  desugarLet [] b = desugar b
  desugarLet ((p, v):ps) b = AppC (FunC p (desugarLet ps b)) (desugar v)

  --Funcion auxiliar para desazucarar las funcioneslambda
  desugarLmb :: [String] -> ASA -> AST
  desugarLmb [] b = desugar b
  desugarLmb (p:ps) b = FunC p (desugarLmb ps b)

  --Funcion auxiliar para desazucarar las aplicaciones de funcion
  desugarApp :: AST -> [ASA] -> AST
  desugarApp f [] = f
  desugarApp f (a:as) = desugarApp (AppC f (desugar a)) as

  --Funcion auxiliar para construir listas como cons y nil
  desugarList :: [ASA] -> AST
  desugarList [] = NiL
  desugarList [x] = desugar x
  desugarList (x:xs) = ConS (desugar x) (desugarList xs)
\end{lstlisting}

\bigskip

Concluimos así con la implementación del proceso de desazucarado de nuestros Árboles de Sintaxis Abstracta, obteniendo como resultado el núcleo fundamental de nuestro lenguaje \minilisp. Como vimos, este procedimiento es esencial, pues nos permite simplificar la estructura del lenguaje al eliminar construcciones sintácticas superficiales y reducirlas a formas más primitivas.

Este enfoque no solo disminuye la complejidad de las reglas necesarias para el intérprete, sino que también evita operaciones redundantes, facilitando el análisis, la evaluación y el mantenimiento del lenguaje. Además, garantiza el respeto absoluto a la intención del usuario, manteniendo intacta la semántica del programa original.

En resumen, la eliminación del azúcar sintáctica constituye una fase clave en el diseño de lenguajes funcionales, ya que nos permite trabajar con una base sólida, mínima y expresiva sobre la cual construir el resto de las capacidades del lenguaje, asegurando claridad, consistencia y elegancia en su implementación.

\bigskip

%%%%%%%%%%%%%%%%%%%%%%%%%%%%%%%%%%

% ----- Semántica Operacional -----
%%%%%%%%%%%%%%%%%%%%%%%%%%%%%%%%%%
\chapter{Semántica Operacional}

Una vez hemos establecido nuestro sintaxis del lenguaje en núcleos, lo siguiente a realizar darle el significado a estos programas. Para ello recurrimos a la semántica operacional, un enfoque que describe la dinámica de ejecución de los programas mediante un conjunto de reglas de inferencia.

\begin{quote}
  \textit{Desarrollar la introducción y explicación de la formalización de la semántica y semántica operacional.}
\end{quote}

La semántica operacional...
\begin{quote}
  \textit{Continuar con la definicion de semantica operacional .}
\end{quote}

%se formula como un sistema de transición, donde el comportamiento de un programa se entiende como una sucesión de pasos que transforman configuraciones intermedias hasta llegar a un resultado final. Este enfoque no solo captura el proceso de evaluación de manera formal y precisa, sino que además mantiene una correspondencia directa entre la estructura sintáctica del lenguaje y su dinámica de ejecución. Facilita la implementación de intérpretes, ya que las reglas pueden traducirse de manera casi inmediata en algoritmos. Proporciona un marco para el razonamiento formal sobre las propiedades del lenguaje (corrección, terminación, equivalencias, etc.).

Existen dos estilos principales:

Semantica Natural conocido como paso grande (Big-step)
\begin{quote}
  \textit{Definir brevemente lo que es paso pequeño y quien lo definio.}
\end{quote}

Semantica Estructural conocida como paso pequeño (Small-step)
\begin{quote}
  \textit{Definir brevemente lo que es paso grande y quien lo definio.}
\end{quote}

Para nuestro lenguaje nos enfocaremos en este último.

%%%%%%%%%%%%%%%%%%%%%%%%%%%%%%%%%

\section{Semántica Estructural (Paso pequeño)}
Se le conoce también como semántica de paso pequeño o de transición describe paso a paso la ejecución mostrando los cómputos que genera en cada paso individualmente.

\begin{quote}
  \textit{Aqui ya describimos toda la historia de paso pequeño, qué significa paso pequeño y por que debemos impleemntarlo}
\end{quote}

%Para modelar esta noción, definimos la siguiente relación:
%\begin{center}$e_1 \rightarrow e_2$}end{center}
%que a diferencia de la semántica natural, no relaciona necesariamente expresiones con su valor final, si no que relaciona expresiones con expresiones (pudiendo ser estados finales). Se lee como en un paso e1 se reduce a e2. La definición de reglas en este tipo de semántica, es más complicada pues se tienen que considerar pasos intermedios que en ocasiones pasan desapercibidos al ejecutar o analizar un programa. antes de definir las reglas debemos definir lo que es un sistema de transicion.

\subsection{Sistema de transición}
\begin{quote}
  \textit{Dar la definicion y explicacion de un sistema de transision y como la aplicaremos a nustro proyecto. Tambien mencionamos y describimos los estados finales pero solo eso, no los definimos para el lenguaje}
\end{quote}

%Para modelar formalmente la ejecución, definimos un **sistema de transición**, el cual especifica cómo se produce la evolución de los estados durante la evaluación. **Definición (Sistema de transición):** Un sistema de transición es una tupla (E,$\delta$,I,F) donde: $E$: conjunto (posiblemente infinito) de estados. $\delta \subseteq E \times E$: relación de transición, que indica cómo un estado puede evolucionar a otro. $I\subseteq E$: conjunto de estados iniciales, desde los cuales comienza la ejecución. $F \subseteq E$: conjunto de estados finales, que representan los puntos de término de la ejecución.
%En este marco, el comportamiento de un programa se formaliza como una secuencia de transiciones: $e_0 \;\to\; e_1 \;\to\; e_2 \;\to\; \dots \;\to\; e_f$ Donde $e_0 \in I$, $e_f \in F$ y cada paso $(e_i, e_{i+1}) \in \delta$ está justificado por una regla de inferencia. $\#\#$ Sistema de transición  *small-step* para Iris El sistema de transición de paso pequeño para el lenguaje se define como la tupla $(E, \to, I, F)$, donde: $E = \{\, e \mid e \in ASA \,\}$ es el conjunto de expresiones bien formadas en sintaxis abstracta. $\to \;\subseteq\; E \times E$ es la relación de reducción (small-step) que relaciona una expresión con otra expresión (intermedia o final). Usaremos la notación $e \to e'$ para indicar una reducción de un paso de $e$ a $e'$. $I = E$, es decir, cualquier expresión bien formada puede considerarse un estado inicial. $F$ es el conjunto de estados finales (valores canónicos) dado por: $F \;=\; \{\, Num(n) \mid n \in \mathbb{Z} \,\} \;\cup\; \{\, Boolean(b) \mid b \in \{True,False\} \,\} \;\cup\; \{\, Pair(f,s) \mid f,s \in F \,\} \;\cup\; \{\, Closure(l,\varepsilon) \mid l \text{ es una función y } \varepsilon \text{ es un ambiente léxico} \,\} \;\cup\; \{\, Error \mid \text{la ejecución es interrumpida} \,\}.$ En este sistema, la evaluación se modela como una secuencia finita o infinita de reducciones de un solo paso: $e_0 \;\to\; e_1 \;\to\; e_2 \;\to\; \cdots$ donde $ e_0 \in I$. Una ejecución ha terminado normalmente cuando se alcanza $e_k \in F$. Si en algún paso se llega a la expresión especial $Error$, la ejecución queda interrumpida y se considera una terminación anómala.

\begin{quote}
  \textit{Explicar lo que es la cerradura }
\end{quote}

\subsection{Estados Finales en\minilisp}

Definimos los estados finales para el lenguaje\minilisp{-0.2cm}:

\begin{align*}
  F = &\{\, Num_{V}(n) \mid n \in \Z\,\}\: \cup \:\{\, Boolean_{V}(b) \mid b \in \{True,False\}\,\}\: \cup \:\{\, Pair_{V}(f,s) \mid f,s\in F\,\}\: \cup \\
  &\{\, Cons_{V}(h,t) \mid h,t\in C\,\}\: \cup \:\{\, Nil_{V} \mid \text{ es la lista vacía}\}\: \cup \\
  &\{\, Closure (f,\varepsilon) \mid f \text{ es una función y } \varepsilon\text{ es un ambiente léxico}\}
\end{align*}

Intuitivamente implementamos los estados finales en nuestro lenguaje como un tipo de dato en Haskell:

\begin{lstlisting}[style=haskellstyle, caption={Tipo de dato ASV, representan los estados finales}]
module ASV where

-- ASA Values
data ASV
  = VarV String
  | NumV Int
  | BoolV Bool
  | NiV
  | PairV ASV ASV
  | ConV ASV ASV
  | Closure String ASV [(String, ASV)]
  deriving (Show, Eq)
\end{lstlisting}

El tipo de dato \texttt{ASV} serán con el que modelaremos los estados finales del lenguaje. Sin embargo, en nuestra implementación, es necesario agregar las demás estructuras que trabajan con estos estados finales -como lo son los operadores aritméticos o las operaciones sobre pares o listas- ya que \texttt{ASV} sigue modelando una estructura de tipo árbol la cual sigue el mismo principio que las estructuras \texttt{ASA}:

\begin{center}
  \textit{Una expresión es \texttt{ASV} si y solo si sus hijos también son \texttt{ASV}.}
\end{center}

Por lo que debemos hacer que nuestros Para diferenciar entre los verdaderos estados finales y las demás estructuras de tipo \texttt{ASV} será que nos referimos a los finales como \textit{\textbf{valores canónicos}}.\footnote{Más adelante enfatizaremos en como los diferenciamos entre Valores \texttt{ASV} y expresiones \texttt{ASV} en Haskell.}
Lo que nos queda en la nueva definición del tipo de dato \texttt{ASV}:

\begin{lstlisting}[style=haskellstyle, caption={Tipo de dato ASV completo}]
-- ASA Values
data ASV
  = VarV String
  | NumV Int
  | BoolV Bool
  | NiV
  | AddV ASV ASV
  | SubV ASV ASV
  | MulV ASV ASV
  | DiV ASV ASV
  | SqrtV ASV
  | NotV ASV
  | EqualV ASV ASV
  | LessV ASV ASV
  | GreaterV ASV ASV
  | DiffV ASV ASV
  | LeqV ASV ASV
  | GeqV ASV ASV
  | PairV ASV ASV
  | FstV ASV
  | SndV ASV
  | IfV ASV ASV ASV
  | FunV String ASV
  | AppV ASV ASV
  | ConV ASV ASV
  | HeadV ASV
  | TailV ASV
  | Closure String ASV [(String, ASV)]
  deriving (Show, Eq)
\end{lstlisting}

Ya que hemos definido el tipo de dato \texttt{ASV} veamos como convertimos nuestras estructuras \texttt{AST} a estados finales \texttt{ASV}:

\begin{lstlisting}[style=haskellstyle, caption={Función \texttt{toFinalState} que transforma los núcleos \texttt{AST} a estados finales \texttt{ASV}}]
  {-- Convertimos los AST a estados finales ASV --}
  toFinalState :: AST -> ASV
  toFinalState (VarC x) = VarV x
  toFinalState (NumC n) = NumV n
  toFinalState (BoolC b) = BoolV b
  toFinalState (AddC i d) = AddV (toFinalState i) (toFinalState d)
  toFinalState (SubC i d) = SubV (toFinalState i) (toFinalState d)
  toFinalState (MulC i d) = MulV (toFinalState i) (toFinalState d)
  toFinalState (DivC i d) = DiV (toFinalState i) (toFinalState d)
  toFinalState (SqrtC n) = SqrtV (toFinalState n)
  toFinalState (NotC x) = NotV (toFinalState x)
  toFinalState (EqualC i d) = EqualV (toFinalState i) (toFinalState d)
  toFinalState (LessC i d) = LessV (toFinalState i) (toFinalState d)
  toFinalState (GreaterC i d) = GreaterV (toFinalState i) (toFinalState d)
  toFinalState (DiffC i d) = DiffV (toFinalState i) (toFinalState d)
  toFinalState (LeqC i d) = LeqV (toFinalState i) (toFinalState d)
  toFinalState (GeqC i d) = GeqV (toFinalState i) (toFinalState d)
  toFinalState (PairC f s) = PairV (toFinalState f) (toFinalState s)
  toFinalState (FstC p) = FstV (toFinalState p)
  toFinalState (SndC p) = SndV (toFinalState p)
  toFinalState (ConS f s) = ConV (toFinalState f) (toFinalState s)
  toFinalState (HeadC p) = HeadV (toFinalState p)
  toFinalState (TailC p) = TailV (toFinalState p)
  toFinalState (IfC c t e) = IfV (toFinalState c) (toFinalState t) (toFinalState e)
  toFinalState (FunC p b) = FunV p (toFinalState b)
  toFinalState (AppC f a) = AppV (toFinalState f) (toFinalState a)
  toFinalState NiL = NiV
\end{lstlisting}

La función \texttt{toFinalState} transforma cada estructura del núcleo AST en su equivalente ASV. Aunque a primera vista parezca una simple correspondencia entre constructores, su propósito es fundamental: garantizar que todas las expresiones que se evalúan dentro del intérprete sean expresadas en términos de ASV, permitiendo así que la semántica del lenguaje opere únicamente sobre estructuras homogéneas y compatibles con los valores finales del lenguaje.

\subsection{Reglas de evaluación}

\begin{quote}
  \textit{Dar una introduccion de que son las reglas de evaluacion y por que son necesarias, ademas completar con la explicacion de las reglas de evaluacion. Hablar de los ambientes}
\end{quote}

\begin{itemize}
\item Expresiones atómicas:
  \begin{itemize}
  \item VarV(i):
    \[
    \frac{lookup\ i\ \varepsilon = v}{\langle VarV(i), \varepsilon \rangle \to \langle v, \varepsilon \rangle}
    \]
    \[
    \frac{lookup\ i\ \varepsilon \ \text{no está definido}}{\langle Varv(i), \varepsilon \rangle \to \text{Error en la ejecución de la evaluación}}
    \]

    \begin{quote}
      \textit{Aqui se puede poner la explicacion breve de lo que hace lookup}
    \end{quote}
    
  \item NumV(n):
    \[
    \frac{}{\langle NumV(n), \varepsilon \rangle \to \langle NumV(n), \varepsilon \rangle}
    \]
    
  \item BoolV(b):
    \[
    \frac{}{\langle BoolV(b), \varepsilon \rangle \to \langle BoolV(b), \varepsilon \rangle}
    \]
  \item NiV:
    \[
    \frac{}{\langle NivV, \varepsilon \rangle \to \langle NiV, \varepsilon \rangle}
    \]
  \end{itemize}
  
\item AddV(i,d):
  \[
  \frac{\langle i, \varepsilon \rangle \to \langle i', \varepsilon \rangle}{\langle AddV(i,d), \varepsilon \rangle \to \langle AddV(i',d), \varepsilon \rangle}
  \]
  \[
  \frac{\langle d, \varepsilon \rangle \to \langle d', \varepsilon \rangle}{\langle AddV(NumV(n),d), \varepsilon \rangle \to \langle Add(NumV(n),d'), \varepsilon \rangle}
  \]
  \[
  \frac{}{\langle AddV(NumV(n),NumV(m)), \varepsilon \rangle \to \langle NumV(n +_{\Z} m), \varepsilon \rangle}
  \]
  
\item SubV(i,d):
  \[
  \frac{\langle i, \varepsilon \rangle \to \langle i', \varepsilon \rangle}{\langle SubV(i,d), \varepsilon \rangle \to \langle SubV(i',d), \varepsilon \rangle}
  \]
  \[
  \frac{\langle d, \varepsilon \rangle \to \langle d', \varepsilon \rangle}{\langle SubV(NumV(n),d), \varepsilon \rangle \to \langle SubV(NumV(n),d'), \varepsilon \rangle}
  \]
  \[
  \frac{}{\langle SubV(NumV(n),NumV(m)), \varepsilon \rangle \to \langle NumV(n -_{\Z} m), \varepsilon \rangle}
  \]
 
\item MulV(i,d):
  \[
  \frac{\langle i, \varepsilon \rangle \to \langle i', \varepsilon \rangle}{\langle MulV(i,d), \varepsilon \rangle \to \langle MulV(i',d), \varepsilon \rangle}
  \]
  \[
  \frac{\langle d, \varepsilon \rangle \to \langle d', \varepsilon \rangle}{\langle MulV(NumV(n),d), \varepsilon \rangle \to \langle MulV(NumV(n),d'), \varepsilon \rangle}
  \]
  \[
  \frac{}{\langle MulV(NumV(n),NumV(m)), \varepsilon \rangle \to \langle NumV(n *_{\Z} m), \varepsilon \rangle}
  \]
  
\item DiV(i,d):
  \[
  \frac{\langle i, \varepsilon \rangle \to \langle i', \varepsilon \rangle}{\langle DiV(i,d), \varepsilon \rangle \to \langle DiV(i',d), \varepsilon \rangle}
  \]
  \[
  \frac{\langle d, \varepsilon \rangle \to \langle d', \varepsilon \rangle}{\langle DiV(NumV(n),d), \varepsilon \rangle \to \langle DiV(NumV(n),d'), \varepsilon \rangle}
  \]
  \[
  \frac{}{\langle DiV(NumV(n),NumV(0)), \varepsilon \to \text{Error en la ejecución de la evaluación}}
  \]
  \[
  \frac{}{ \langle DiV(NumV(n),NumV(m)), \varepsilon \rangle \to \langle NumV(n\ /_{\Z}\ m), \varepsilon \rangle} \quad (m \neq 0)
  \]
  
\item SqrtV(n):
  \[
  \frac{\langle n, \varepsilon \rangle \to \langle n', \varepsilon \rangle}{\langle SqrtV(n), \varepsilon \rangle \to \langle SqrtV(n'), \varepsilon \rangle}
  \]
  \[
  \frac{n < 0}{\langle SqrtV(NumV(n)), \varepsilon \to \text{Error en la ejecución de la evaluación}}
  \]
  \[
  \frac{}{\langle SqrtV(NumV(n)), \varepsilon \rangle \to \langle NumV(\sqrt{n}_{\N}), \varepsilon \rangle}  \quad (n \geq 0)
  \]
  
\item NotV(b):
  \[
  \frac{\langle b, \varepsilon \rangle \to \langle n', \varepsilon \rangle}{\langle NotV(b), \varepsilon \rangle \to \langle NotV(b'), \varepsilon \rangle}
  \]
  \[
  \frac{}{\langle NotV(BoolV(b)), \varepsilon \rangle \to \langle BoolV(\neg_{\P}b) \varepsilon \rangle}
  \]
  
\item EqualV(i,d):
  \[
  \frac{\langle i, \varepsilon \rangle \to \langle i', \varepsilon \rangle}{\langle EqualV(i,d), \varepsilon \rangle \to \langle EqualV(i',d), \varepsilon \rangle}
  \]
  \[
  \frac{\langle d, \varepsilon \rangle \to \langle d', \varepsilon \rangle}{\langle EqualV(NumV(n),d), \varepsilon \rangle \to \langle EqualV(NumV(n),d'), \varepsilon \rangle}
  \]
  \[
  \frac{}{\langle EqualV(NumV(n),NumV(m)), \varepsilon \rangle \to \langle BoolV(n =_{\Z} m), \varepsilon \rangle}
  \]

\item LessV(i,d)
  \[
  \frac{\langle i, \varepsilon \rangle \to \langle i', \varepsilon \rangle}{\langle LessV(i,d), \varepsilon \rangle \to \langle LessV(i',d), \varepsilon \rangle}
  \]
  \[
  \frac{\langle d, \varepsilon \rangle \to \langle d', \varepsilon \rangle}{\langle LessV(NumV(n),d), \varepsilon \rangle \to \langle LessV(NumV(n),d'), \varepsilon \rangle}
  \]
  \[
  \frac{}{\langle LessV(NumV(n),NumV(m)), \varepsilon \rangle \to \langle BoolV(n <_{\Z} m), \varepsilon \rangle}
  \]
  
\item GreaterV(i,d)
  \[
  \frac{\langle i, \varepsilon \rangle \to \langle i', \varepsilon \rangle}{\langle GreaterV(i,d), \varepsilon \rangle \to \langle GreaterV(i',d), \varepsilon \rangle}
  \]
  \[
  \frac{\langle d, \varepsilon \rangle \to \langle d', \varepsilon \rangle}{\langle GreaterV(NumV(n),d), \varepsilon \rangle \to \langle GreaterV(NumV(n),d'), \varepsilon \rangle}
  \]
  \[
  \frac{}{\langle GreaterV(NumV(n),NumV(m)), \varepsilon \rangle \to \langle BoolV(n >_{\Z} m), \varepsilon \rangle}
  \]
  
\item DiffV(i,d)
  \[
  \frac{\langle i, \varepsilon \rangle \to \langle i', \varepsilon \rangle}{\langle DiffV(i,d), \varepsilon \rangle \to \langle DiffV(i',d), \varepsilon \rangle}
  \]
  \[
  \frac{\langle d, \varepsilon \rangle \to \langle d', \varepsilon \rangle}{\langle DiffV(NumV(n),d), \varepsilon \rangle \to \langle DiffV(NumV(n),d'), \varepsilon \rangle}
  \]
  \[
  \frac{}{\langle DiffV(NumV(n),NumV(m)), \varepsilon \rangle \to \langle BoolV(n \neq_{\Z} m), \varepsilon \rangle}
  \]
  
\item LeqV(i,d)
  \[
  \frac{\langle i, \varepsilon \rangle \to \langle i', \varepsilon \rangle}{\langle LeqV(i,d), \varepsilon \rangle \to \langle LeqV(i',d), \varepsilon \rangle}
  \]
  \[
  \frac{\langle d, \varepsilon \rangle \to \langle d', \varepsilon \rangle}{\langle LeqV(NumV(n),d), \varepsilon \rangle \to \langle LeqV(NumV(n),d'), \varepsilon \rangle}
  \]
  \[
  \frac{}{\langle LeqV(NumV(n),NumV(m)), \varepsilon \rangle \to \langle BoolV(n \leq_{\Z} m), \varepsilon \rangle}
  \]
  
\item GeqV(i,d)
  \[
  \frac{\langle i, \varepsilon \rangle \to \langle i', \varepsilon \rangle}{\langle GeqV(i,d), \varepsilon \rangle \to \langle GeqV(i',d), \varepsilon \rangle}
  \]
  \[
  \frac{\langle d, \varepsilon \rangle \to \langle d', \varepsilon \rangle}{\langle GeqV(NumV(n),d), \varepsilon \rangle \to \langle GeqV(NumV(n),d'), \varepsilon \rangle}
  \]
  \[
  \frac{}{\langle GeqV(NumV(n),NumV(m)), \varepsilon \rangle \to \langle BoolV(n \geq_{\Z} m), \varepsilon \rangle}
  \]
\end{itemize}

Como se puede ver en las reglas para los comparadores, la regla final cuando ambas expresiones del comparador son el resultado deriva en

\begin{itemize}  
\item PairV(f,s):
  \[
  \frac{\langle f, \varepsilon \rangle \to \langle f', \varepsilon \rangle}{\langle PairV(f,s), \varepsilon \rangle \to \langle PairV(f',s), \varepsilon \rangle}
  \]
  \[
  \frac{\langle s, \varepsilon \rangle \to \langle s', \varepsilon \rangle}{\langle PairV(v_f,s), \varepsilon \rangle \to \langle PairV(v_f,s'), \varepsilon \rangle}
  \]
  \[
  \frac{v_f, v_s \text{ son valores canónicos}}{\langle PairV(v_f,v_s), \varepsilon \rangle \to \langle PairV(v_f,v_s), \varepsilon \rangle}
  \]
  
\item FstV(p):
  \[
  \frac{\langle p, \varepsilon \rangle \to \langle p', \varepsilon \rangle}{\langle FstV(p), \varepsilon \rangle \to \langle FstV(p'), \varepsilon \rangle}
  \]
  \[
  \frac{v_1, v_2 \text{ son valores canónicos}}{\langle FstV(PairV(v_1,v_2)), \varepsilon \rangle \to \langle v_1, \varepsilon \rangle}
  \]
  
\item SndV(p):
  \[
  \frac{\langle p, \varepsilon \rangle \to \langle p', \varepsilon \rangle}{\langle SndV(p), \varepsilon \rangle \to \langle SndV(p'), \varepsilon \rangle}
  \]
  \[
  \frac{v_1, v_2 \text{ son valores canónicos}}{\langle SndV(PairV(v_1,v_2)), \varepsilon \rangle \to \langle v_2, \varepsilon \rangle}
  \]
  
\item IfV(c,t,e), IfV solo evalúa la condicional hasta llegar a un \texttt{BoolV} mas no evalúa el \textbf{then} o \textbf{else} dependiendo del resultado de la condición, solo se encarga de retornar alguno de los dos dependiendo del caso:
  \[
  \frac{\langle c, \varepsilon \rangle \to \langle c', \varepsilon \rangle}{\langle IfV(c,t,e), \varepsilon \rangle \to \langle IfV(c',t,e), \varepsilon \rangle}
  \]
  \[
  \frac{}{\langle IfV(BoolV(\#t),t,e), \varepsilon \rangle \to \langle t, \varepsilon \rangle}
  \]
  \[
  \frac{}{\langle IfV(BoolV(\#f),t,e), \varepsilon \rangle \to \langle e, \varepsilon \rangle}
  \]
  
\item ConV(i,d):
  \[
  \frac{\langle i, \varepsilon \rangle \to \langle i', \varepsilon \rangle}{\langle ConV(i,d), \varepsilon \rangle \to \langle ConV(i',d), \varepsilon \rangle}
  \]
  \[
  \frac{v_i \text{ es valor canónico} \; \langle d, \varepsilon \rangle \to \langle d', \varepsilon \rangle}{\langle ConV(v_i,d), \varepsilon \rangle \to \langle ConV(v_i,d'), \varepsilon \rangle}
  \]
  \[
  \frac{v_i, v_d \text{ son valores canónicos}}{\langle ConV(v_i,v_d), \varepsilon \rangle \to \langle ConV(v_i,v_d), \varepsilon \rangle}
  \]
  
\item HeadV(l):
  \[
  \frac{\langle p, \varepsilon \rangle \to \langle p', \varepsilon \rangle}{\langle HeadV(l), \varepsilon \rangle \to \langle HeadV(l'), \varepsilon \rangle}
  \]
  \[
  \frac{}{\langle HeadV(ConV(v_i,v_d)), \varepsilon \rangle \to \langle v_i, \varepsilon \rangle}
  \]
  
\item TailV(l);
  \[
  \frac{\langle l, \varepsilon \rangle \to \langle l', \varepsilon \rangle}{\langle TailV(l), \varepsilon \rangle \to \langle TailV(l'), \varepsilon \rangle}
  \]
  \[
  \frac{v_d \text{ es valor pero } v_d = ConV(i,d) \;\; v_d \to v_d'}{\langle TailV(ConV(v_i,v_d)), \varepsilon \rangle \to \langle TailV(v_i,v_d'), \varepsilon \rangle}
  \]
  \[
  \frac{v_d \text{ es valor pero} v_d \neq ConV(i,d)}{\langle TailV(ConV(v_i,v_d)), \varepsilon \rangle \to \langle v_d, \varepsilon \rangle}
  \]
  
\item FunV(p,c):
  \[
  \frac{}{\langle FunV(p,c), \varepsilon \rangle \to \langle Closure(FunV(p,c), \varepsilon), \varepsilon \rangle}
  \]
  
\item AppV(f,a):
  \[
  \frac{\langle f, \varepsilon \rangle \to \langle f', \varepsilon \rangle}{\langle AppV(f,a), \varepsilon \rangle \to \langle AppV(f',a), \varepsilon \rangle}
  \]
  \[
  \frac{\langle a, \varepsilon \rangle \to \langle a', \varepsilon \rangle}{\langle AppV(Closure(FunV(p,c), \varepsilon'),a), \varepsilon \rangle \to \langle AppV(Closure(FunV(p,c), \varepsilon'),a'), \varepsilon \rangle}
  \]
  \[
  \frac{\^{a} \text{ es un valor canónico}}{\langle AppV(Closure(FunV(p,c), \varepsilon'), \^{a}), \varepsilon \rangle \to \langle c, \varepsilon'[p \leftarrow \^{a}] \rangle}
  \]
\end{itemize}

\begin{quote}
  \textit{Puede que sea buena idea dar un cierre a esto antes de entrar en Haskell}
\end{quote}

%%%%%%%%%%%%%%%%%%%%%%%%%%%%%%%%%

\section{Intérprete para\minilisp}

Una vez definimos formalmente lo que será la \textbf{\textit{Semántica Operacional Estrcutural}} para nuestro lenguaje podemos programar el interprete del mismo que será el encargado de aplica la evaluación al programa del usuario, más precisamente a las expreciones \texttt{ASV} que ya han depurado el programa original.

\begin{quote}
  \textit{creo podemos extendernos mas aqui}
\end{quote}

%%%%%%%%%%%%%%%%%%%%%%%%%%%%%%%%%

\subsection{Función paso pequeño en\minilisp}

\begin{lstlisting}[style=haskellstyle, caption={}]

\end{lstlisting}

%%%%%%%%%%%%%%%%%%%%%%%%%%%%%%%%%%

% ----- Puntos Estrictos -----
%%%%%%%%%%%%%%%%%%%%%%%%%%%%%%%%%%
\chapter{Puntos Estrictos}

Este es un capítulo corto pero escencial y crucial para la implementación de nuestro intéprete del lenguaje, ya que al meter recursión de la forma tradicional, la manera en la que hemos planteado la evaluación hasta ahora necesita ser remodela.\\

Para evitar los de variables no definidas al utilizar recursion y combinadores, debemos volver a\minilisp perezo, es decir, debemos reeplantear la estrategia de evaluacion de\minilisp de glotón a perezoso.

Al usar evaluación perezosa, existen ciertos puntos dentro de la implementación en los que no puede
postergarse la evaluación.

\begin{quote}
  \textit{Explicar que es el alcance perezoso y lo que son puntos estrictos.}
\end{quote}

Para ello tenemos que definir un forma que nos permitia aplicar los \textit{puntos estrictos} de nuestro lenguaje. Este forma la modelaremos mediante una función \textbf{strict}, que reduce la expresión hasta llegar a un \textbf{valor canónico} del lenguaje.

\[
\textbf{strict}(\langle e, \varepsilon \rangle) =
\begin{cases}
e, & \text{si } e \text{ es valor},\\[4pt]
\langle e', \varepsilon \rangle, & \text{si } \langle e, \varepsilon \rangle \to \langle e', \varepsilon' \rangle.
\end{cases}
\]

Para aplicar este mecanismo, sí o sí necesitamos nuestras reglas semánticas de modo que apliquen los \textit{puntos estrictos} correspondientes usando \textbf{strict}. El resto de reglas se mantienen sin cambios, pero el cambio radica en que: aquellas expresiones donde detectamos la presencia de \textit{puntos estrictos} y recibian cerraduras de expresión ⟨e, $\varepsilon$⟩, es donde llamamos \textbf{strict}.

\section{Reglas semánticas perezosas con strict}

\begin{itemize}
\item AddV(i,d):
\item SubV(i,d):
\item MulV(i,d):
\item DiV(i,d):
\item SqrtV(n):
\item NotV(b):
\item EqualV(i,d):
\item LessV(i,d):
\item GreaterV(i,d):
\item DiffV(i,d):
\item LeqV(i,d):
\item GeqV(i,d):
\item PairV(f,s):  
\item FstV(p):
\item SndV(p):
\item IfV(c,t,e):
\item ConV(i,d):
\item HeadV(l):
\item TailV(l);
\item FunV(p,c):
\item AppV(f,a):
\end{itemize}

\begin{quote}
  \textit{Consultar la pagina 13-14 del ultimo pdf del profesor para ver contexto}
\end{quote}

Una vez definido todo lo anterior, veamos su implementación en Haskell y cómo se modifica el interprete de\minilisp{-0.2cm}.

\section{Evaluación perezosa para \minilisp}

Como se mostró, al cambiar la estrategia de evaluación de nuestro intérprete, debemos hacer muchos cambios en su implementación en Haskell.

Con este nuevo enfoque, nuestro tipo de dato \texttt{ASV} ya es más reducido donde solo están los valores canónicos:

\begin{lstlisting}[style=haskellstyle, caption={Tipo de dato \texttt{ASV} con valores canónicos}]
module ASV where

import AST

{--
Definimos la representacion del ambiente de ejecucion.
Un ambiente es una lista de pares (id, valor).
--}
type Env = [(String, ASV)]

{-- ASA Values --}
data ASV
  = NumV Int
  | BoolV Bool
  | NiV
  | PairV ASV ASV
  | ConV ASV ASV
  | ClosureF String AST Env
  | ExprV AST Env
  deriving (Show, Eq)
\end{lstlisting}

%https://lambdasspace.github.io/LDP/notas/ldp_n06.pdf
%https://lambdasspace.github.io/LDP/notas/ldp_n07.pdf
%https://lambdasspace.github.io/LDP/notas/ldp_n12.pdf
%https://lambdasspace.github.io/LDP/notas/ldp_n13.pdf
%https://lambdasspace.github.io/LDP/notas/ldp_n14.pdf

A diferencia de la implementación anterior, en este nuevo interprete ya no tenemos que usar la función \texttt{toFinalState} para convertir los \textbf{ASA} en estados finales, ya que los únicos estados finales son los valores canónicos ({Ambas definiciones significan lo mismo pero recordemos que en la primer implementación hicimos el abuso de notación para marcar la diferencia entre ambos}, de eso se ecargará el nuevo intérprete, de convertir las expresion \texttt{AST} a estados finales (valores canónicos).\\

Definimos la función de evaluación perezosa con puntos estrictos como sigue:

\begin{lstlisting}[style=haskellstyle, caption={Evaluación perezosa con \textbf{strict} para\minilisp{-0.2cm}}]
module EvalStrict where

import AST
import ASV
import Interprete

{-- Funcion de evaluacion perezosa --}
evalS :: AST -> Env -> ASV
--Valores
evalS (VarC i) env = lookupS i env
evalS (NumC n) _   = (NumV n)
evalS (BoolC b) _  = (BoolV b)
evalS NiL _        = NiV
--Operadores aritmeticos
evalS (AddC i d) env =
  let i' = strict (evalS i env)
      d' = strict (evalS d env)
   in NumV (numN i' + numN d')
evalS (SubC i d) env =
  let i' = strict (evalS i env)
      d' = strict (evalS d env)
   in NumV (numN i' - numN d')
evalS (MulC i d) env =
  let i' = strict (evalS i env)
      d' = strict (evalS d env)
   in NumV (numN i' * numN d')
evalS (DivC i d) env =
  let n = numN (strict (evalS i env))
      m = numN (strict (evalS d env))
   in if m == 0
        then error "No se puede dividir entre 0"
        else NumV (div n m)
evalS (SqrtC n) env =
  let n' = numN (strict (evalS n env))
   in if n' < 0
        then error "No se puede obtener la raiz de un numero negativo"
        else NumV (integerSquareRoot n')
--Not
evalS (NotC e) env =
  let e' = strict (evalS e env)
   in BoolV (not (boolN e'))
--Comparadores
evalS (EqualC i d) env =
  let i' = strict (evalS i env)
      d' = strict (evalS d env)
   in BoolV (numN i' == numN d')
evalS (LessC i d) env =
  let i' = strict (evalS i env)
      d' = strict (evalS d env)
   in BoolV (numN i' < numN d')
evalS (GreaterC i d) env =
  let i' = strict (evalS i env)
      d' = strict (evalS d env)
   in BoolV (numN i' > numN d')
evalS (DiffC i d) env =
  let i' = strict (evalS i env)
      d' = strict (evalS d env)
   in BoolV (numN i' /= numN d')
evalS (LeqC i d) env =
  let i' = strict (evalS i env)
      d' = strict (evalS d env)
   in BoolV (numN i' <= numN d')
evalS (GeqC i d) env =
  let i' = strict (evalS i env)
      d' = strict (evalS d env)
   in BoolV (numN i' >= numN d')
--Pares
evalS (PairC i d) env =
  let i' = strict (evalS i env)
      d' = strict (evalS d env)
   in PairV i' d'
evalS (FstC p) env =
  case strict (evalS p env) of
    PairV f _ -> f
    _         -> error "Fst espera un par"
evalS (SndC p) env =
  case strict (evalS p env) of
    PairV _ s -> s
    _         -> error "Snd espera un par"
--Cons
evalS (ConS i d) env =
  let i' = strict (evalS i env)
      d' = strict (evalS d env)
   in ConV i' d'
evalS (HeadC p) env =
  case strict (evalS p env) of
    ConV h _ -> h
    _        -> error "Head espera una lista"
evalS (TailC p) env =
  case strict (evalS p env) of
    ConV _ t ->
      let t' = strict t
       in if not (isConV t')
          then t'
          else tailDeep t'
    _        -> error "Tail espera una lista"
--If
evalS (IfC c t e) env =
  let cond = boolN (strict (evalS c env))
   in if cond then evalS t env else evalS e env
--Funciones
evalS (FunC p c) env = ClosureF p c env
--Aplicacion de funciones
evalS (AppC f a) env =
  let f' = evalS f env
      funV = strict f'
   in evalS (closureC funV) (((closureP funV), ExprV a env) : (closureE funV))

{-- Funcion strict para forzar la evaluacion de los puntos estrictos --}
strict :: ASV -> ASV
strict (NumV n) = NumV n
strict (BoolV b) = BoolV b
strict (PairV f s) = PairV (strict f) (strict s)
strict (ConV i d) = ConV (strict i) (strict d)
strict (ExprV a e) = strict (evalS a e)
strict (NiV) = NiV
strict (ClosureF p c e) = ClosureF p c e

{-- Funcion auxiliar para devolver el numero de NumV--}
numN :: ASV -> Int
numN (NumV n) = n

{-- Funcion auxiliar para devolver el booleano de BoolV --}
boolN :: ASV -> Bool
boolN (BoolV b) = b
boolN _ = False

{-- Funcion auxiliar para devoler el parametro de la cerradura --}
closureP :: ASV -> String
closureP (ClosureF p _ _) = p

{-- Funcion auxiliar para devoler el cuerpo de la cerradura --}
closureC :: ASV -> AST
closureC (ClosureF _ c _) = c

{-- Funcion auxiliar para devoler el ambiente de la cerradura --}
closureE :: ASV -> Env
closureE (ClosureF _ _ e) = e

{-- Funcion auxiliar para encontrar el ultimo elemento canonico de los ConV anidados--}
tailDeep :: ASV -> ASV
tailDeep (ConV _ rest) =
  let rest' = strict rest
  in if not (isConV rest')
     then rest'
     else tailDeep rest'
tailDeep v = v

\end{lstlisting}

La mayor diferencia a \texttt{eval} es que la función \texttt{evalS} ya no utiliza la evaluación de paso pequeño, sino que implementa una evaluación directa (\textit{\textbf{big-step}}). Ahora utilizamos funciones auxiliares (\texttt{strict}, \texttt{numN}, \texttt{boolN}, etc.) para obtener valores “$forzados$”.

\noindent
Además de este cambio, lo más notorio es \texttt{TailC} y la aplicación de funciones. La primer expresión ahora utiliza una función auxiliar \texttt{tailDeep} para seguir buscando el último elemento en la anidación de \texttt{ConS}, ya que usar \texttt{evalS} nos da una inconsistencia de tipos, pues al evaluar por primera vez el par, este se transforma en un \texttt{ASV} siendo que \texttt{evalS} recibe \texttt{AST}.

\noindent
Para el caso de \texttt{AppC}:

\begin{enumerate}
\item Evalúa $f$ en el ambiente actual, esto debería devolver una cerradura.
\item Fuerza \textit{f'} con \texttt{strict}, asegurándose de que sea una cerradura concreta.
\item Extrae información de la cerradura:
  \begin{itemize}
  \item \texttt{closureC funV}: el cuerpo de la función.
  \item \texttt{closureP funV}: el nombre del parámetro formal.
  \item \texttt{closureE funV}: el ambiente donde la función fue creada.
\end{itemize}

\item Crea un nuevo ambiente donde:
  \begin{itemize}
  \item El parámetro formal (\texttt{closureP funV}) se asocia a la expresión real (\texttt{ExprV a env});
  \item Se añade al ambiente original de la cerradura (\texttt{closureE funV}).
  \end{itemize}
  
\item Evalúa el cuerpo de la función (\texttt{closureC funV}) en ese nuevo ambiente extendido.
\end{enumerate}

De este modo tenemos una evaluación adecuada para introducir recursión con \texttt{LetRec} en nuestro lenguaje\minilisp{-0.2cm}, pues como mencionamos, necesitamos meter al ambiente inicial la evaluación de nuestro combinador \textbf{Z}, el cuál se mostró que presentaba errores con la evaluación glotona.

\begin{quote}
  \textit{Aqui podemos poner un cierre del interprete, recursion y evaluacion glotona}
\end{quote}

%%%%%%%%%%%%%%%%%%%%%%%%%%%%%%%%%%

% ----- Resultados -----
%%%%%%%%%%%%%%%%%%%%%%%%%%%%%%%%%%
\chapter{Resultados}
Bien, una vez hemos visto toda la teoría de nuestro\minilisp y además de que hemos mostrado el código mismo.
En este capítulo presentamos los resultados obtenidos tras la implementación completa del lenguaje\minilisp.
Se mostrará el funcionamiento de cada una de las etapas principales del lenguaje (\textbf{\textit{lexer}},
\textbf{\textit{parser}}, \textbf{\textit{desugar}}, \textbf{\textit{interprete}}), acompañadas de ejemplos que
ilustran tanto su entrada como su salida.

\noindent
Además, se presenta el \textbf{\textit{menú interactivo}} del proyecto, que permite al usuario ejecutar comandos
y evaluar expresiones sin hacer el proceso paso a paso. Este componente sirve como punto de unión entre todas las
fases del lenguaje, permitiendo observar la interacción completa desde la lectura del código fuente hasta la
obtención del resultado final.\\

\noindent
En conjunto, buscamos que este capítulo tenga como propósito mostrar de forma integrada y funcional el resultado del trabajo desarrollado a lo largo de este reporte. Más que solo validar la implementación, buscamos evidenciar la coherencia entre el diseño teórico del lenguaje y su comportamiento práctico, demostrando que los principios formales de la sintaxis y la semántica pueden efectivamente traducirse en un sistema ejecutable, expresivo y consistente.

\section{\minilisp}

Primero mostramos los resultados individuales de cada fase del código implementado que le da vida a nuestro lenguaje\minilisp{-0.2cm}.

\noindent
Utilizamos el intérprete interactivo de Haskell \texttt{GHCi} para compilar, cargar módulos y ejecutar las pruebas de de nuestra implementación. En el archivo \texttt{README.md} mostramos más a detalle como inicializar y compilar nuestro proyecto.

%%%%%%%%%%%%%%%%%%
\subsection{Lexer}
Como se vió a inicios del reporte, el \textit{analizador léxico} es el primer paso en la ejecución del lenguaje.
Recibe el programa del usuario como cadena de caracteres y devuelve una lista de \textit{Tokens} ya definido.
Generamos el analizador léxico con \texttt{Alex}:

\begin{minted}[fontsize=\small, bgcolor=black!5, frame=single]{text}
  $ alex Lexer.x
\end{minted}

Iniciamos el intérprete interactico de Haskell y para no tener que escribir siempre el nombre completo del módulo \texttt{Lexer}, cargalo:

\begin{minted}[fontsize=\small, bgcolor=black!5, frame=single]{text}
  $ ghci
  GHCi, version 9.4.5: https://www.haskell.org/ghc/  :? for help
  ghci> :l Lexer.hs
\end{minted}

Ya con esto podemos probar nuestra función \texttt{lexer}, que no es muy impresionante, como dijimos, va verificando la entrada detectando los \textit{Tokens}:

\begin{itemize}
\item \textbf{Variables:}
  
  \begin{minted}[fontsize=\small, bgcolor=black!5, frame=single]{text}
    ghci> lexer "var12 #f 512 #t sum 90ba"
    [TokenVar "var12",TokenBool False,TokenNum 512,
      TokenBool True,TokenVar "sum",TokenNum 90,TokenVar "ba"]
  \end{minted}

  Podemos ver que el \texttt{lexer} separa adecuadamente entre variables, números y booleanos asignándolos a su respectivo \texttt{Token}. Incluso \textit{var12} se guarda completo como \texttt{TokenVar "var12"} justo como lo definimos, donde las variables comienzan siempre con un caracter seguido de una combinación de caracteres o números. Se puede apreciar también con el vaso de \textit{90ba} donde lo separa como dos \textit{Token} distintos, pues lo números no puden tener caracteres ni las variables pueden comenzar con números.

\item \textbf{Operadores:}

  Veamos la lista de operadores generada por una cadena de operaciones (para acortar la extensión de esta sección):

  \begin{minted}[fontsize=\small, bgcolor=black!5, frame=single]{text}
    ghci> lexer "(expt (+ (- (* 10 3) (/ (add1 5) (sub1 4))) (sqrt 81)))"
    [TokenPA,TokenExpt,
      TokenPA,TokenAdd,
       TokenPA,TokenSub,
        TokenPA,TokenMul,TokenNum 10,TokenNum 3,TokenPC,
         TokenPA,TokenDiv,
          TokenPA,TokenAdd1,TokenNum 5,TokenPC,
          TokenPA,TokenSub1,TokenNum 4,TokenPC,
          TokenPC,TokenPC,
       TokenPA,TokenSqrt,TokenNum 81,TokenPC,TokenPC,TokenPC]
  \end{minted}

  Como se puede apreciar, los símbolos y palabras reservadas para operadores son detectadas correctamente por el
  \texttt{lexer}. No verificamos aún que los argumentos sean válidos en cantidad y tipo, pero si verificamos que los símbolos sean solo los definidos:

  \begin{minted}[fontsize=\small, bgcolor=black!5, frame=single]{text}
    ghci> lexer "(+ 3)"
    [TokenPA,TokenAdd,TokenNum 3,TokenPC]
    ghci> lexer "(sqrt hola)"
    [TokenPA,TokenSqrt,TokenVar "hola",TokenPC]
    ghci> lexer "(% 3 5)"
    [TokenPA,*** Exception: Lexical error:
      caracter no reconocido = "%" | codepoints = [37]
      CallStack (from HasCallStack):
      error, called at Lexer.hs:10816:24 in main:Lexer
  \end{minted}

\item \textbf{Comparadores:}
  
  De manera análoga, tenemos los comparadores:

  \begin{minted}[fontsize=\small, bgcolor=black!5, frame=single]{text}
    ghci> lexer "(!= 0 9 9) (= 6 6 1)
                 (> 1 1) (< 4 3) (>= 7 3 5) (<= 22 22 2) (not #t)"
    [TokenPA,TokenNeq,TokenNum 0,TokenNum 9,TokenNum 9,TokenPC,
      TokenPA,TokenEq,TokenNum 6,TokenNum 6,TokenNum 1,TokenPC,
      TokenPA,TokenGt,TokenNum 1,TokenNum 1,TokenPC,
      TokenPA,TokenLt,TokenNum 4,TokenNum 3,TokenPC,
      TokenPA,TokenGeq,TokenNum 7,TokenNum 3,TokenNum 5,TokenPC,
      TokenPA,TokenLeq,TokenNum 22,TokenNum 22,TokenNum 2,TokenPC,
      TokenPA,TokenNot,TokenBool True,TokenPC]
  \end{minted}
  
  Como se mencionó, en este punto no es relevante para el \texttt{lexer} los argumentos de cada comparador.
  
\item \textbf{Condicionales:}
  
  \begin{minted}[fontsize=\small, bgcolor=black!5, frame=single]{text}
    ghci> lexer "(cond [(= (sqrt 1000) (expt 10)) -1]
                  [(!= (sub1 9) (add1 8)) 1] [else 0])"
    [TokenPA,TokenCond,
      TokenLI,TokenPA,TokenEq,TokenPA,TokenSqrt,TokenNum 1000,TokenPC,
      TokenPA,TokenExpt,TokenNum 10,TokenPC,TokenPC,TokenNum (-1),TokenLD,
      TokenLI,TokenPA,TokenNeq,TokenPA,TokenSub1,TokenNum 9,TokenPC,
      TokenPA,TokenAdd1,TokenNum 8,TokenPC,TokenPC,TokenNum 1,TokenLD,
      TokenLI,TokenElse,TokenNum 0,TokenLD,TokenPC]
  \end{minted}
  
\item \textbf{Pares y Listas:}

  \begin{minted}[fontsize=\small, bgcolor=black!5, frame=single]{text}
    ghci> lexer "[(8,10),[],(#t,#f)]"
    [TokenLI,
      TokenPA,TokenNum 8,TokenComma,TokenNum 10,TokenPC,TokenComma,
      TokenLI,TokenLD,TokenComma,
      TokenPA,TokenBool True,TokenComma,TokenBool False,TokenPC,
     TokenLD]
  \end{minted}
  
\item \textbf{Lets y Expresiones Lambda:}
  
  \begin{minted}[fontsize=\small, bgcolor=black!5, frame=single]{text}
    ghci> lexer "(let (x 10) (expt x))"
    [TokenPA,TokenLet,TokenPA,TokenVar "x",TokenNum 10,TokenPC,
      TokenPA,TokenExpt,TokenVar "x",TokenPC,TokenPC]
  \end{minted}
  
  \begin{minted}[fontsize=\small, bgcolor=black!5, frame=single]{text}
    ghci> lexer "(let* ((x 2)) (+ x 3))"
    [TokenPA,TokenLetStar,TokenPA,TokenPA,TokenVar "x",TokenNum 2,TokenPC,TokenPC,
      TokenPA,TokenAdd,TokenVar "x",TokenNum 3,TokenPC,TokenPC]
  \end{minted}

  \begin{minted}[fontsize=\small, bgcolor=black!5, frame=single]{text}
    ghci> lexer "((lambda (x) (+ x 1)) 5)"
    [TokenPA,TokenPA,TokenLambda,TokenPA,TokenVar "x",TokenPC,
      TokenPA,TokenAdd,TokenVar "x",TokenNum 1,TokenPC,TokenPC,TokenNum 5,TokenPC]
  \end{minted}
  
\end{itemize}

Así, aunque los resultados del \texttt{lexer} no son tan emocionantes como lo pudieran ser para el \texttt{desugar} o \texttt{eval}, hemos mostrado con estos ejemplos que hasta el momento, el análisis sintáctico para el lenguaje funciona.

\noindent
A partir de este momento no mostraremos los resultados de aplicar las fases a únicamente variables ya que es redundante su procedimiento pues podemos ver su progreso en las fases del lenguaje a través de las demás expresiones.

%%%%%%%%%%%%%%%%%%
\subsection{Parser}

En la fase del \texttt{Parser} la situación se vuelve más interesante pues es donde aplicamos la gramática del lenguaje y decidimos las estructuras del programa que son válidas. Veamos ejemplos para algunas expresiones donde son rechazados por el \texttt{Parser}:

\begin{minted}[fontsize=\small, bgcolor=black!5, frame=single]{text}
  ghci> tokens = lexer "(+ 3)"
  ghci> parse tokens 
  *** Exception: Error al Parsear los Tokens
  CallStack (from HasCallStack):
  error, called at ./Grammar.hs:1265:16 in main:Grammar
\end{minted}

Falla porque la suma es un operador variádico que requiere de al menos dos elementos, por eso hay erroe en el \texttt{Parser}. De manera similar, fallan los operadores de resta, multiplicación y división con un solo argumentos, pues, requieren también de dos argumentos como mínimo.

\noindent
Los ejemplos válidos serían:

\begin{minted}[fontsize=\small, bgcolor=black!5, frame=single]{text}
  ghci> tokens = lexer "(+ 52 34 42)"
  ghci> parse tokens 
  Add [Num 52,Num 34,Num 42]
  ghci> tokens = lexer "(- 22 -11 7)"
  ghci> parse tokens 
  Sub [Num 22,Num (-11),Num 7]
  ghci> tokens = lexer "(* 2 200)"
  ghci> parse tokens 
  Mul [Num 2,Num 200]
  ghci> tokens = lexer "(/ 21 0)"
  ghci> parse tokens 
  Div [Num 21,Num 0]
\end{minted}

Nótese que en la división no marcamos error al dividir por cero, pues recrodemos que eso es trabajo de la semántica, estamos en el \textit{análisis sintáctico}.\\

Otro caso son los operadores unarios:

\begin{minted}[fontsize=\small, bgcolor=black!5, frame=single]{text}
  ghci> tokens = lexer "(sqrt 33 81)"
  ghci> parse tokens 
  *** Exception: Error al Parsear los Tokens
  CallStack (from HasCallStack):
  error, called at ./Grammar.hs:1265:16 in main:Grammar
  ghci> tokens = lexer "(expt 21 1 3)"
  ghci> parse tokens 
  *** Exception: Error al Parsear los Tokens
  CallStack (from HasCallStack):
  error, called at ./Grammar.hs:1265:16 in main:Grammar
\end{minted}

\texttt{Sqrt} y \texttt{Expt} fallan porque son operadores unarios, la gramática rechaza tener más de uno:

\begin{minted}[fontsize=\small, bgcolor=black!5, frame=single]{text}
  ghci> tokens = lexer "(sqrt 81)"
  ghci> parse tokens 
  Sqrt (Num 81)
  ghci> tokens = lexer "(expt 16)"
  ghci> parse tokens 
  Expt (Num 16)
\end{minted}

Las demás expresiones funcionan igual, dan error en el \texttt{Parser} con una gramática inválida, por ello veamos como queda el resultado de parsear programas válidos:

\begin {itemize}
\item \textbf{Comparadores:}

  \begin{minted}[fontsize=\small, bgcolor=black!5, frame=single]{text}
    ghci> tokens = lexer "(= (+ 2 3) (* 5 1))"
    ghci> parse tokens
    Equal [Add [Num 2,Num 3],Mul [Num 5,Num 1]]
  \end{minted}

  \begin{minted}[fontsize=\small, bgcolor=black!5, frame=single]{text}
    ghci> tokens = lexer "(< (+ 1 2) (* 2 3))"
    ghci> parse tokens
    Less [Add [Num 1,Num 2],Mul [Num 2,Num 3]]
  \end{minted}

  \begin{minted}[fontsize=\small, bgcolor=black!5, frame=single]{text}
    ghci> tokens = lexer "(> (* 3 3) (+ 4 2))"
    ghci> parse tokens
    Greater [Mul [Num 3,Num 3],Add [Num 4,Num 2]]
  \end{minted}

  \begin{minted}[fontsize=\small, bgcolor=black!5, frame=single]{text}
    ghci> tokens = lexer "(!= (* 2 3) (+ 3 3))"
    ghci> parse tokens
    Diff [Mul [Num 2,Num 3],Add [Num 3,Num 3]]
  \end{minted}

  \begin{minted}[fontsize=\small, bgcolor=black!5, frame=single]{text}
    ghci> tokens = lexer "(<= (+ 2 3) (* 2 3))"
    ghci> parse tokens
    Leq [Add [Num 2,Num 3],Mul [Num 2,Num 3]]
  \end{minted}

  \begin{minted}[fontsize=\small, bgcolor=black!5, frame=single]{text}
    ghci> tokens = lexer "(>= (* 3 3) (+ 4 2))"
    ghci> parse tokens
    Geq [Mul [Num 3,Num 3],Add [Num 4,Num 2]]
  \end{minted}
  
  Notemos que las estructuras se muestran como se deben, con sus argumento guardados como listas y sus etiquetas respectivas a sus \textit{Tokens}.
  
\item \textbf{Condicionales:}
  
  \begin{minted}[fontsize=\small, bgcolor=black!5, frame=single]{text}
    ghci> tokens = lexer "(if (> 3 2) (+ 1 2) (* 2 2))"
    ghci> parse tokens
    If (Greater [Num 3,Num 2]) (Add [Num 1,Num 2]) (Mul [Num 2,Num 2])
  \end{minted}

  \begin{minted}[fontsize=\small, bgcolor=black!5, frame=single]{text}
    ghci> tokens = lexer "(if0 (- 3 3) (+ 1 2) (* 3 3))"
    ghci> parse tokens
    If0 (Sub [Num 3,Num 3]) (Add [Num 1,Num 2]) (Mul [Num 3,Num 3])
  \end{minted}

  \begin{minted}[fontsize=\small, bgcolor=black!5, frame=single]{text}
    ghci> tokens = lexer "(cond [(< x 0) (- 0 x)] [(= x 0) 0] [else (+ x 1)])"
    ghci> parse tokens
    Cond [(Less [Var "x",Num 0],Sub [Num 0,Var "x"]),
      (Equal [Var "x",Num 0],Num 0)] (Add [Var "x",Num 1])
  \end{minted}
  
\item \textbf{Pares y Listas:}

  \begin{minted}[fontsize=\small, bgcolor=black!5, frame=single]{text}
    ghci> tokens = lexer "((+ 1 2), (* 3 4))"
    ghci> parse tokens
    Pair (Add [Num 1,Num 2]) (Mul [Num 3,Num 4])
  \end{minted}

  \begin{minted}[fontsize=\small, bgcolor=black!5, frame=single]{text}
    ghci> tokens = lexer "(fst ((+ 1 2), (sqrt 9)))"
    ghci> parse tokens
    Fst (Pair (Add [Num 1,Num 2]) (Sqrt (Num 9)))
  \end{minted}

  \begin{minted}[fontsize=\small, bgcolor=black!5, frame=single]{text}
    ghci> tokens = lexer "(snd ((sqrt 16), (+ 3 5)))"
    ghci> parse tokens
    Snd (Pair (Sqrt (Num 16)) (Add [Num 3,Num 5]))
  \end{minted}

  \begin{minted}[fontsize=\small, bgcolor=black!5, frame=single]{text}
    ghci> tokens = lexer "[[1, 2, (3, 4)], #t, (+ 1 2)]"
    ghci> parse tokens 
    List [List [Num 1,Num 2,Pair (Num 3) (Num 4)],
      Boolean True,Add [Num 1,Num 2]]
  \end{minted}
  
  \begin{minted}[fontsize=\small, bgcolor=black!5, frame=single]{text}
    ghci> tokens = lexer "(head [[1, 2], (+ 3 4), #f])"
    ghci> parse tokens
    Head (List [List [Num 1,Num 2],Add [Num 3,Num 4],Boolean False])
  \end{minted}
  
  \begin{minted}[fontsize=\small, bgcolor=black!5, frame=single]{text}
    ghci> tokens = lexer "(tail [[(+ 1 2)], (* 3 4), #t])"
    ghci> parse tokens
    Tail (List [List [Add [Num 1,Num 2]],Mul [Num 3,Num 4],Boolean True])
  \end{minted}
  
\item \textbf{Lets y Expresiones Lambda:}
  
  \begin{minted}[fontsize=\small, bgcolor=black!5, frame=single]{text}
    ghci> tokens = lexer "(let ((x 2) (y (* x 3))) (+ x y))"
    ghci> parse tokens
    Let [("x",Num 2),("y",Mul [Var "x",Num 3])] (Add [Var "x",Var "y"])
  \end{minted}

  \begin{minted}[fontsize=\small, bgcolor=black!5, frame=single]{text}
    ghci> tokens = lexer "(let* ((x 2) (y (+ x 3)) (z (* y 2))) (+ x y z))"
    ghci> parse tokens
    LetStar [("x",Num 2),("y",Add [Var "x",Num 3]),
      ("z",Mul [Var "y",Num 2])] (Add [Var "x",Var "y",Var "z"])
  \end{minted}

  \begin{minted}[fontsize=\small, bgcolor=black!5, frame=single]{text}
    ghci> tokens =
           lexer "(letrec (fact
                   (lambda (n) (if0 n 1 (* n (fact (sub1 n)))))) (fact 5))"
    ghci> parse tokens
    LetRec "fact" (Lambda ["n"] (If0 (Var "n") (Num 1)
    (Mul [Var "n",App (Var "fact") [Sub1 (Var "n")]])))
    (App (Var "fact") [Num 5])
  \end{minted}

  \begin{minted}[fontsize=\small, bgcolor=black!5, frame=single]{text}
    ghci> tokens = lexer "(lambda (x y) (if (> x y) (- x y) (+ x y)))"
    ghci> parse tokens
    Lambda ["x","y"] (If (Greater [Var "x",Var "y"])
    (Sub [Var "x",Var "y"]) (Add [Var "x",Var "y"]))
  \end{minted}

  \begin{minted}[fontsize=\small, bgcolor=black!5, frame=single]{text}
    ghci> tokens = lexer "((lambda (f x) (f x)) (lambda (y) (* y y)) 4)"
    ghci> parse tokens
    App (Lambda ["f","x"] (App (Var "f") [Var "x"]))
    [Lambda ["y"] (Mul [Var "y",Var "y"]),Num 4]
  \end{minted}
  
\end{itemize}

Podemos observar que las salidas generadas por el parser corresponden correctamente a la estructura del \textbf{\textit{Árbol de Sintaxis Abstracta}} definido para \minilisp. Cada expresión se traduce en una construcción interna con sus operadores y argumentos organizados en listas.

\noindent
En los ejemplos de comparadores y condicionales, se refleja cómo las expresiones se agrupan jerárquicamente, respetando el orden y los paréntesis del código fuente.
Las secciones de pares y listas muestran la correcta interpretación de estructuras anidadas y de funciones de acceso como \texttt{fst}, \texttt{snd}, \texttt{head} y \texttt{tail}.
Por último, las construcciones de \texttt{let}, \texttt{let*}, \texttt{letrec} y \texttt{lambda} evidencian el manejo de entornos locales y funciones como valores, preparando el terreno para su posterior desazucarización y evaluación semántica.\\


En conjunto, estos resultados confirman que la gramática y el parser generan correctamente los ASA esperados para cada tipo de expresión del lenguaje MiniLisp. Nótese que además que estas estructuras son \textbf{ASA} con azúcar, pues se puede apreciar por ejemplo, el uso listas en Haskell para ciertas estructuras. Aún nos falta la fase de desazucarización.


%%%%%%%%%%%%%%%%%%
\subsection{Desugar}

Veremos los resultados interesantes del proceso de desazucarización:

\begin{minted}[fontsize=\small, bgcolor=black!5, frame=single]{text}
  ghci> tokens = lexer "(add1 (* 2 3))"
  ghci> asa = parse tokens
  ghci> desugar asa
  AddC (MulC (NumC 2) (NumC 3)) (NumC 1)
\end{minted}

\begin{minted}[fontsize=\small, bgcolor=black!5, frame=single]{text}
  ghci> tokens = lexer "(sub1 (+ 4 (* 2 3)))"
  ghci> asa = parse tokens
  ghci> desugar asa
  SubC (AddC (NumC 4) (MulC (NumC 2) (NumC 3))) (NumC 1)
\end{minted}

\begin{minted}[fontsize=\small, bgcolor=black!5, frame=single]{text}
  ghci> tokens = lexer "(expt (add1 (* 2 2)))"
  ghci> asa = parse tokens
  ghci> desugar asa
  MulC (AddC (MulC (NumC 2) (NumC 2)) (NumC 1))
       (AddC (MulC (NumC 2) (NumC 2)) (NumC 1))
\end{minted}

Como se puede ver, \texttt{add1}, \texttt{sub1} y \texttt{expt}, se convierten en suma, resta y multiplicación respectivamente.

\begin{minted}[fontsize=\small, bgcolor=black!5, frame=single]{text}
  ghci> tokens = lexer "(>= (* 3 3) (+ 4 2))"
  ghci> asa = parse tokens
  ghci> desugar asa
  GeqC (MulC (NumC 3) (NumC 3)) (AddC (NumC 4) (NumC 2))
\end{minted}

\begin{minted}[fontsize=\small, bgcolor=black!5, frame=single]{text}
  ghci> tokens = lexer "(= 4 0 (+ 9 3))"
  ghci> asa = parse tokens 
  ghci> desugar asa
  IfC (EqualC (NumC 4) (NumC 0)) (EqualC (NumC 0) (AddC (NumC 9) (NumC 3)))
      (BoolC False)
\end{minted}

\begin{minted}[fontsize=\small, bgcolor=black!5, frame=single]{text}
  ghci> tokens = lexer "(!= 1 1 (- 7 7))"
  ghci> asa = parse tokens 
  ghci> desugar asa
  IfC (DiffC (NumC 1) (NumC 1)) (DiffC (NumC 1) (SubC (NumC 7) (NumC 7)))
      (BoolC False)
\end{minted}

De igual manera, los comparadores ya no son una lista de comparaciones, sino un encadenamiento de condicionales. De manera similar con las condiciones \texttt{If0} y \texttt{Cond} que pasan a \texttt{IfC}.

\begin{minted}[fontsize=\small, bgcolor=black!5, frame=single]{text}
  ghci> tokens = lexer "(if0 (- 3 3) (+ 1 2) (* 3 3))"
  ghci> asa = parse tokens
  ghci> desugar asa
  IfC (EqualC (SubC (NumC 3) (NumC 3)) (NumC 0)) (AddC (NumC 1) (NumC 2))
      (MulC (NumC 3) (NumC 3))
\end{minted}

\begin{minted}[fontsize=\small, bgcolor=black!5, frame=single]{text}
  ghci> tokens = lexer "(cond [(< x 0) (- 0 x)] [(= x 0) 0] [else (+ x 1)])"
  ghci> asa = parse tokens
  ghci> desugar asa
  IfC (LessC (VarC "x") (NumC 0)) (SubC (NumC 0) (VarC "x"))
      (IfC (EqualC (VarC "x") (NumC 0)) (NumC 0) (AddC (VarC "x") (NumC 1)))
\end{minted}

En el caso de las listas notemos que se han convertido en una encadenación de $cons$.

\begin{minted}[fontsize=\small, bgcolor=black!5, frame=single]{text}
  ghci> tokens = lexer "[1, 2, 3]"
  ghci> asa = parse tokens
  ghci> desugar asa
  ConS (NumC 1) (ConS (NumC 2) (NumC 3))
  ghci> tokens = lexer "[[1, 2, (3, 4)], #t, (+ 1 2)]"
  ghci> asa = parse tokens
  ghci> desugar asa
  ConS (ConS (NumC 1) (ConS (NumC 2) (PairC (NumC 3) (NumC 4))))
       (ConS (BoolC True) (AddC (NumC 1) (NumC 2)))
\end{minted}

Para los lets, como se puede apreciar a continuación, se han convertido en aplicaciones de funciones.

\begin{minted}[fontsize=\small, bgcolor=black!5, frame=single]{text}
  ghci> tokens = lexer "(let ((x 5)) (+ x 1))"
  ghci> asa = parse tokens
  ghci> desugar asa
  AppC (FunC "x" (AddC (VarC "x") (NumC 1))) (NumC 5)
  ghci> tokens = lexer "(let ((x 2) (y (* x 3))) (+ x y))"
  ghci> asa = parse tokens
  ghci> desugar asa
  AppC (FunC "x" (AppC (FunC "y" (AddC (VarC "x") (VarC "y")))
       (MulC (VarC "x") (NumC 3)))) (NumC 2)
\end{minted}

\begin{minted}[fontsize=\small, bgcolor=black!5, frame=single]{text}
  ghci> tokens = lexer "(let* ((x 2) (y (+ x 3)) (z (* y 2))) (+ x y z))"
  ghci> asa = parse tokens
  ghci> desugar asa
  AppC (FunC "x" (AppC (FunC "y" (AppC (FunC "z" (AddC (VarC "x")
         (AddC (VarC "y") (VarC "z")))) (MulC (VarC "y") (NumC 2))))
       (AddC (VarC "x") (NumC 3)))) (NumC 2)
\end{minted}

\begin{minted}[fontsize=\small, bgcolor=black!5, frame=single]{text}
  ghci> tokens = lexer "(letrec (fact (lambda (n)
                        (if0 n 1 (* n (fact (sub1 n)))))) (fact 5))"
  ghci> asa = parse tokens
  ghci> desugar asa
  AppC (FunC "fact" (AppC (VarC "fact") (NumC 5))) (AppC (VarC "Z")
         (FunC "fact" (FunC "n" (IfC (EqualC (VarC "n") (NumC 0))
       (NumC 1) (MulC (VarC "n") (AppC (VarC "fact") (SubC (VarC "n") (NumC 1)))
       )))))
\end{minted}

%%%%%%%%%%%%%%%%%%%%%%%%%%%%%%%%%

\section{Menú interactivo}

Para mostrar los resultados del intérprete, utilizaremos un menú interactivo que tendrá la función de recibir la entrada del usuario, para procesarla a través de todas las fases de \texttt{Lexer}, \texttt{Parser} y \texttt{Desugar} para eventualmente realizar el proceso de evaluación semántica.\\

El menú interactivo queda definido en el archivo \texttt{MiniLisp.hs} como sigue:

\begin{lstlisting}[style=haskellstyle, caption={Menú interactivo de\minilisp}]
module MiniLisp where
import Token
import ASA
import AST
import ASV
import Lexer
import Grammar
import Desugar
import Interprete
import EvalStrict
import Saca
import Control.Exception (catch, SomeException)
-- Combinador Z
combZ :: String
combZ = "(lambda (f) ((lambda (x) (f (lambda (v) ((x x) v)))) (lambda (x) (f (lambda (v) ((x x) v))))))"
-- Evaluamos el combinador Z
z :: ASV
z = evalS (desugar $ parse $ lexer combZ) []
-- Punto de entrada principal
main :: IO ()
main =
  do
  putStrLn "\nBienvenido a MiniLisp. "
  putStrLn "Escriba (exit) para salir."
  minilisp
-- Bucle principal del interprete
minilisp =
  do
    putStr "[MiniLisp]> "
    str <- getLine
    if str == ""
      then minilisp
      else if str == ":q"
           then putStrLn "Bye :)"
      else do
      run str
      minilisp
-- Envuelve la evaluacion con manejo de errores
run :: String -> IO ()
run input =
  catch
    (do
      let tokens = lexer input
      let asa = parse tokens
      let ast = desugar asa
      let asv = evalS (ast) [("Z", z)]
      putStrLn (saca asv))
    errors
-- Manejador de errores
errors :: SomeException -> IO ()
errors e = putStrLn $ "[Error]: " ++ show e
\end{lstlisting}

Importamos todos los módulos a usar en el proyecto. Definimos el combinador \textbf{Z} y aplicamos us evaluación como es requerido para la recursión en\minilisp{-0.2cm}.

\noindent
El punto de entrada pincipal del lenguaje es la función \texttt{main}, donde damos la bienvenida al usuario y comnzamos el intérprete interactivo de\minilisp{-0.2cm}. De este modo hacemos lo siguiente para correr el proyecto:

\begin{minted}[fontsize=\small, bgcolor=black!5, frame=single]{text}
  $ ghci
  GHCi, version 9.4.5: https://www.haskell.org/ghc/  :? for help
  ghci> :l MiniLisp.hs
  ghci> main

  Bienvenido a MiniLisp. 
  Escriba (exit) para salir.
  [MiniLisp]> 
\end{minted}

Como primer instancia, definimos y evaluamos el \textbf{combinador Z}. Este combinador es una herramienta fundamental para implementar la recursión en \minilisp{}, ya que el lenguaje carece de recursión directa a nivel del núcleo. El resultado de evaluar el combinador se guarda en la variable \texttt{z}, que se introduce en el ambiente inicial de evaluación. Así, los programas que usan \texttt{letrec} pueden implementar funciones recursivas correctamente.

\noindent
El bucle principal del intérprete que, en una arrebato increíble de originalidad lo llamamos \texttt{minilisp}, es donde leemos la entrada del usuario, y la pasamos a una función \texttt{run} para que sea procesada. En este bucle antes de mandar la cadena a ser evaluada comprobamos que si es vacía o la cadena reservada para salir del intérprete.

\noindent
La función \texttt{run} es la que coordina todas las fases de análisis, evaluación y muestreo al usuario. Desde iniciar el proceso pasando como argumento la cadena recibida al \texttt{Lexer}, como su evaluación en \texttt{AST} (con \texttt{EvalS}) dentro de un ambiente inicial que contiene la definición del combinador \textbf{Z}, necesario para la recursión. Esta función mete de inicio a ese ambiente, la variable \textbf{Z} con su respectiva evaluación. \texttt{run} se ejecuta dentro de un bloque \texttt{catch} permite capturar cualquier excepción que ocurra durante el análisis o la evaluación, evitando que el intérprete se detenga ante un error. En su lugar, se muestra un mensaje informativo en pantalla y el programa continúa su ejecución de forma segura.\\

Finalmente, para la visualización de los resultados, nos hace falta explicar la función \texttt{saca}. Esta función es una función auxiliar que se encarga de mostrar el resultado de la evaluación al usuario.  
Esto es necesario porque los valores evaluados en \minilisp{} se representan mediante constructores internos del tipo \texttt{ASV}, los cuales no son legibles directamente. La función \texttt{saca} transforma estos valores en una representación textual clara y amigable para el usuario.\\

La función \texttt{saca} queda implementada en el archivo \texttt{Saca.hs} como sigue:

\begin{lstlisting}[style=haskellstyle, caption={Procedimiento \texttt{saca} para mostrar el resultado al usuario}]
module Saca where

import ASV

--Funcion para obtener el resultado e imprimirlo como cadena y no como tipo de dato ASV
saca :: ASV -> String
saca (NiV) = "[]"
saca (NumV n) = show n
saca (BoolV b)
  | b == True = "#t"
  | otherwise = "#f"
saca (ClosureF p c e) = "#<procedure>"
saca (ConV f s) = "[" ++ sacaElems (ConV f s) ++ "]"
saca (PairV f s) = "(" ++ saca f ++ "," ++ saca s ++ ")"
saca _ = "#<unknown>"

-- Funcion auxiliar para recorrer ConV
sacaElems :: ASV -> String
sacaElems NiV = "[]"
sacaElems (ConV x xs) = saca x ++ ", " ++ sacaElems xs
sacaElems x = saca x
\end{lstlisting}

Nos apoyamos de \textit{pattern matching} para proceder con los casos correspondientes, estos son solo los valores canónicos:

\begin{itemize}
\item \texttt{NiV}: devolvemos la lista vacía y nada más.
\item \texttt{NumV}: devolvemos el valor $n$ que guarda \texttt{NumV}, este ya es un número en Haskell y utilizamos \texttt{show} para mostrarlo en su representación de salida.
\item \texttt{BoolV}: comprobamos que valor tiene $b$. Si es \texttt{True} devolvemos la cadena "$\#t$", en otro caso "$\#f$" para \texttt{False}.
\item \texttt{ClosureF}: si el valor es una una cerradura, no se imprime su contenido interno sino que se representa como \texttt{$\#$<procedure>}. Esto es un indicador textual, no un valor de lenguaje; se usa para que el usuario sepa que el valor es una función, pero no puede imprimirse directamente,  no es un valor real del lenguaje, sino una representación simbólica para el usuario.
\item \texttt{ConV}: cuando caemos en este valor, quiere decir que debemos representarlo como listas. Para ello nos auxiliaremos en la función \texttt{sacaElems} para formar una cadena con el formato adecuado y así devolver una representación fiel de los elementos.\\

  Esta función \texttt{sacaElems} recorre el encadenamiento de \texttt{ConV}, si hay dos elementos representamos estos recursivamente con la función\texttt{saca} separados por comas, cuando hemos llegado al último elemento, lo regresamos tal cual.
  
\item \texttt{PairV}: es más simple que \texttt{ConV}, solo es representar dos elementos recursivamente con \texttt{saca} entre paréntesis y separados por una coma.

\item Si el valor canónico \texttt{ASV} recibido no es ninguno de los anteriores, entonces este no es válido, cosa que no debería ni puede suceder pero lo ponemos por completitud.
\end{itemize}

Ya con esto, veamos los resultados finales del lenguaje\minilisp{-0.2cm}:

\begin{minted}[fontsize=\small, bgcolor=black!5, frame=single]{text}
[MiniLisp]> (+ (* 2 3 -1) (- 10 4 9 -22) (sqrt 16) (expt 36) (/ 1 2))
1331
\end{minted}

Comprobemos arugmento por argumento que la expresion anterior es correcta:

\begin{minted}[fontsize=\small, bgcolor=black!5, frame=single]{text}
[MiniLisp]> (* 2 3 -1)
-6
[MiniLisp]> (- 10 4 9 -22)
37
[MiniLisp]> (sqrt 16)
4
[MiniLisp]> (expt 36)
1296
[MiniLisp]> (/ 1 2)
0
[MiniLisp]> 
\end{minted}

Al sumar los resultado se puede apreciar que los valores coinciden.

\begin{minted}[fontsize=\small, bgcolor=black!5, frame=single]{text}
[MiniLisp]> (+ -6 37 4 1296 0)
1331
\end{minted}

Veamos otro ejemplo:

\begin{minted}[fontsize=\small, bgcolor=black!5, frame=single]{text}
[MiniLisp]> (let ((y 31) (x -22)) (cond [(< x 0 y) (- 0 x y)] [(= x 0) 0]
                                        [else (+ x 1)]))
53
\end{minted}

Aquí, al resolver paso por paso, tenemos que se asigna $31$ a $y$ y $-22$ a $x$, por lo que la primer comprobación debería ser correcta al sustituir las variables:

\begin{minted}[fontsize=\small, bgcolor=black!5, frame=single]{text}
[MiniLisp]> (< -22 0 31)
#t
\end{minted}

Por lo que se resuelve su cláusula:

\begin{minted}[fontsize=\small, bgcolor=black!5, frame=single]{text}
[MiniLisp]> (- 0 -22 31)
53
\end{minted}

Por último veamos un ejemplo de evaluación para listas, se espera que cada elemento de la lista se evalúe, regresando una lista con valores evaluados. Hacemos el recordatorio también de que nuestras listas son heterogéneas:

\begin{minted}[fontsize=\small, bgcolor=black!5, frame=single]{text}
[MiniLisp]> [[1, 2], (3, 4), (not #t), (+ 1 2 3 4 5 6), (!= 9 7 3 5),
             [], (/ 2 4), (not (not (= 9 9))), (tail [1, 2, 3]),
             ((lambda (f x) (f x)) (lambda (y) (* y y)) 4)]
[[1, 2], (3,4), #f, 21, #t, [], 0, #t, 3, 16]
\end{minted}

Al evaluar cada elemento por separado, se puede apreciar que estos resultados coinciden:

\begin{minted}[fontsize=\small, bgcolor=black!5, frame=single]{text}
[MiniLisp]> [1, 2]
[1, 2]
[MiniLisp]> (3, 4)
(3,4)
[MiniLisp]> (not #t)
#f
[MiniLisp]> (+ 1 2 3 4 5 6)
21
[MiniLisp]> (!= 9 7 3 5)
#t
[MiniLisp]> []
[]
[MiniLisp]> (/ 2 4)
0
[MiniLisp]> (not (not (= 9 9)))
#t
[MiniLisp]> (tail [1, 2, 3])
3
[MiniLisp]> ((lambda (f x) (f x)) (lambda (y) (* y y)) 4)
16
\end{minted}

En el archivo \texttt{README.md} se incluyen ejemplos más detallados para cada expresión.
%%%%%%%%%%%%%%%%%%%%%%%%%%%%%%%%%

\section{Funciones de prueba}

En esta sección implementamos tres funciones especiales: la suma de los primeros $n$ números naturales, el factorial de un número y el n-ésimo número de Fibonacci.
Estas funciones nos sirven como ejemplos prácticos para probar las capacidades del lenguaje\minilisp{-0.2cm} y en especial, de su capacidades recursivas.

\noindent
Debido a que cada una de estas funciones se define naturalmente mediante recursión, nos auxiliamos de la expresión \texttt{letrec}, con la cual definimos funciones recursivas dentro del lenguaje.\\

Para poder ejecutar estas funciones directamente desde el menú interactivo, modificamos la función \texttt{run} de nuestro archivo \texttt{MiniLisp.hs}.
Agregamos lógica que detecta cuando la entrada del usuario comienza con alguna palabra clave: \texttt{fact}, \texttt{sum} o \texttt{fibo}.

\[
\text{\texttt{fact}(\texttt{Int})} \quad \text{\texttt{sum}(\texttt{Int})} \quad \text{\texttt{fibo}(\texttt{Int})}
\]

Cuando se detecta uno de estos comandos, se genera dinámicamente una expresión \texttt{MiniLisp} equivalente que utiliza \texttt{letrec} y luego se evalúa normalmente.

\begin{lstlisting}[style=haskellstyle, caption={Implementación de las funciones especiales de\minilisp}]
-- Envuelve la evaluacion con manejo de errores
run :: String -> IO ()
run input =
  catch
    (do
        expr <-
          if "fact" `isPrefixOf` input
          then return $ fact (read (last (words input)))
          else if "sum" `isPrefixOf` input
               then return $ sumSum (read (last (words input)))
          else if "fibo" `isPrefixOf` input
               then return $ fibo (read (last (words input)))
          else return input 
      
        let tokens = lexer expr
        let asa = parse tokens
        let ast = desugar asa
        let asv = evalS (ast) [("Z", z)]
        putStrLn (saca asv))
    errors
\end{lstlisting}

La función \texttt{isPrefixOf} de Haskell, nos permite verificar si la cadena introducida por el usuario comienza con un determinado prefijo. De esta forma, si el usuario escribe \texttt{fact 5}, el intérprete genera internamente el código \texttt{MiniLisp} equivalente y lo evalúa como si el usuario lo hubiese escrito explícitamente.

%%%%%%%%%%%%%%%%%%%%%%%%%%%%%%%%%

\subsection{Suma primeros $n$ números naturales}

Para calcular la suma de los primeros $n$ números naturales se utiliza una definición recursiva simple:

\[
\texttt{sum}(n) =
\begin{cases}
0, & \text{si } n {=} 0,\\[2pt]
n {+} \texttt{sum}(n-1) & \text{en otro caso}
\end{cases}
\]

\bigskip

La implementación en Haskell genera una expresión en \minilisp con \texttt{letrec} que define y ejecuta esta función:

\begin{lstlisting}[style=haskellstyle, caption={Función para obtener el factorial de un número con \texttt{letrec}}]
--Generamos la suma de los primeros n numeros con letrec
sumSum :: Int -> String
sumSum n = "(letrec (sum (lambda (n) (if0 n 0 (+ n (sum (- n 1)))))) (sum " ++ show n ++ "))"
\end{lstlisting}

Algunos ejemplos de ejecución son:
\begin{minted}[fontsize=\small, bgcolor=black!5, frame=single]{text}
[MiniLisp]> sum 15
120
[MiniLisp]> sum 5
15
[MiniLisp]> sum 50
1275
[MiniLisp]> sum 3
6
\end{minted}

El intérprete convierte internamente esta entrada en la expresión:
\[
\texttt{(letrec (sum (lambda (n) (if0 n 0 (+ n (sum (- n 1)))))) (sum 5))}
\]

%%%%%%%%%%%%%%%%%%%%%%%%%%%%%%%%%

\subsection{Factorial}

De manera análoga, para el factorial usamos la definición recursiva clásica:

\[
\texttt{fact}(n) =
\begin{cases}
1, & \text{si } n {=} 0,\\[2pt]
n \times \texttt{fact}(n-1) & \text{en otro caso}
\end{cases}
\]

\bigskip

La función generadora en Haskell construye la expresión correspondiente:

\begin{lstlisting}[style=haskellstyle, caption={Función para obtener la suma de los primeros $n$ números naturales con \texttt{letrec}}]
--Generamos la funcion factorial con letrec
fact :: Int -> String
fact n = "(letrec (fact (lambda (n) (if0 n 1 (* n (fact (- n 1)))))) (fact " ++ show n ++ "))"
\end{lstlisting}

Algunos ejemplos de ejecución serían:
\begin{minted}[fontsize=\small, bgcolor=black!5, frame=single]{text}
[MiniLisp]> fact 5
120
[MiniLisp]> fact 8
40320
[MiniLisp]> fact 3
6
[MiniLisp]> fact 7
5040
\end{minted}

Donde la entrada \texttt{fact 7} por ejemplo, se traduce internamente como:
\[
\texttt{(letrec (fact (lambda (n) (if0 n 1 (* n (fact (- n 1)))))) (fact 5))}
\]

%%%%%%%%%%%%%%%%%%%%%%%%%%%%%%%%%

\subsection{Fibonacci}

La sucesión de Fibonacci se define recursivamente como:

\[
\texttt{fibo}(n) =
\begin{cases}
0, & \text{si } n {=} 0,\\
1, & \text{si } n {=} 1,\\
\texttt{fibo}(n-1) {+} \texttt{fibo}(n-2) & \text{en otro caso}
\end{cases}
\]

\bigskip

Su implementación en Haskell genera el código\minilisp{-0.2cm} correspondiente con \texttt{letrec}:

\begin{lstlisting}[style=haskellstyle, caption={Función para obtener el n-ésimo número de la suceción de Fibonacci con con \texttt{letrec}}]
--Generamos el n-esimo numero de Fibonacci con letrec
fibo :: Int -> String
fibo n = "(letrec (fib (lambda (n) (if0 n 0 (if0 (- n 1) 1 (+ (fib (- n 1)) (fib (- n 2))))))) (fib " ++ show n ++ "))"
\end{lstlisting}

Como ejemplos de ejecución tenemos:
\begin{minted}[fontsize=\small, bgcolor=black!5, frame=single]{text}
[MiniLisp]> fibo 12
144
[MiniLisp]> fibo 18
2584
[MiniLisp]> fibo 7
13
[MiniLisp]> fibo 2
1
\end{minted}

El intérprete genera internamente la siguiente definición:
\[
\texttt{(letrec (fib (lambda (n) (if0 n 0 (if0 (- n 1) 1 (+ (fib (- n 1)) (fib (- n 2))))))) (fib 7))}
\]

Estas funciones de prueba no sólo demuestran la expresividad del lenguaje donde el sistema de evaluación de\minilisp{-0.2cm} maneja correctamente la recursión, sino también que la implementación del combinador \textbf{Z} y las expresiones \texttt{letrec} permiten definir y evaluar funciones complejas sin necesidad de estructuras externas. Además, muestran cómo el menú interactivo puede extenderse para admitir comandos personalizados que simplifican la interacción del usuario con el intérprete.

%%%%%%%%%%%%%%%%%%%%%%%%%%%%%%%%%%

% ----- Conclusiones -----
%%%%%%%%%%%%%%%%%%%%%%%%%%%%%%%%%%
\chapter{Conclusiones}

\subsection*{Reflexión}
El desarrollo de \textbf{MINILISP} ha sido un ejercicio riguroso de diseño e implementación de un lenguaje de programación, cubriendo las etapas fundamentales de la teoría de lenguajes: desde la \textbf{Sintaxis Concreta} hasta la \textbf{Semántica Operacional}. La elección de \textbf{Haskell} y sus herramientas (\texttt{Alex} y \texttt{Happy}) fue muy útil, ya que la naturaleza funcional e inmutable del lenguaje anfitrión facilitó la construcción clara del modelo y la adaptación de conceptos abstractos, como los \textbf{Árboles de Sintaxis Abstracta (ASA)} y los estados finales.

El proyecto se caracteriza por un proceso de diseño bien estructurado que priorizó la claridad:
\begin{itemize}
    \item \textbf{Análisis Formal}: La definición de la \textbf{Sintaxis Concreta} mediante la gramática \textbf{EBNF} sentó una base sólida para el análisis léxico y sintáctico.
    \item \textbf{Gestión de la Complejidad (Azúcar Sintáctica)}: La etapa de \textbf{desazucarización} fue crucial, transformando construcciones expresivas pero complejas (como operadores variádicos, \texttt{let*}, \texttt{cond} y la aplicación de funciones n-arias) en un núcleo (\textbf{AST}) más simple y homogéneo. Esto redujo drásticamente el número de reglas necesarias para la semántica, haciendo el intérprete más limpio y fácil de verificar. La correcta desazucarización de operadores variádicos y comparadores a estructuras binarias anidadas, manteniendo la consistencia de tipos, fue un punto crítico.
    \item \textbf{Estrategia Semántica}: La transición inicial de la \textbf{Semántica Estructural (Paso Pequeño)} a la implementación de \textbf{evaluación perezosa (Paso Grande o Big-Step)} con \textbf{Puntos Estrictos} fue esencial para manejar la recursión de forma correcta a través del constructo \texttt{letrec} y el \textbf{Combinador Z}, superando el problema de la evaluación glotona inicial.
\end{itemize}

\subsection*{Limitaciones Encontradas}
A pesar de los logros, se identificaron áreas con limitaciones:
\begin{itemize}
    \item \textbf{Tipado y Verificación}: El lenguaje \textbf{carece de un sistema de tipos explícito} que verifique la corrección de los programas antes de la ejecución. Errores de tipo sólo se detectan durante la evaluación semántica (por ejemplo, intentar operar con un valor no numérico).
    \item \textbf{Pares y Listas}: Aunque se depuró la ambigüedad en la sintaxis de listas utilizando \texttt{Cons} y \texttt{Nil} frente a un simple encadenamiento de pares, la implementación del \texttt{TailV} en el intérprete requirió una función auxiliar (\texttt{tailDeep}) para forzar la evaluación perezosa en el segundo elemento de la lista anidada, lo que añade complejidad.
    \item \textbf{Ausencia del Combinador Z Explícito}: Si bien el concepto del \textbf{Combinador Z} fue fundamental para habilitar \texttt{letrec}, la solución final lo integra de manera implícita en la regla de \texttt{letrec} para la evaluación perezosa, sin una representación formal completamente visible en el ambiente como un valor canónico evaluado.
\end{itemize}

\subsection*{Posibles Extensiones Futuras}
El diseño modular de MINILISP permite varias extensiones:
\begin{itemize}
    \item \textbf{Sistema de Tipos Estático (Type Checker)}:
    Implementar un \textbf{Type Checker} para verificar la coherencia del programa antes de la evaluación. Esto podría basarse en reglas de inferencia para un subconjunto del lenguaje, mejorando la robustez y la detección temprana de errores.
    \item \textbf{Manejo de Cadenas (Strings)}:
    Extender el alfabeto y los tokens para soportar operaciones con cadenas de caracteres.
\end{itemize}

%%%%%%%%%%%%%%%%%%%%%%%%%%%%%%%%%%

% ----- Bibliografía -----
%%%%%%%%%%%%%%%%%%%%%%%%%%%%%%%%%%
\addcontentsline{toc}{chapter}{Bibliografía}
\begin{thebibliography}{99}
\bibitem{ref1} \textcolor{links}
Lee.D.(2014).\textit{Foundations Of Programming Languges}, Springer.

\bibitem{ref2} Aho, A. V., Lam, S. M., Sethi, R., \& Ullman, J. D. Compilers: Principles, Techniques, and Tools. [Second Edition]. 2007.

\bibitem{ref3} Disponible en:
Soto Romero, M. (2025). Sintaxis Concreta [Nota de clase de la Unidad 2: Especificación Formal de los Lenguajes de Programación]. Curso de Lenguajes del Programación, Facultad de Ciencias, Universidad Nacional Autónoma de México.
\textcolor{links}{\uline{\url{https://lambdasspace.github.io/LDP/notas/ldp_n04.pdf}}}

\bibitem{ref4} Disponible en:
Soto Romero, M. (2025).Sintaxis Abstracta [Nota de clase de la Unidad 2: Especificación Formal de los Lenguajes de Programación]. Curso de Lenguajes del Programación, Facultad de Ciencias, Universidad Nacional Autónoma de México.
\textcolor{links}{\uline{\url{https://lambdasspace.github.io/LDP/notas/ldp_n05.pdf}}}

\bibitem{ref5} Haskell.org. (s.f.). Documentación Haskell. Disponible en \textcolor{links}{\uline{\url{https://www.haskell.org}}}

\bibitem{ref6} Documentación Alex(Haskell) The Alex Lexer Generator for Haskell Programming in Haskell (Graham Hutton, 2nd Edition). Sección sobre parsers y lexers. Disponible en: \textcolor{links}{\uline{\url{https://www.haskell.org/alex/}}}

\bibitem{ref7} Marlow, S., Gill, A. (2009). Happy. Disponible en: \textcolor{links}{\uline{\url{https://www.haskell.org/happy/}}}

\bibitem{ref8} Soto Romero, M. (2025). Expresiones let [Nota de clase de la Unidad 2: Especificación Formal de los Lenguajes de Programación]. Curso de Lenguajes del Programación, Facultad de Ciencias, Universidad Nacional Autónoma de México.
\textcolor{links}{\uline{\url{https://lambdasspace.github.io/LDP/notas/ldp_n08.pdf}}}

\bibitem{ref9} Soto Romero, M. (2025). El Cálculo lambda como núcleo para funciones [Nota de clase de la Unidad 2: Estilo Funcional]. Curso de Lenguajes del Programación, Facultad de Ciencias, Universidad Nacional Autónoma de México.
\textcolor{links}{\uline{\url{https://lambdasspace.github.io/LDP/notas/ldp_n09.pdf}}}

\bibitem{ref10} Soto Romero, M. (2025). Expresiones lambda [Nota de clase de la Unidad 2: Estilo Funcional]. Curso de Lenguajes del Programación, Facultad de Ciencias, Universidad Nacional Autónoma de México.
\textcolor{links}{\uline{\url{https://lambdasspace.github.io/LDP/notas/ldp_n10.pdf}}}

\bibitem{ref11} Documentación Racket. Disponible en: \textcolor{links}{\uline{\url{https://docs.racket-lang.org/reference/let.html}}}

\bibitem{ref12} LispWorks. (2005). Special Operator LET, LET*. Common Lisp HyperSpec. \textcolor{links}{\uline{\url{https://www.lispworks.com/documentation/HyperSpec/Body/s_let_l.htm}}}

\bibitem{EFLP-SC}
Soto Romero, M. (2025). Sintaxis Concreta [Nota de clase de la Unidad 2: Especificación Formal de los Lenguajes de Programación]. Curso de Lenguajes del Programación, Facultad de Ciencias, Universidad Nacional Autónoma de México.
\textcolor{links}{\uline{\url{https://lambdasspace.github.io/LDP/notas/ldp_n04.pdf}}}

\bibitem{HyU}
Hopcroft.J.E.,Ullman.J.D.(1939),
\textit{Introduction to Automata Theory, Languages, and Computation},Library of Congress Cataloging-in-Publication Data.

\bibitem{landin}
Landin.P.J.(1965)\textit{The Next 700 Programming Languages}.Univac Division of Sperry Rand Corp,9(3),157-166.

\bibitem{winskel}
Winske.G.(1993)
\textit{The Formal Semantics of Programming Languages}, Massachusetts Institute of Technology.

\bibitem{Plotkin}
Plotkin.G.D.(1981)
\textit{A Structural Approach to Operational Semantics}, University of Aarhus.

\bibitem{N14}
Soto Romero, M. (2025). Estrategias de evaluación [Nota de clase de la Unidad 2: Estilo Funcional]. Curso de Lenguajes del Programación, Facultad de Ciencias, Universidad Nacional Autónoma de México.
\textcolor{links}{\uline{\url{https://lambdasspace.github.io/LDP/notas/ldp_n14.pdf}}}

\bibitem{N15}
Soto Romero, M. (2025). Recursión [Nota de clase de la Unidad 2: Estilo Funcional]. Curso de Lenguajes del Programación, Facultad de Ciencias, Universidad Nacional Autónoma de México.
\textcolor{links}{\uline{\url{https://lambdasspace.github.io/LDP/notas/ldp_n15.pdf}}}

\bibitem{iso14977}
International Organization for Standardization. (1996). 
Information technology — Syntactic metalanguage — Extended BNF
(ISO/IEC Standard No. 14977).

\bibitem{PierceBC}
Pierce, B. C. (2002). Types and programming languages. The MIT Press.

\bibitem{huttonHaskell}
Programming in Haskell,Hutton, Graha.(2016), Cambridge University Press.

\bibitem{Baren-Lambda}
 North-Holland, (1984).The Lambda Calculus: Its Syntax and Semantics (2nd ed., pp. 152-154)North-Holland.

\bibitem{Friedman1996}
The Little Schemer, Friedman, Daniel P. and Felleisen, Matthias, (1996). MIT Press, Cambridge, MA.

\end{thebibliography}

%%%%%%%%%%%%%%%%%%%%%%%%%%%%%%%%%%

\end{document}
%-------------------------------
