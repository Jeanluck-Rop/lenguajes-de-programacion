\chapter{Introducción}
%%%%%%%%%%%%%%%%%%%%%%%%%%%%%%%%%
\section{Motivación}

El estudio de los lenguajes de programación es uno de los pilares fundamentales de las ciencias de la computación. Comprender cómo se definen, interpretan y ejecutan permite no solo utilizarlos como herramientas, sino también analizarlos y extenderlos desde la perspectiva formal. En este contexto, la implementación de un lenguaje estilo \textbf{Lisp} ofrece un terreno fértil para explorar la relación entre teoría y práctica: su sintaxis minimalista, su semántica clara y su estructura recursiva facilitan su análisis formal.

El proyecto \textbf{MiniLisp} surge con la intención de integrar los conceptos teóricos vistos en clase dentro de una implementación concreta en \textbf{Haskell}. De este modo, este trabajo busca fortalecer la comprensión del proceso completo de formalización de un lenguaje de programación: desde la definición léxica y gramatical, hasta la evaluación de programas mediante un intérprete funcional.

%%%%%%%%%%%%%%%%%%%%%%%%%%%%%%%%%

\section{Objetivos}

\subsubsection{Objetivo general}
Desarrollar e implementar una versión extendida del lenguaje \textbf{MiniLisp} que formalice su sintaxis y semántica, y que permita ejecutar programas a través de un intérprete en \textbf{Haskell}, manteniendo coherencia entre el modelo teórico y la implementación práctica.

\subsubsection{Objetivos específicos}
\begin{enumerate}
    \item Definir formalmente la sintaxis léxica y libre de contexto del lenguaje, incluyendo operadores, estructuras de control y mecanismos de definición local.
    \item Implementar un analizador léxico y un analizador sintáctico utilizando las herramientas \textbf{Alex} y \textbf{Happy}, respectivamente.
    \item Diseñar la sintaxis abstracta usando dos niveles, un \textbf{ASA} con azúcar sintáctica y \textbf{AST} como núcleo del lenguaje.
    \item Desarrollar un módulo de eliminación de azúcar sintáctica (\textbf{Desugar}) que traduzca expresiones superficiales a su representación mínima.
    \item Implementar un intérprete funcional (\textbf{Interp}) basado en la semántica operacional, utilizando ambientes y \textbf{evaluación ansiosa} (eager evaluation).
    \item Proveer una interfaz interactiva que permita ejecutar programas escritos en la sintaxis concreta de \textbf{MINILISP}.
\end{enumerate}

%%%%%%%%%%%%%%%%%%%%%%%%%%%%%%%%%

\section{Delimitación del Proyecto}

El proyecto se reduce al diseño e implementación de un subconjunto de \textbf{Lisp} llamado \textbf{MINILISP}, con el propósito de estudiar los principios fundamentales de los lenguajes funcionales y su formalización. Por lo tanto:
\begin{itemize}
    \item No se abordará la \textbf{gestión de tipos} ni el \textbf{análisis estático}.
    \item El sistema de evaluación se restringe a la \textbf{evaluación ansiosa} (eager evaluation).
    \item La semántica implementada se limita al \textbf{nivel estructural}, sin considerar aspectos de optimización, compilación ni concurrencia.
    \item El alcance del proyecto comprende la construcción del \textbf{intérprete}, no un compilador ni un entorno gráfico.
\end{itemize}
