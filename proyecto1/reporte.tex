\documentclass[12pt,letterpaper]{book}

\usepackage[utf8]{inputenc}
\usepackage[T1]{fontenc}
\usepackage[spanish]{babel}
\usepackage{listings, float, xcolor}
\usepackage{algorithm,algpseudocode, chngcntr}
\usepackage{graphicx, enumitem, geometry}
\usepackage{amsmath, amssymb, amsthm}

\usepackage[most]{tcolorbox}
\usepackage{titlesec}
\usepackage{lipsum}

\geometry{margin=2.5cm}

\newcommand{\imp}{\rightarrow}
\newcommand{\vep}{\varepsilon}
\definecolor{azulin}{HTML}{130F87}
\definecolor{grisin}{HTML}{6D6C91}
\renewcommand{\lstlistingname}{Código}

%------ Definimos los colores para la sintaxis del código --------%
\definecolor{keywordcolor}{rgb}{0.5, 0.0, 0.5}  % Morado para palabras clave
\definecolor{commentcolor}{rgb}{0.25, 0.5, 0.35} % Verde para comentarios
\definecolor{stringcolor}{rgb}{0.88, 0.68, 0.18}  % Mostaza anaranjado para los strings
\definecolor{backgroundcolor}{rgb}{0.95, 0.95, 0.95} % Gris claro para fondo

%------ Configuración para mostrar código en Python --------%
\lstdefinestyle{pythonstyle}{
  language=Python,
  basicstyle=\ttfamily\footnotesize,
  keywordstyle=\color{keywordcolor}\bfseries,
  commentstyle=\color{commentcolor}\itshape,
  stringstyle=\color{stringcolor},
  numbers=left,
  numberstyle=\tiny,
  stepnumber=1,
  numbersep=8pt,
  backgroundcolor=\color{backgroundcolor},
  tabsize=4,
  showspaces=false,
  showstringspaces=false,
  breaklines=true,
  frame=single,
  captionpos=b
}

\titleformat{\chapter}[display]
  {\bfseries\huge} % formato de letra
  {Capítulo \thechapter} % lo que aparece arriba
  {0.5ex} % espacio entre número y título
  {\Huge} % formato del título

\begin{document}

\begin{titlepage}
  \thispagestyle{empty}
  %--------------- COMANDO LINEA----------------->
  \newcommand{\linea}{\rule{\linewidth}{0.5mm}}                 
  \center
  %<<<<<<<<<<<<<<<<<<<<<<<<<<<<<<<<<<<<<<<<<<<

  %Add logos
  \includegraphics[width=0.24\textwidth]{../../unam_logo.png} \hspace{1.5cm}
  \includegraphics[width=0.25\textwidth]{../../fc_logo.png} \\[1cm]
  \textbf{\Large UNIVERSIDAD NACIONAL AUTÓNOMA}\\[3mm]
  \textbf{\Large DE MÉXICO} \\[6mm]
  \textit{\Large Facultad de Ciencias}

  \vfill

  %---------------TÍTULO----------------->
  \linea
  \vfill \vspace{1cm}
  \textbf{\Large Lenguajes de Programación}\\[1.1cm]
  \textbf{\LARGE M\large INI\LARGE L\large ISP}\\[0.7cm]
  \linea \\
  \vfill
  \vspace{0.8cm}
  %Title of the Research
  \textbf{\LARGE  Proyecto 1}\\[6mm]
  %<<<<<<<<<<<<<<<<<<<<<<<<<<<<<<<<<<<<<<<<<<<

  %Team
  \vfill
  \textit{\small Presenta:}\\
  \textbf{\large  \textbf{Lugo Díaz Ordaz Gretel Alexandra}}\\[3mm]
  \textbf{\large  \textbf{Ramírez Juárez María Fernanda}}\\[3mm]
  \textbf{\large  \textbf{Rojo Peña Manuel Ianluck}}\\[4mm]

  %Profesor y Grupo
  \textit{\small Profesor:}\\
  \textbf{Manuel Soto Romero}\\[1.5mm]
  \textit{\small Ayudantes:}\\
  \textbf{Diego Méndez Medina}\\[1mm]
  \textbf{Erick Daniel Arroyo Martínez}\\[1mm]
  \small Grupo: 7121, 2026-1\\[3mm]
  \vfill

  %Fecha or Date
  \textit{\small Fecha de entrega:}\\
         {\large 3 de noviembre, 2025}
         
\end{titlepage}
 % Importa el archivo portada.tex

\tableofcontents  % genera el índice

\chapter{Introducción}
\noindent
Definimos la Gramática para \large M\small INI\large L\small ISP \normalsize en \textbf{EBNF}:

\begin{tcolorbox}[colback=azulin!5!white, colframe=azulin!80, title=Gramática MINILISP]
  \[
  \renewcommand{\arraystretch}{1.1}
  \begin{array}{rcl}
    Expr &::=& Var \\
    &\mid& Num \\
    &\mid& Bool \\
    &\mid& (+\ Expr\ \{Expr\}) \\
    &\mid& (-\ Expr\ \{Expr\}) \\
    &\mid& (*\ Expr\ \{Expr\}) \\
    &\mid& (/\ Expr\ \{Expr\}) \\
    &\mid& (add1\ Expr) \\
    &\mid& (sub1\ Expr) \\
    &\mid& (sqrt\ Expr) \\
    &\mid& (expt\ Expr) \\
    &\mid& (not\ Expr) \\
    &\mid& (=\ Expr\ \{Expr\}) \\
    &\mid& (<\ Expr\ \{Expr\}) \\
    &\mid& (>\ Expr\ \{Expr\}) \\
    &\mid& (<=\ Expr\ \{Expr\}) \\
    &\mid& (>=\ Expr\ \{Expr\}) \\
    &\mid& (!=\ Expr\ \{Expr\}) \\
    &\mid& (Expr,\ Expr) \\
    &\mid& (fst\ Expr) \\
    &\mid& (snd\ Expr) \\
    &\mid& (let\ (Var\ Expr)\ \{Expr\}) \\
    &\mid& (letrec\ (Var\ Expr)\ \{Expr\}) \\
    &\mid& (let*\ (Var\ Expr)\ \{Expr\}) \\
    &\mid& (if0\ Expr\ Expr\ Expr) \\
    &\mid& (if\ Expr\ Expr\ Expr) \\ 
    &\mid& (lambda\ (\{Var\})\ \{Expr\}) \\
    &\mid& (Expr\ Expr) \\
    &\mid& [\ \{Expr,\}\ ] \\
    &\mid& (head\ Expr) \\
    &\mid& (tail\ Expr) \\
    &\mid& (cond\ [Expr\ Expr]\ \{[Expr\ Expr]\}\ [else\ Expr]) \\
    Var &::=& \textit{Identificador de variable} \\
    Num &::=& \textit{Constante entera} \\
    Bool &::=& \#t \mid \#f
  \end{array}
  \]
\end{tcolorbox}

\chapter{Principios básicos}
\section{Algoritmos y computadoras}
\subsection{Breve historia del concepto de algoritmo}


\section{Representación de la información}
\subsection{Sistemas numéricos}

\end{document}
